%------------------------------------------------------------------------------%
%
%
%------------------------------------------------------------------------------%

%%%%%%%%%%%%%%%%%%%%%%%%%%%%%%%%%%%%%%%%%%%%%%%%%%%%%%%%%%%%%%%%%%%%%%%%%%%%%%%%
% Document and encoding 
\documentclass[11pt,a4paper]{article}

%%%%%%%%%%%%%%%%%%%%%%%%%%%%%%%%%%%%%%%%%%%%%%%%%%%%%%%%%%%%%%%%%%%%%%%%%%%%%%%%
% Local fonts, styles, packages, and references.
\usepackage{local_doc}
%\usepackage{jupyter}

\addbibresource{references.bib}

%%%%%%%%%%%%%%%%%%%%%%%%%%%%%%%%%%%%%%%%%%%%%%%%%%%%%%%%%%%%%%%%%%%%%%%%%%%%%%%%
% Title info
\author{Stuart Kingham: ID 21014912}
\title{21014912 MSc Dissertation}
\date{\today}

%%%%%%%%%%%%%%%%%%%%%%%%%%%%%%%%%%%%%%%%%%%%%%%%%%%%%%%%%%%%%%%%%%%%%%%%%%%%%%%%
%
\begin{document}
	\doublespacing
	
	\begin{titlepage}
		\topskip0pt
		\vspace*{\fill}
		\begin{center}
			\vspace*{1cm}
			
			{\LARGE CT7P01 MSc Project}
			
			
			\vspace*{1cm}
			{\large \textbf{Preparing Cryptographers for the Quantum Era}}
			
			\vspace{0.2cm}
			{\large Discussion points}
			
			%\vspace{1.5cm}
			
			\vfill
			
			\textbf{Stuart Kingham: ID 2101491}
			
			\vfill
			
			\vspace{0.8cm}
			
			%\includegraphics[width=0.4\textwidth]{university}
			
			Supervisor: Graham Taylor-Russel \\
			MSc in Cryptography \\
			School of Computing and Digital Media\\
			London Metropolitan University\\
			\date{\today}
			%Jan 06, 2025
			
		\end{center}
		\vspace*{\fill}
	\end{titlepage}
	
	\pagebreak
	
	%\begin{abstract}
	

%Give the recent advancement and ubiquity of quantum platforms and advanced tooling,
%an introductory course in quantum computation and quantum information 
%is not only an achievable objective for masters and undergraduate students, 
%but desirable for the democratisation of these new technological advances.

Recent progress in cloud-based quantum hardware, high-fidelity simulators and accessible SDKs 
has lowered the barrier to hands-on quantum system experimentation, making an introductory course in quantum computation 
and information both feasible for undergraduate and master's cohorts and essential for the 
democratisation of this emerging technology.

This research project seeks to deliver a comprehensive framework for teaching 
introductory quantum computing algorithms, focusing on the core foundational knowledge needed,
along with a targeted delivery of topics that resonate with students of mathematics, cryptography and data science.

This paper develops a program using Software Development Kits (SDKs) from leading quantum hardware and software companies
that deliver world class quantum systems, emulators and quantum simulators via cloud environments.
This work will identify the essential learning outcomes required for new researchers to become proficient
in constructing quantum circuits of increasing complexity.

%This project develops a teaching framework that couples core mathematical foundations with carefully selected topics designed to resonate with students of mathematics, cryptography and data science. Leveraging industry-standard toolchains of IBM\,Qiskit, Google\,Cirq, Pennylane and D-Wave\,Leap, the curriculum guides learners from single-qubit gates to the construction of increasingly sophisticated circuits on real or emulated hardware.

Starting with the core mathematical concepts, we present the building blocks of current quantum 
algorithms through the lens of Shor's landmark 1990's quantum algorithm for integer factorisation and discrete logarithms.
Using these well understood and demonstrable building blocks, this work then introduces more advanced topics
in a manner that new students will feel confident in further research and work in this exciting industry.

% Shor's landmark algorithm for integer factorisation and discrete logarithms provides the organising lens: its component techniques (state preparation, modular arithmetic, Quantum Fourier Transform and amplitude amplification) are unpacked, demonstrated on contemporary NISQ devices, and then extended to more advanced themes such as block encoding, quantum machine-learning kernels and hybrid classical-quantum workflows.

The programme focuses on using an Outcomes Based Learning (OBL) approach that should give students: 
a practical, hands-on, pathway into the field of quantum computing; 
understanding the limitations of delivering solutions using noisy hardware;
to critically evaluate claims of quantum advantage in current literature; and,
to gain the skills to compete in this rapidly evolving and exciting area of research.

%Explicit learning outcomes ensure graduates can (i) build and execute quantum circuits, (ii) articulate the limitations of noisy hardware, and (iii) critically evaluate claims of quantum advantage in current literature. The framework thus offers a practical pathway for new entrants to gain the competencies required to contribute meaningfully to this rapidly evolving field.


%We will perform a survey of available quantum computing platforms, evaluating them for their usability, scalability
%and suitability for cryptographic applications.  We will look to implement key quantum algorithms used in
%cryptanalysis (Shor, Grover, Quantum phase estimation algorithm (QPE), Quadratic Unconstrained Binary Optimization
%(Ising-QUBO), etc.), as well as introducing the supporting quantum principles and
%mathematical models that underpin these algorithms (Hidden-subgroup problems, combinatorics, optimizations, etc.).
%Further, we will look to see examples of how researchers are applying quantum techniques to attempt to solve the
%Shortest Vector Problem (SVP) and to attack Substitution-Permutation Networks (SPN), which underpins AES symmetric
%encryption.
%Some approaches, such as Quantum Annealing and Coherent Ising Machines, are claiming to show evidence of quantum
%advantage and are of great interest.

%The expectation of this project is to offer a practical pathway for new entrants to the field of quantum computing
%to gain the skills to compete in this rapidly evolving and exciting area of research.
\end{abstract}

	
	\newpage
	
	\singlespacing 
	\tableofcontents
	\listoffigures
	%\listoftables
	\newpage
	
	\doublespacing
	\pagenumbering{roman}
	\pagenumbering{arabic}

\section{Note on Analogue Quantum Computing}

The user wants to understand the difference between two types of systems. One is analogue quantum simulators, like cold atom or trapped ion systems, simulating models like the Ising or Hubbard model. The other is D-Wave's quantum annealers, which solve QUBO problems using parameterized solvers. The primary difference lies in the control parameters and how the Hamiltonian is mapped—D-Wave fits QUBO onto the Ising model, while analogue systems emulate physical Hamiltonians directly. I'll answer using LaTeX for precision, sticking with the established style.

\subsubsection{Analogue Quantum Simulators vs.\ D‑Wave‐style QUBO Annealers}

\paragraph{Core Distinction}
\begin{itemize}
	\item \textbf{Analogue quantum simulation} seeks to \emph{mimic the dynamics} of a target Hamiltonian $H_{\text{target}}$ by engineering a controllable device whose \emph{native Hamiltonian} $H_{\text{device}}(\boldsymbol{\lambda})$ approximates it for some knob setting $\boldsymbol{\lambda}^\star$.  The output of interest is typically time‑dependent correlation functions or phase diagrams of many‑body systems (e.g.\ Fermi–Hubbard, lattice gauge theories).
	\item \textbf{D‑Wave QUBO machines} implement a \emph{parameterised Ising Hamiltonian}
	\[
	H_{\text{QUBO}} = \sum_i h_i \sigma_i^z + \sum_{i<j} J_{ij}\,\sigma_i^z\sigma_j^z,
	\]
	whose ground state encodes a quadratic‐unconstrained binary optimization (QUBO) instance.  The goal is to \emph{find a low‑energy bitstring}, not to replicate real‑world physics.
\end{itemize}

\paragraph{Practical Axis‐By‐Axis Contrast}
\begin{enumerate}
	\item \textbf{Scientific objective}  
	\begin{itemize}
		\item \emph{Analogue sim}: Obtain quantitative insight into strongly correlated matter; validate condensed‑matter models; benchmark quantum field theories.  
		\item \emph{QUBO annealer}: Heuristic optimiser for logistics, ML hyper‑tuning, finance portfolios; physics of the device is merely a computational means to an end.
	\end{itemize}
	
	\item \textbf{Hamiltonian flexibility}  
	\begin{itemize}
		\item \emph{Analogue sim}: Limited but \emph{physics‑faithful}.  Tunable lattices, interaction ranges, dimensionality—but still bounded by hardware (e.g.\ Rydberg array geometry).  
		\item \emph{QUBO annealer}: Fixed Ising form; flexibility only in scalar weights $h_i$, $J_{ij}$.  Any higher‑order terms must be gadgetised, bloating qubit count.
	\end{itemize}
	
	\item \textbf{Control schedule}  
	\begin{itemize}
		\item \emph{Analogue sim}: Often a quench or slow parameter sweep; aims to follow the natural time‑evolution of $H_{\text{target}}$.  
		\item \emph{QUBO annealer}: Adiabatic schedule $H(s)=A(s)H_{\text{init}}+B(s)H_{\text{QUBO}}$ with $s\in[0,1]$; user can tweak $A,B$ profiles but not structure.
	\end{itemize}
	
	\item \textbf{Read‑out \& verification}  
	\begin{itemize}
		\item \emph{Analogue sim}: Site‑resolved fluorescence (cold atoms), state‑selective shelving (ions), or scattering probes.  Validation is non‑trivial; often requires cross‑checks with numerics on small instances.  
		\item \emph{QUBO annealer}: Binary samples measured in the $\sigma^z$ basis; solution quality gauged by classical post‑processing; quantum advantage still contested.
	\end{itemize}
	
	\item \textbf{Error landscape}  
	\begin{itemize}
		\item \emph{Analogue sim}: Decoherence competes with desired dynamics; no error correction but some regimes exhibit many‑body localisation‐like robustness.  
		\item \emph{QUBO annealer}: Thermal noise, control error, and biased freeze‑out compromise adiabaticity; scaling of the minimum gap with problem size remains the bottleneck.
	\end{itemize}
	
	\item \textbf{Computational universality}  
	\begin{itemize}
		\item \emph{Analogue sim}: \emph{Not} universal in the circuit‑model sense; each device targets a Hamiltonian class.  
		\item \emph{QUBO annealer}: Equivalent to \emph{adiabatic} quantum computing only if one can arbitrarily engineer $H(t)$ and suppress noise—conditions far from present D‑Wave hardware.
	\end{itemize}
\end{enumerate}

\paragraph{Take‑away for Practitioners}
\emph{Analogue simulators} are bespoke microscopes for quantum matter; \emph{QUBO annealers} are purpose‑built optimisation engines.  Both live in the NISQ swamp—neither offers fault tolerance—but they inhabit opposite ends of the “physics‑faithful vs.\ computation‑oriented” spectrum.  Confusing the two risks overselling hardware capabilities (marketing loves that move); keep your benchmarking metrics, validation protocols, and expectations strictly aligned with the device’s native Hamiltonian.

\section{Digital Quantum Systems: Photonic vs Gate}


For the question on digital quantum systems (photonic vs trapped-ion gates), I'll focus on high-level differences like the type of qubits, coherence, gate implementation, connectivity, scaling, temperature requirements, and error sources. The user likely wants an easy-to-understand comparison for a syllabus, so I'll present this in LaTeX markup. I don't need to browse, since I can tap into the knowledge up to 2025. However, it could help to double-check recent updates, especially for entities like PsiQuantum and Xanadu.

To make sure I provide accurate information, I should look up the latest updates on photonic and trapped-ion quantum computing, especially in terms of qubit count and fidelities, in 2025. I'll search for specifics like "photonic quantum computing 2025 qubit fidelity" and "trapped ion quantum computing 2025 gate fidelity." Since the user didn't say not to browse, this seems like a good idea to stay up to date and gather some sources to cite about photonic quantum computing and gate implementations.

The differences between photonic and trapped-ion quantum systems revolve around a few key points, like qubit implementation, operation environment (vacuum vs temperature), gate mechanism (measurement-based vs laser pulses), connectivity, and gate speed. I'll include factors like coherence, error sources, scaling challenges, and cooling needs, as well as considerations for footprint and the systems' maturity. I can compile all this into a neat LaTeX format, using either a table or enumerated list. This gives a comprehensive, easy-to-read comparison.

\subsubsection{Digital Photonic vs.\ Trapped‑Ion Gate–Based Quantum Computers — A Skeptical Fly‑over}

\begin{description}
	\item[Qubit Embodiment]
	\begin{itemize}
		\item \textbf{Photonics} — single‑photon time‑bin or dual‑rail modes carried in silicon or III–V waveguides; room‑temperature operation.  Qubits propagate at (nearly) light speed; no idle decoherence, but loss is fatal.  Recent prototypes: Xanadu’s 12‑qubit \emph{Aurora}.:contentReference[oaicite:0]{index=0}
		\item \textbf{Trapped ions} — hyperfine or Zeeman levels of ${}^{171}\mathrm{Yb}^+$, ${}^{40}\mathrm{Ca}^+$, etc., confined in RF or surface traps under ultra‑high‑vacuum.  Natural multi‑second $T_1/T_2$, but motional heating and laser phase noise nibble away at fidelity.  Flagship: Quantinuum H2 (56 fully connected qubits).:contentReference[oaicite:1]{index=1}
	\end{itemize}
	
	\item[Native Gate Mechanism]
	\begin{itemize}
		\item \textbf{Photonics:} Fusion / beam‑splitter + phase‑shifter networks build cluster states; computation proceeds via adaptive measurements (measurement‑based QC).  Deterministic two‑qubit gates remain probabilistic without multiplexing and loss‑tolerant codes.
		\item \textbf{Ions:} Laser‑driven Mølmer–Sørensen or light‑shift gates; all‑to‑all connectivity within a trap zone; mid‑circuit measurement and qubit reuse already demonstrated on H2.
	\end{itemize}
	
	\item[Gate Speed]
	\begin{itemize}
		\item \textbf{Photonics:} Picosecond optical delays but effective clock governed by detector latency ($\mathcal{O}(10$–$100\,$ns)).  Throughput bottleneck is photon‑source repetition and feed‑forward electronics.
		\item \textbf{Ions:} $\sim\!10$–$100\;\mu$s per two‑qubit gate — five orders of magnitude slower in wall‑clock, but operations are high‑fidelity and deterministic.
	\end{itemize}
	
	\item[Fidelity Snapshot (2025 claims)]
	\begin{itemize}
		\item \textbf{Photonics:} $\ge \!99.9\%$ single‑qubit, $>99\%$ two‑qubit on wafer‑scale silicon photonics (PsiQuantum internal data).:contentReference[oaicite:2]{index=2}  System‑level loss and detector dark counts remain the weak links.
		\item \textbf{Ions:} $99.997\%$ single‑qubit, $99.87\%$ two‑qubit on Quantinuum H2; record gate fidelities in any platform.:contentReference[oaicite:3]{index=3}
	\end{itemize}
	
	\item[Connectivity \& Scaling Argument]
	\begin{itemize}
		\item \textbf{Photonics:} Chips can be wafer‑tiled and linked via fibre with negligible crosstalk; genuine room‑temperature “data‑centre” vision.  Scaling hinges on bright, indistinguishable photon sources and on‑chip loss $<\!0.1\,$dB/cm — not solved yet.
		\item \textbf{Ions:} All‑to‑all within a trap, but scaling beyond $\sim\!100$ ions per chain requires shuttling or photonic inter‑connects (modular architecture).  Mechanical fragility and vacuum footprint are non‑trivial.
	\end{itemize}
	
	\item[Error Landscape]
	\begin{itemize}
		\item \textbf{Photonics:} Loss dominates; phase noise from thermo‑optic drift; detector jitter.  Error‑correcting proposals rely on fusion‑based cluster states with $\sim\!10^{3}$ physical photons per logical qubit.
		\item \textbf{Ions:} Depolarising errors from laser fluctuations and motional decoherence; electric‑field noise scales badly as traps miniaturise.
	\end{itemize}
	
	\item[Operational Overheads]
	\begin{itemize}
		\item \textbf{Photonics:} Cryogenic SNSPDs, precise phase stabilisation, and massive classical feed‑forward; nonetheless, no dilution fridge.
		\item \textbf{Ions:} Ultra‑high‑vacuum chambers, multi‑wavelength laser systems, RF drive electronics, micro‑fabricated trap chips — complexity grows super‑linearly with qubit number.
	\end{itemize}
	
	\item[Use‑Case Posture (2025)]
	\begin{itemize}
		\item \textbf{Photonics:} Promising for fault‑tolerant networking and photonic quantum repeaters; universal digital QC still at proto‑stage (12‑qubit Aurora).
		\item \textbf{Ions:} Leading NISQ algorithms today (randomised benchmarking, chemistry) and first logical qubits projected by 2027.:contentReference[oaicite:4]{index=4}
	\end{itemize}
\end{description}

\medskip
\noindent\emph{Bottom line:} Photonics offers the seductive scalability of semiconductor fabs and room‑temperature optics, but is currently throttled by loss and non‑determinism.  Trapped‑ion rigs deliver textbook‑grade gates today, yet wrestle with glacial speed and a hardware zoo of lasers and vacuum gear.  Betting on either platform without scrutinising these trade‑offs is venture‑capital theatre, not engineering.  For syllabus purposes, highlight \emph{why} each limitation matters when students later benchmark algorithms like QFT or QSVT under realistic noise and timing budgets.


\section{Syllabus attempt}

%%%%%%%%%%%%%%%%%%%%%%%%%%%%%%%%%%%%%%%%%%%%%%%%%%%%%%%%%%%%%%%%%%%%%%%%%%%%%%%%%%%%%%%%%%%%%%%%%%%%%%%%%%%%%%%%%%%%%%%%%
%%%%%%%%%%%%%%%%%%%%%%%%%%%%%%%%%%%%%%%%%%%%%%%%%%%%%%%%%%%%%%%%%%%%%%%%%%%%%%%%%%%%%%%%%%%%%%%%%%%%%%%%%%%%%%%%%%%%%%%%%
\subsection{Unit 1 — Quantum Foundations \& Hardware}


%%%%%%%%%%%%%%%%%%%%%%%%%%%%%%%%%%%%%%%%%%%%%%%%%%%%%%%%%%%%%%%%%%%%%%%%%%%%%%%%%%%%%%%%%%%%%%%%%%%%%%%%%%%%%%%%%%%%%%%%%
\paragraph{Unit Overview}
This opening unit sets the intellectual stage: why quantum computation is \emph{not} just faster classical processing dressed up in fancy notation, but a qualitatively different paradigm rooted in superposition, interference, and entanglement.  We move from the historical provocations of Feynman and Deutsch, through the minimal mathematics of single–qubit state vectors, to a reality‑check tour of contemporary hardware (superconducting, trapped‑ion, photonic).  The goal is to build solid conceptual intuition \emph{before} the linear‑algebra machinery arrives in Unit 2.

%%%%%%%%%%%%%%%%%%%%%%%%%%%%%%%%%%%%%%%%%%%%%%%%%%%%%%%%%%%%%%%%%%%%%%%%%%%%%%%%%%%%%%%%%%%%%%%%%%%%%%%%%%%%%%%%%%%%%%%%%
\paragraph{Why It Matters}
\begin{itemize}
	\item \textbf{Cognitive framing.}  Students trained in data‑science and crypto carry strong classical mental models; dismantling those early prevents cargo‑cult coding later on.
	\item \textbf{Hardware‑aware coding.}  NISQ era devices leak, decohere, and mis‑gate.  A sober understanding of physical qubits and cross‑talk is essential for writing algorithms that survive reality.
	\item \textbf{Risk perspective.}  For cryptographers, Shor’s algorithm is an existential threat only if scalable qubits materialise.  A candid hardware survey shows how far (or close) that threat really is.
\end{itemize}

%%%%%%%%%%%%%%%%%%%%%%%%%%%%%%%%%%%%%%%%%%%%%%%%%%%%%%%%%%%%%%%%%%%%%%%%%%%%%%%%%%%%%%%%%%%%%%%%%%%%%%%%%%%%%%%%%%%%%%%%%
\paragraph{Intended Learning Outcomes (ILOs)}
By the end of Unit 1, a student will be able to:
\begin{enumerate}
	\item \textbf{Explain} the historical motivation for quantum computation and contrast it with the Turing model.
	\item \textbf{Describe} and \textbf{diagram} single‑qubit states on the Bloch sphere, including global versus relative phase.
	\item \textbf{Evaluate} the practical trade-offs of leading hardware platforms (superconducting, trapped-ion, photonic, neutral‑atom) using publicly reported coherence, gate‑fidelity, and qubit-connectivity data.
	\item \textbf{Implement} basic one- and two-qubit gate sequences on a cloud SDK (e.g.\ IBM Qiskit, Xanadu PennyLane) and \textbf{interpret} backend noise reports.
	\item \textbf{Critically assess} vendor marketing claims versus published benchmark numbers (e.g. Xanadu’s \emph{Aurora} photonic prototype, PsiQuantum’s wafer‑scale roadmap).
\end{enumerate}

%%%%%%%%%%%%%%%%%%%%%%%%%%%%%%%%%%%%%%%%%%%%%%%%%%%%%%%%%%%%%%%%%%%%%%%%%%%%%%%%%%%%%%%%%%%%%%%%%%%%%%%%%%%%%%%%%%%%%%%%%
\paragraph{Indicative Subject Matter}
\begin{itemize}
	\item \textbf{1.1 History \& Conceptual Sparks}
	\begin{itemize}
		\item Feynman’s 1981 simulation argument; Deutsch’s universal quantum computer (1985).
		\item Pop‑science to precision: from double‑slit lore to Hilbert‑space formality.
		\item Reading: Nielsen–Chuang §1.1–1.3; Preskill (``40 Years Later’’, 2023) §1.
	\end{itemize}
	\item \textbf{1.2 Qubits, Gates, and Minimal Math}
	\begin{itemize}
		\item Dirac notation crash‑course; $\ket{0}$, $\ket{1}$, $\alpha\ket{0}+\beta\ket{1}$ with $\lvert\alpha\rvert^{2}+\lvert\beta\rvert^{2}=1$.
		\item Bloch sphere and single‑qubit rotations; CNOT for entanglement.
		\item Lab: build a ``hello‑world’’ Bell‑state circuit on cloud hardware, pull calibration data, dissect error rates.
		\item Reading: Nielsen–Chuang §2.1–2.4; Lipton §1 (vectors, bras \& kets).
	\end{itemize}
	\item \textbf{1.3 Modern Photonics Snapshot}
	\begin{itemize}
		\item How photonic qubits are generated (SPDC sources, time‑bin encoding), manipulated (interferometers, phase shifters), and measured.
		\item Case study: Xanadu’s \emph{Aurora} (12‑qubit universal photonic prototype, Jan 2025) and PsiQuantum’s wafer‑scale ``Omega’’ chip — scaling claims versus current error budgets.%
		\footnote{See Xanadu press release, Jan 22 2025; PsiQuantum ``Omega’’ announcement, Feb 26 2025.}
		\item Practical: run a Gaussian‑boson‑sampling demo in PennyLane and compare to superconducting backend timings.
	\end{itemize}
	\item \textbf{Cross‑Unit Bridges}
	\begin{itemize}
		\item Noise sources preview (T$_1$, T$_2$, photon loss) pointing forward to the decoherence unit.
		\item Relevance to Shor: why gate depth and qubit count projections matter.
	\end{itemize}
\end{itemize}

\vspace{0.5em}
\noindent\emph{Assessment touchpoint:} 5‑question critical‑reading quiz (vendor data sheets), plus a notebook submission generating and analysing a Bell pair on at least two cloud backends.  The rubric checks that students do \emph{not} blindly trust SDK ``success’’ metrics but cross‑validate with backend calibration logs.

\section{Quantum OC-SVM}

	A \textbf{quantum one-class support-vector machine (Q OC-SVM)} is an anomaly-detection model that retains the classical OC-SVM objective—finding the smallest-radius “soft” hypersphere (or maximum-margin hyper-plane through the origin) enclosing only the training data—but replaces the \emph{kernel} computation with a quantum circuit.
	
	\subsection{Workflow}
	\begin{enumerate}
		\item \textbf{Encoding:} Each data point $\mathbf{x}$ is mapped to a quantum state $\lvert \phi(\mathbf{x})\rangle$ by a \textbf{feature-map circuit} that entangles qubits and embeds nonlinear structure.
		\item \textbf{Kernel Evaluation:} Inner products $\langle \phi(\mathbf{x}_i) \mid \phi(\mathbf{x}_j) \rangle$ are estimated on quantum hardware (e.g., with the Swap-test or duplicated circuits). This populates the kernel matrix $K$.
		\item \textbf{Classical Optimization:} The usual OC-SVM quadratic program is solved on a classical CPU, producing support vectors and the decision function.
		\item \textbf{Inference:} For a new sample $\mathbf{z}$, the quantum computer supplies kernel values $K(\mathbf{z}, \mathbf{x}_i)$; the classical decision rule flags “inliers” or “anomalies.”
	\end{enumerate}
	
	The hope is that a well-chosen quantum feature map yields a kernel that is \textbf{hard to approximate classically}, giving tighter one-class boundaries with fewer training points on NISQ-era devices.
	
	For more details, refer to:
	\begin{itemize}
		\item \href{https://arxiv.org/abs/2407.20753}{Efficient Quantum One-Class Support Vector Machines for Anomaly Detection}
		\item \href{https://arxiv.org/abs/2210.02674}{Variational Quantum One-Class Classifier}
	\end{itemize}
	
	\subsection{Recursive Breakdown into Beginner-friendly Building Blocks}
	
	Below is a dependency tree. Each level expands the node above it; stop when bullets are \emph{atomic} (no further division helps a novice).
	
	\begin{table}[h]
		\centering
		\renewcommand{\arraystretch}{1.2}
		\begin{tabular}{|c|l|l|}
			\hline
			\textbf{Level} & \textbf{Topic} & \textbf{Sub-topics a beginner must master} \\
			\hline
			\textbf{0} & Quantum OC-SVM & See Levels 1–4 \\
			\hline
			\textbf{1} & A. Machine-learning objective & Anomaly-detection use-cases, classical OC-SVM formulation ($\nu$-parameter, slack variables), kernel trick concept \\
			\hline
			& B. Quantum advantage idea & Quantum feature maps \& kernels, NISQ vs fault-tolerant limits \\
			\hline
			& C. Implementation workflow & Data encoding $\rightarrow$ kernel $\rightarrow$ optimization $\rightarrow$ inference, hybrid quantum–classical loop \\
			\hline
			\textbf{2} & A1 $\rightarrow$ Anomaly-detection basics & Definition of outliers, Precision/Recall vs ROC, Train-only-on-normal paradigm \\
			\hline
			& A2 $\rightarrow$ OC-SVM maths & Support vectors \& margins, Primal/dual QP, Geometric view (hypersphere/hyper-plane) \\
			\hline
			& A3 $\rightarrow$ Kernel trick & Feature space concept, Positive-definite kernels, Common kernels (RBF, polynomial) \\
			\hline
			& B1 $\rightarrow$ Quantum kernels & Inner products as kernels, Expressivity vs barren plateaus \\
			\hline
			& B2 $\rightarrow$ NISQ limitations & Noise, depth, qubit count, Error-mitigation vs full QEC \\
			\hline
			\textbf{3} & OC-SVM maths (detail) & Lagrange dual, $\nu$ parameter interpretation, Support-vector decision function \\
			\hline
			& Kernel theory (detail) & Mercer's theorem, Gram matrix properties \\
			\hline
			& Quantum feature maps & Data encoding schemes (angle, amplitude, IQP-type), Circuit depth vs expressivity \\
			\hline
			& Hybrid optimization loop & Classical QP solvers (CVXOPT, LIBSVM), Data reuse to amortize quantum calls \\
			\hline
			\textbf{4} & Foundational prerequisites & Linear algebra, complex numbers, probability, quantum basics, circuit model skills \\
			\hline
		\end{tabular}
		\caption{Layered approach to learning Quantum OC-SVM}
	\end{table}
	
	\subsection{When to Teach Each Piece in Your Syllabus}
	\begin{itemize}
		\item \textbf{Levels 4 $\rightarrow$ 2 (foundations, gates, kernels)} naturally fit in \textbf{Section 2} (Quantum Gates \& Circuits plus bra-ket/matrix maths).
		\item \textbf{Machine-learning fundamentals (Level 2-A)} belong in \textbf{Section 4.4 (Quantum ML)} as a preliminary sub-module.
		\item \textbf{Quantum kernel \& feature-map specifics (Levels 2-B1, 3)} pair with \textbf{Section 4.4} practical labs (Pennylane/Qiskit QSVM demos).
		\item \textbf{NISQ/noise topics (Levels 2-B2, 3 noise)} reinforce \textbf{Section 2.5 (NISQ \& Error-Correction)}.
		\item \textbf{Hybrid optimization loop \& SDK skills (Levels 2-C, 3)} should be a hands-on lab immediately after the theory block in \textbf{Section 4.4}.
	\end{itemize}
	
	Providing this layered roadmap ensures a beginner can climb from \emph{linear-algebra fluency} all the way to \textbf{implementing and testing a quantum OC-SVM on real cloud hardware}.

\section{Another QSVM Syllabus}


	This section provides a detailed analysis of the quantum OC-SVM syllabus. 
	\textbf{Bold text} flags factual errors, while \textit{italic text} highlights weaknesses or gaps requiring further exposition.
	
	\subsection{1. Quantum vs Classical Information \& The Qubit}
	\begin{itemize}
		\item Accurate, but add a one-sentence pointer to \textbf{global phase} (why $\lvert\psi\rangle$ and $e^{i\theta}\lvert\psi\rangle$ are physically identical) so learners don’t carry a silent misconception into bra-ket algebra later.
		\item Superposition alone doesn’t create quantum advantage; entanglement and interference deserve a teaser here.
	\end{itemize}
	
	\subsection{2. Quantum States, Tensor Product, and Measurement}
	\begin{itemize}
		\item \textbf{“Measurement is a non-intrusive operation on quantum systems.”} This is incorrect. Projective measurement irreversibly collapses the state; even weak or non-demolition measurements disturb conjugate variables.
		\item Good to cite Holevo, but underscore that the theorem applies to \textit{one copy} of the state; collective measurements on many copies can extract more classical information at a cost.
		\item \textit{Missing}: density matrices / mixed states. OC-SVM kernels will be estimated from \textit{noisy} density operators on NISQ devices; students need that concept soon.
	\end{itemize}
	
	\subsection{3. The Quantum Circuit Model \& Universal Gates}
	\begin{itemize}
		\item Solid. Mention that \textbf{measurement gates are the only non-unitary primitive allowed} in the model (ties back to the collapse comment).
		\item \textit{Weakness}: No-cloning and reversibility belong here (or in §1) because they constrain algorithm design and data-loading tricks later.
	\end{itemize}
	
	\subsection{4. Introduction to Quantum Algorithms \& Paradigms}
	\begin{itemize}
		\item Good list, but \textbf{remove HHL and VQE from the OC-SVM track unless teaching kernelized linear-system solvers} (they are overkill).
		\item \textit{Missing}: explicit mention of “quantum kernel estimation” and “swap test / Hadamard test” paradigms—these are the core of QSVM/OC-SVM.
	\end{itemize}
	
	\subsection{Mapping Learning Outcomes to the Layered Syllabus}
	\begin{longtable}{|c|l|l|}
		\hline
		\textbf{Outline Section} & \textbf{Fixes/Gaps} & \textbf{Where It’s Mastered in the Syllabus} \\
		\hline
		1, 2, 3 & Add entanglement, collapse, density matrices, no-cloning & Foundations (0) $\rightarrow$ Gates (2) \\
		\hline
		4 & Remove HHL/VQE or flag as optional; insert kernel paradigms & Canonical Algos (3) + QML intro (4.4.1) \\
		\hline
		5 & Add encoding trade-offs, data-reuploading & QML data-prep lab (4.4.1) \\
		\hline
		6-8 & Move to optional “linear-algebra extras” appendix & Optional deep-dive week \\
		\hline
		9 & Include noise analysis & Kernel-estimation lab (4.4.3) \\
		\hline
		10 & Formal dual, $\nu$ parameter & Classical OC-SVM primer (4.4.1) \\
		\hline
		11 & Split feature-map vs measurement, add barren-plateau discussion & Quantum kernel lecture (4.4.2) \\
		\hline
		12 & Hybrid workflow, shot budget, mitigation & OC-SVM capstone lab (4.4.3) \\
		\hline
	\end{longtable}
	
	\subsection{Bottom Line}
	\begin{itemize}
		\item \textbf{Biggest factual error}: Calling measurement non-intrusive.
		\item \textbf{Over-ambitious detour}: Block-encoding $\rightarrow$ QSVT $\rightarrow$ QLSS is not needed for QSVM; prune or make optional.
		\item \textbf{Pedagogical gaps}: Density matrices, noise \& shot variance, kernel PSD requirement, feature-map design, $\nu$-parameter tuning.
		\item Once these are patched, every learning outcome maps cleanly onto the staged syllabus—learners progress from foundational concepts through kernel estimation, finally coding and benchmarking a quantum OC-SVM on real hardware.
	\end{itemize}

	
\section{Block-Encoding: Origin and Timeline}
	
	\subsection{Formal Definition}
	The \textbf{formal notion of block-encoding} was introduced by \textbf{Guang Hao Low and Isaac Chuang} in their 2016–17 quantum signal-processing papers. It was later generalized and popularized by \textbf{András Gilyén, Yuan Su, Guang Hao Low, and Nathan Wiebe} in their 2018 work on Quantum Singular Value Transformation (QSVT). Earlier Hamiltonian simulation papers by Dominic Berry, Andrew Childs, and co-authors contained the \textit{idea} in disguise via the Linear Combination-of-Unitaries (LCU) technique, but the explicit block-encoding definition and notation come from Low \& Chuang.
	
	\begin{itemize}
		\item \href{https://www.ias.edu/sites/default/files/Tang%20qsvt_lect_1.pdf?utm_source=chatgpt.com}{PDF: The Block-Encoding}
		\item \href{https://dl.acm.org/doi/pdf/10.1145/3313276.3316366?utm_source=chatgpt.com}{Quantum Singular Value Transformation and Beyond}
		\item \href{https://arxiv.org/abs/1806.01838?utm_source=chatgpt.com}{Quantum Singular Value Transformation on arXiv}
	\end{itemize}
	
	\subsection{Historical Timeline}
	
	\textbf{2015 and earlier – LCU Era}  
	Berry, Childs, Cleve, Kothari, and others show that sparse Hamiltonians can be simulated by writing them as linear combinations of short unitaries. This implicitly embeds a (scaled) Hamiltonian in the top-left block of a larger unitary, but no formal name is given.
	
	\textbf{2016–2017 – Birth of the Term}  
	Low \& Chuang’s \textit{Optimal Hamiltonian Simulation by Quantum Signal Processing} (PRL 118 010501, 2017; preprint Oct 2016) introduces the \textbf{block-encoding notation}
	\[
	U = \begin{pmatrix} A/\alpha & * \\ * & * \end{pmatrix}
	\]
	and proves that such an encoding plus phase-factor sequences can implement polynomial functions of \( A \).
	
	\textbf{2018 – Generalization to QSVT}  
	Gilyén, Su, Low, \& Wiebe (arXiv:1806.01838) fuse block-encoding with signal-processing techniques, showing a single unifying primitive—Quantum Singular Value Transformation (QSVT)—that subsumes amplitude amplification, Hamiltonian simulation, and various quantum ML routines. Their STOC-19 paper firmly established the term in the literature.
	
	Since then, block-encoding has become the standard approach whenever one needs to \textit{"smuggle"} a non-unitary matrix into a quantum circuit.
	

\section{syllabus outline}

	Continuing our discussion on the syllabus outline, the source \cite{Wong:2011} "wong\_2011\_Outcome-based-teaching-and-learning-in-computer\_8-W82.pdf" 
	offers insights into curriculum design from an \emph{outcome-based teaching and learning (OBTL)} perspective, 
	which would heavily influence the creation of a syllabus.
	
	According to the paper, an \emph{outcome-based approach (OBA)} to curriculum design focuses on what students are expected to learn and be able to do upon completion of a program or course, rather than solely on what the teacher intends to teach. This philosophy would be central to outlining a syllabus.
	
	The paper provides an example of an OBA curriculum design for a \emph{Higher Diploma in Computer Studies}. This example highlights several key components that would be reflected in a detailed syllabus:
	
	\begin{itemize} 
	\item \emph{Objectives} The program design starts with clearly stated \emph{Concentration Objectives} and \emph{Curricular Objectives}, as well as more specific objectives for graduates. These objectives set the overall direction for the program and individual courses, and learning outcomes within a syllabus would directly contribute to achieving these broader goals.
	
	\item \emph{Course Components:} The curriculum outlines the \emph{courses to be taken and their major contents descriptions}. A syllabus for each course would expand on these descriptions, detailing the specific topics to be covered.
	
	\item \emph{Teaching and Learning Activities:} The program emphasises the use of various \emph{teaching methods} such as lectures, tutorials, seminars, case studies, guest talks, and projects. A syllabus would specify which of these activities will be used for each topic or module, providing students with a clear understanding of how they will learn the material. The aim is to create an active learning environment where students can apply their knowledge.
	
	\item \emph{Learning Outcomes:} The paper stresses that \emph{learning outcome statements should reflect what students are expected to be able to do at the end of the learning period}. A well-designed syllabus would include specific and measurable learning outcomes for each module or topic, guiding both the teaching and the assessment. The paper even provides examples of how to write effective learning outcomes using action verbs.
	
	\item \emph{Assessment:} Although the example doesn't detail course-level assessments, the discussion on OBA mentions \emph{outcome-based assessment}, which focuses on measuring whether students have achieved the stated learning outcomes. A syllabus would outline the assessment methods (e.g., projects, assignments, exams) and how they align with the learning outcomes.
	
	\end{itemize}
	
	The paper also briefly mentions the use of a system called \emph{PACOS (Program and Courses Outcomes System)} to ensure that course learning outcomes align with the program's ultimate objectives. This highlights the interconnectedness that an outcome-based syllabus would have with the broader curriculum goals.
	
	In essence, when developing a syllabus within an outcome-based framework, the focus shifts from simply listing topics to be taught, to clearly defining what students should be able to achieve, and how the teaching and assessment methods will facilitate and measure that achievement.
	
\section{Topic One}

Quantum Foundations and Hardware

\subsection{Overview}

	\begin{itemize} 

	\item 1.1 Basic quantum computing history (e.g., Feynman, Deutsch)
Reference: Feynman (1982) "Simulating Physics with Computers"; Deutsch (1985) "Quantum theory, the Church-Turing principle and the universal quantum computer."

	\item 1.2 Quantum Computing Overview

	Classical vs. quantum computing
	
	Superposition, entanglement, measurement, quantum bits vs classical bits
	
	Recommended: Nielsen \& Chuang (2010), Chapters 1–2.

	\item 1.3 The Qubit: Representations and Realizations

	Bloch sphere, quantum state vectors, coherence/decoherence
	
	Hardware: superconductors, ion traps, quantum dots, initial photonic systems
	
	Recommended: Preskill (2018); Monroe et al. (2021); Arute et al. (2019)
	
	\item 1.4 Modern Photonic Quantum Machinery

	Photonic qubits, single-photon detection, integrated photonic circuits
	
	Industry examples: PsiQuantum, ORCA Computing
	
	Recommended: Rudolph (2017); Bartolucci et al. (2023)

\end{itemize}


Critique:

While the topics covered in this section seem fundamental and provide a good overview of quantum computing foundations and hardware, several areas could be enhanced from an OBTL perspective

	\begin{itemize}
		
	\item Lack of Explicit Learning Outcomes: The current outline primarily lists topics. According to Wong and Cheung, an outcome-based approach emphasises explicit statements of what students will be expected to be able to do after completing this section. The syllabus should clearly articulate the knowledge and skills students will gain.

	\item Limited Connection to Practical Applications: While industry examples are mentioned in 1.4, the overall section could better integrate the theoretical concepts with the practical use cases explored in the NQCC hackathon. The report highlights various applications across sectors like healthcare, energy, finance, and telecommunications, solved using different quantum hardware.

	\item Implicit Assessment: The outline doesn't suggest how the learning in this section will be assessed. In an OBA framework, assessment should measure whether students have achieved the stated learning outcomes.

	\item High-Level Hardware Overview: While different hardware types are mentioned (superconductors, ion traps, etc.), the level of detail and the connection to their specific strengths and weaknesses for different types of problems (as hinted at in the hackathon report
) could be more explicit. For instance, the report notes the usefulness of quantum annealers for optimisation problems.

	\item No Mention of Ethical Considerations: The NQCC hackathon in 2023 specifically asked participants to consider responsible and ethical quantum computing (REQC). This foundational section could briefly introduce the societal impact and ethical considerations of quantum technologies.

\end{itemize}

Proposed Changes:

To strengthen this section using an outcome-based approach and connect it more effectively with practical applications and ethical considerations, I propose the following changes:

	\begin{itemize}	

	\item Add Specific Learning Outcomes: For each subsection (1.1 to 1.4), clearly define what students should know or be able to do upon completion. Use action verbs that are measurable, as suggested in "wong\_2011\_Outcome-based-teaching-and-learning-in-computer\_8-W82.pdf". For example:

1.1 Basic quantum computing history: "Students will be able to describe the key contributions of Feynman and Deutsch to the field of quantum computing."

1.2 Quantum Computing Overview: "Students will be able to compare and contrast classical and quantum computing, explain the principles of superposition, entanglement, and measurement, and differentiate between classical and quantum bits."

1.3 The Qubit: "Students will be able to represent qubit states using the Bloch sphere and quantum state vectors, explain the concepts of coherence and decoherence, and identify the basic working principles of different quantum computing hardware platforms (superconductors, ion traps, quantum dots, photonic systems)."

1.4 Modern Photonic Quantum Machinery: "Students will be able to describe the properties of photonic qubits and single-photon detection, explain the concept of integrated photonic circuits, and provide examples of companies utilising photonic quantum computing (e.g., PsiQuantum, ORCA Computing)."

	\item Integrate Practical Use Cases from the NQCC Hackathon: When discussing each fundamental concept or hardware type, briefly refer to relevant use cases from the hackathon report. For example:

When discussing optimisation in 1.2, mention Team 2 (route optimisation in healthcare), Team 3 (unit commitment for electricity grid), Team 7 (vehicle routing), Team 9 (network routing), and Team 10 (vaccination centre location).

When discussing different hardware in 1.3, mention which platforms were used by different teams (e.g., D-Wave for annealing, IBMQ, IonQ, OQC for gate-based
).

When discussing photonic systems in 1.4, you could highlight that ORCA Computing was a provider at the hackathon.

	\item Suggest Preliminary Assessment Methods: Even in the syllabus outline, you could briefly hint at how the learning in this section might be assessed (e.g., short quizzes on basic concepts, brief explanations of hardware types, conceptual questions linking theory to hackathon use cases). This aligns with the OBA principle that assessment should be aligned with learning outcomes.

	\item Expand Hardware Discussion with Respect to Problem Types: When introducing different hardware, briefly mention the types of problems they are generally suited for based on the insights from the hackathon. For instance, highlight that quantum annealers are often used for combinatorial optimisation problems, as seen with multiple teams using D-Wave.
	
	\item Introduce Responsible and Ethical Quantum Computing (REQC): Consider adding a brief subsection (e.g., 1.5) introducing the potential societal impact of quantum computing and the importance of ethical considerations, aligning with the focus on REQC in the NQCC hackathon. 
\end{itemize}

	
	This could lead into more detailed discussions later in the syllabus.
	
By incorporating these changes, section one of your syllabus will move towards a more explicit outcome-based structure, clearly indicating what students will learn and connecting the foundational knowledge to real-world applications and ethical considerations, as demonstrated in the NQCC Quantum Hackathon. This will provide a clearer roadmap for students and facilitate more effective teaching and assessment.

\section{Lack of integrating Curriculum}

When describing the motivation for developing this syllabus, I want to start by highlighting how, as Wong et. al. state, that each course should fulfil a part of a program curriculum.  In preparing this quantum course, we can try to build on learning outcomes from other program units, e.g. data science, cryptography, coding theory, and cyber-security, but a challenge is that there is no guarantee that each student has taken that unit, or that they all have had exposures to certain analogues or mental models, that would aid in the delivery of new quantum computing concepts.  External to the program, there is equally, no clear cut quantum curriculum that can be used to manage outcomes.  Even the various nomenclatures used in reference books mean that concepts may not be clear when the students are given reading materials.   In the context of Wong's Outcome based teaching and learning, can you reflect these problems back to me in a way that demonstrates the issues.

Okay, let's reflect on the challenges you've identified in developing this quantum computing syllabus, specifically in the context of Section 1 (Quantum Foundations and Hardware) and drawing on the principles of Outcome-Based Teaching and Learning (OBTL) discussed in the "wong\_2011\_Outcome-based-teaching-and-learning-in-computer\_8-W82.pdf" source and the practical context from the "NQCC\_Quantum-Hackathon-2023\_Technical-report.pdf".

You've highlighted several crucial points that directly impact the implementation of an outcome-based approach:


\begin{itemize}	

\item 1.
Connecting Course Outcomes to Program Curriculum: As Wong and Cheung state, the OBTL framework suggests that all course learning outcomes should point to the ultimate objective outcomes in the program studies.

You aim to build on learning outcomes from other program units like data science, cryptography, coding theory, and cybersecurity. However, you face the significant challenge that there is no guarantee that each student has taken these specific units, or that they have equivalent foundational knowledge or mental models. This directly impedes the OBTL goal of setting clear, measurable expectations for what students should be able to do upon completing your quantum course. If students arrive with varied levels of prerequisite knowledge, achieving uniform learning outcomes becomes inherently difficult, making it harder to confirm that the course is contributing to higher-level program objectives in a consistent way.

\item 2. Absence of a Standardised Quantum Curriculum: A core challenge in applying OBTL is the need for explicit statements of what students will be expected to be able to do. Wong and Cheung discuss the importance of defining learning outcomes and mention frameworks like Bloom's taxonomy, although noting they aren't always a perfect fit for computer science. For a field as rapidly evolving and interdisciplinary as quantum computing, you point out the problem: there is no clear-cut quantum curriculum that can be used to manage outcomes. This lack of a widely agreed-upon structure and set of expected competencies across different programs makes it hard to benchmark, align, and clearly define the learning outcomes for your course in a universally understood way. The diverse applications explored in the NQCC hackathon report—spanning optimisation, machine learning, and simulation across various sectors using different hardware types
—further illustrate the breadth and lack of a single, narrow focus in the field, contributing to the difficulty in standardising curriculum.

\item 3. Inconsistent Nomenclature: You note that even the various nomenclatures used in reference books mean that concepts may not be clear when the students are given reading materials. This issue directly conflicts with the OBTL principle of clearly communicating expectations to students. 

If the language used to describe fundamental concepts or techniques varies significantly between resources, it can create confusion, hinder understanding, and make it harder for students to confidently demonstrate that they have achieved a specific learning outcome, regardless of how well that outcome is defined in terms of action verbs.

\end{itemize}

These issues collectively demonstrate the practical difficulties in fully implementing the OBTL framework for a subject like quantum computing. While the ideal is to clearly state what students should know or be able to do
, define appropriate teaching and learning activities, and align assessment methods to measure these outcomes, the fluidity of the field, the varied student backgrounds, and the lack of standardised resources present significant hurdles in creating the kind of specific, measurable, and tightly coupled syllabus that the OBTL approach advocates for. Wong and Cheung themselves acknowledge the challenges computer science faculty face in defining objectives and assessing outcomes within this framework, noting that many are "not originally trained in techniques for setting up objectives and measuring outcomes". Your experience highlights that these challenges are amplified in a cutting-edge, multidisciplinary area like quantum computing.

\section{Example Curriculum}

Wong gives an example of an OBA curriculum design.  My aim is to deliver, concisely, a set of toolkits that will allow a student to engage with the application of quantum algorithms in a novel context, and to have enough exposure to practical limitations and research directions, to be confident in understanding limitations, and to engage with advanced practitioners on new directions when moving from NISQ to FTQC.  What concentraion objectives, curricular objectives, course components and learning activities can you infer from this?

Okay, let's infer the elements of an Outcome-Based Teaching and Learning (OBTL) curriculum for your quantum computing course, drawing on the structure and examples provided by Wong and Cheung and reflecting your specific aims in the context of challenges like varied student backgrounds and the emerging nature of the field, as illustrated by the NQCC hackathon.

Based on Wong and Cheung's approach to designing a sub-degree Computer Studies program
and your goal to deliver toolkits for novel applications, foster understanding of limitations, and enable engagement with advanced practitioners on the NISQ to FTQC transition, here are some potential Concentration Objectives, Curricular Objectives, Course Components, and Teaching and Learning Activities:

\subsection{Inferred Elements}

for Your Quantum Computing Course Curriculum (following Wong \& Cheung's structure):

\subsubsection{Concentration Objectives (Aims):}

Drawing from Wong and Cheung's aims to develop ability in design/development/support, provide basic knowledge for adaptation to new tech, include relevant disciplinary knowledge, and foster independent thought/creativity:


\begin{itemize}	

\item To develop students' ability in applying quantum algorithms and software to formulate computational solutions for novel problems
, contributing to the advancement of quantum computing applications.

\item To provide students with fundamental knowledge in quantum foundations, hardware, and algorithms
that allows them to adapt to and engage with new technological developments in this rapidly evolving field.

\item To provide students with knowledge regarding the current technical capabilities and limitations of quantum computing platforms
, as well as awareness of potential applications across various sectors and the importance of responsible and ethical considerations.

\item To increase students' ability and confidence for independent exploration of quantum computing use cases and creative problem-solving.

\end{itemize}	


\subsubsection{Curricular Objectives:}

Translating Wong and Cheung's curricular objectives regarding project design/management, adaptability, fundamental knowledge for progression, communication, and awareness of broader issues to your quantum context:

\begin{itemize}	
	
\item To provide students with the knowledge and practical techniques required to formulate and prototype computational solutions
for problems potentially addressable by quantum computing, fostering skills applicable to future careers in this domain.

\item To provide students with professional training in quantum computing so that they may possess the ability to adapt to different quantum computing platforms (e.g., gate-based, annealing)
and software environments (SDKs).

\item To equip students with the fundamental knowledge of quantum algorithms and their application
to enable them to engage with advanced concepts, understand the challenges of the NISQ era, and prepare for the transition to fault-tolerant quantum computing (FTQC).

\item To promote students' ability to communicate technical concepts and practical limitations of quantum computing solutions
, and develop awareness in responsible and ethical implications.

\end{itemize}	

\subsubsection{Course Components}

Mapping Wong and Cheung's approach of starting with fundamental building blocks and moving to more advanced/professional concepts and projects
to your quantum course:

\begin{itemize}	

\item Fundamental Quantum Concepts: Introduction to qubits, superposition, entanglement, and quantum gates as the key building blocks.

\item Quantum Programming Toolkits: Formal introduction to quantum programming languages and SDKs (e.g., Qiskit, Pennylane, Classiq, D-Wave Ocean SDK, Amazon Braket SDK)
, shifting emphasis from basic concepts to practical application and industry-standard tools.

\item Key Quantum Algorithms: Study of relevant algorithms (e.g., HHL, VQE, QAOA)
that represent core techniques.

\item Application Domains and Use Cases: Exploration of how quantum computing could be applied to problems in optimisation, machine learning, and simulation
, using examples inspired by real-world or industry use cases.

\item Hardware and Limitations: Discussion of different quantum hardware modalities
, the characteristics and limitations of Noisy Intermediate Scale Quantum (NISQ) devices, including noise, connectivity, and the need for error mitigation.

\item Future Directions: Introduction to concepts related to Fault-Tolerant Quantum Computing (FTQC) and current research challenges.

\item Practical Projects: Assigning projects where students apply the learned toolkits to formulate solutions for a given problem, commensurate with their level of competence. This could involve testing solutions on emulators and/or real hardware where possible.

\end{itemize}	

\subsubsection{Teaching and Learning Activities:} 

Adapting Wong and Cheung's emphasis on active learning, practice, problem-solving, industry exposure, and projects
to your quantum context, incorporating activities seen in the NQCC hackathon:

\begin{itemize}	

\item Interactive Lectures: Introduce and explain quantum concepts, algorithms, hardware types, limitations, and ethical considerations, encouraging student questions and discussion.

\item Tutorials/Labs: Provide hands-on practice with quantum programming using SDKs, running small programs on emulators or available hardware, clarifying misconceptions through discussion.

\item Case Study Analysis: Examine and analyse industry-provided use cases
, discussing the problem formulation and the challenges encountered when attempting quantum solutions. The NQCC hackathon technical summaries could serve as concrete examples here.

\item Guest Talks/Industry Engagement: Invite researchers and practitioners from the quantum computing industry to share up-to-date knowledge, discuss real-world applications, current capabilities, limitations, and future prospects (NISQ to FTQC).  This mirrors the role of mentors and providers in the hackathon.

\item Projects/Team Work/Exercises: Engage students in projects or exercises that require applying and integrating the knowledge and skills learned.
 This could involve designing and coding solutions for simplified use cases, exploring different hardware options, analysing results in the context of NISQ limitations, and presenting their findings. Team-based activities can facilitate peer-to-peer learning.

\item Problem-Solving Sessions: Develop students' ability to formulate quantum approaches to computational problems.

\item Discussions on Responsible Innovation: Engage students in considering the societal impact and ethical implications of quantum computing applications.

\end{itemize}	

By structuring your syllabus development around these elements, you can align your course with the OBTL framework, clearly stating what students should be able to do by the end. The challenges you mentioned (varied backgrounds, lack of standard curriculum, inconsistent nomenclature) become factors to address within these activities and components (e.g., providing foundational review, carefully curating consistent reading materials, using varied activities to accommodate different learning styles and prior exposure, providing extra support for those without specific prerequisites) rather than reasons not to adopt an outcome-focused approach. The NQCC report serves as a valuable real-world illustration of the types of practical engagement and problem-solving that can be integrated into the curriculum and activities.

\subsection{Utilising Existing non-quantum experience}

I would like to make a few points, and then have you reflect those into your last response.  With our department there are no quantum experts, and even for to purposes of delivering my core thesis, that we can develop and include a quantum unit in our masters program, will need me to explain certain quantum concepts in my paper, and I would like to use natural quantum analogies that mathematics and computer science professors will understand, to bring them along with my idea. For example, quantum computational basis states for qudits, form a discrete alphabet, like code-words in classical coding theory. Quantum error correction builds on classical coding ideas like repetition codes and hamming codes.  Quantum fourier transforms naturally uncover periods used in integer factorization, many quantum algorithms can be understood as linear algebra and tensor products, and quantum machine learning has direct classical machine learning antecedents.  Can we utilize existing capabilities in the department to add value in this work

Integrating these insights into the previously outlined OBTL framework will indeed strengthen the proposal for a quantum unit in your Master's program, particularly given the lack of in-house quantum experts and the need to build understanding among colleagues in mathematics and computer science.
Your aim to use natural quantum analogies that mathematics and computer science professors will understand is a key strategy for bridging the knowledge gap and leveraging existing departmental strengths. This approach can be explicitly woven into the curriculum design across the Concentration Objectives, Curricular Objectives, Course Components, and Teaching and Learning Activities.
Here’s how your points reflect into the inferred OBTL structure:

\subsubsection{1. Concentration Objectives (Aims):}

\begin{itemize}	
	
\item The original objective "To provide basic knowledge in quantum foundations, hardware, and algorithms that allows them to adapt to and engage with new technological developments"
can be reinforced by explicitly stating that this foundational knowledge will be built upon existing strengths in mathematics and computer science.

\item The aim "to increase students' ability and confidence for independent exploration of quantum computing use cases and creative problem-solving"
is directly supported by teaching through familiar analogies, making the 'novel context' more accessible by relating it back to known principles.

\end{itemize}	


\subsubsection{2. Curricular Objectives:}

\begin{itemize}	

\item The objective "To provide students with the knowledge and practical techniques required to formulate and prototype computational solutions for problems potentially addressable by quantum computing"
can be framed as applying quantum toolkits [as per your aim] by building upon their existing understanding of computational problem-solving frameworks from classical computing and mathematics.

\item The objective "To equip students with the fundamental knowledge of quantum algorithms and their application to enable them to engage with advanced concepts, understand the challenges of the NISQ era, and prepare for the transition to fault-tolerant quantum computing (FTQC)"
is made more achievable by first grounding the quantum concepts in familiar mathematical and computational terms. This provides a more solid platform for understanding the complexities of NISQ devices and the requirements for FTQC, such as error correction. The NQCC hackathon report clearly highlights the challenges and limitations encountered when trying to run solutions on current NISQ hardware
, reinforcing the need for students to understand these practical constraints alongside the theoretical potential.

\end{itemize}	

\subsubsection{3. Course Components:}

\begin{itemize}	
	
\item Fundamental Quantum Concepts: This component
can be explicitly structured to introduce qubits, superposition, entanglement, and quantum gates using the suggested analogies.
 
Qudits and basis states as a discrete alphabet/code-words: Introduce computational basis states in terms of discrete symbols or states familiar from coding theory.

Quantum error correction building on classical coding: Explain quantum error correction concepts (like the need for redundancy and error detection/correction) by drawing parallels to classical repetition codes or Hamming codes.

Quantum algorithms via linear algebra and tensor products: Frame quantum algorithms and circuit operations primarily as linear algebra and tensor products, concepts deeply familiar to both mathematics and computer science students. The HHL algorithm
, for example, is a quantum linear solver, a concept readily understandable from a numerical linear algebra perspective.

\item Key Quantum Algorithms: When discussing algorithms like VQE or QAOA
, connect them to classical optimisation techniques where possible. For HHL or QSVT
, emphasise their roots in linear algebra. Shor's algorithm, while likely too complex to implement on NISQ, can be introduced by highlighting the Quantum Fourier Transform's role in period finding, linking it to classical Fourier analysis used in signal processing or algorithms.

\item Application Domains and Use Cases: Present use cases (like optimisation, ML, simulation)
by first describing the classical problem formulation before introducing the quantum approach. This allows students to see the quantum method as an alternative or enhancement to familiar classical techniques, such as vehicle routing problems or solving linear systems.

\item Hardware and Limitations (NISQ): 

Discuss NISQ limitations
in a way that relates back to the properties of the quantum state and operations introduced earlier. Noise, limited connectivity, and coherence times are physical realities that break the ideal mathematical picture – comparing ideal (emulated) results to hardware results, as done in the hackathon
, provides concrete examples of these limitations.

\item Future Directions (FTQC): Introduce FTQC concepts, including the need for more sophisticated error correction, explicitly building on the classical coding analogies used earlier. Discussing the resources (e.g., number of physical qubits per logical qubit, as noted in the context of HHL implementation
) needed for FTQC helps ground the transition in concrete terms.

\end{itemize}	


\subsubsection{4. Teaching and Learning Activities:}

\begin{itemize}	
	
\item Interactive Lectures/Tutorials: These are ideal venues to introduce quantum concepts using the suggested analogies. 

A mathematics professor could potentially lead a session on the linear algebra of quantum states and gates, while a computer science professor could discuss computational complexity or classical algorithm antecedents.

\item Problem-Solving Sessions: Design problems that require students to translate between classical problem formulations and quantum ones, explicitly using the analogies and mathematical/CS structures they are familiar with
.

\item Projects/Team Work: Assign projects where students must apply quantum toolkits (SDKs)
to problems, requiring them to draw on their classical problem-solving skills for aspects like problem formulation, data encoding, result analysis, and classical-quantum hybrid approaches. This mirrors the hackathon experience where teams tackled industrially relevant use cases using available tools and expertise. The team-based nature of hackathons
also facilitates peer-to-peer learning, which is crucial when introducing new concepts.

\item Case Study Analysis: Analyse the NQCC hackathon use cases
and their summaries, focusing on how the teams attempted to map the classical problems onto quantum hardware, the algorithms used, and the limitations encountered. Explicitly discuss how the problems (e.g., optimisation, simulation, ML) relate to classical problem domains and the challenges specific to the quantum implementation. The reports on specific use cases detail the problem types and the attempted quantum solutions
, offering concrete examples of problem formulation and the practical issues faced.

\item Guest Talks/Industry Engagement: Invite speakers who can discuss how quantum computing builds on existing technical fields and the practical challenges of moving from NISQ to FTQC. This provides real-world context for the transition and the skills needed.

\end{itemize}	

By explicitly integrating the use of analogies and highlighting the connections to mathematics and classical computer science throughout these elements, you can effectively leverage the existing intellectual capital of your department. This approach makes the quantum material less daunting, demonstrates its relevance within their established fields, and provides a clear pathway for faculty and students to engage with the application, limitations, and future directions of quantum computing using a toolkit-based methodology, much like the hands-on, problem-focused approach demonstrated by the NQCC hackathon.
	
	\printindex
	
\end{document}

%%% Local Variables:
%%% mode: latex
%%% TeX-master: t
%%% End:
