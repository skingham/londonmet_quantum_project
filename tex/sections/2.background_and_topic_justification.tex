
\section{}

\section{Background and Topic Justification}

\begin{quote}\itshape
	\textbf{\emph{Objective}}
	Layout the justification for topic inclusions and exclusions in an essay format that also acts as a primer on quantum topic for lay readers.
	Touch on studies that support effective pedagogic methods for delivering computing materials.
\end{quote}\ignorespacesafterend


\textbf{\emph{Statement}}


The construction of a quantum computing syllabus for any cohort of students presents challenges. 
The 

\emph{
Talk to the need for high-level understanding of a broad range of quantum topics 
(e.g. being able to converse fluently around a broad range of quantum research areas and applications).
And the need for technical mastery of particular quantum techniques 
(e.g. problem solving using quantum algorithms)
}

There are many good overviews of where the subject is currently \cite{Preskill:2023}.

\cite{Abhijith:2022}

There are also many reviews and surveys on currency algorithm classes \cite{Arnault:2024} \cite{Jordan:2024} 

\emph{
Expound on the problem of the lack of mental models from non-quantum topics (eg universal gates for classical digital computing)
that are all students will know.
Show that the introduction of certain quantum ideas may require the fore-shadowing (? better term) of classical ideas 
that are an analogue or antithesis of the quantum phenomenon (e.g no universal gates in quantum circuits).
}

\emph{
	Introduce the concepts of analogue and digital quantum computing
	How are they affected.  What are the benefits and drawbacks.
}

\emph{
	Introduce the concepts of NISQ and FTQC
	What is decoherence in a simple way
}



\subsection{How does Photonics Demonstrate this?}
\index{Photonics:Overview}

With a crowded arena of terms and technologies, how can we apply outcomes based approach to deliver effective learnings outcomes,
and how, especially to cryptography students?

Photonics poses interesting challenges and opportunities. 
Is both one of the earliest \emph{cite{photonics}} practical quantum technologies 
and an expensive one to set up research for \emph{cite{photonics:costs-of-research}}. 
Yet it has easily demonstrates a number of quantum principles in a way which is easily explainable at an under-graduate science level.
Early \emph{Quantum Key Distribution (QKD)}  \index{Quantum Key Distribution} used quantum properties of photons, 
specifically photon polarisation - which can be demonstrated with polarised sunglasses - to develop the delivery of one-time pads 
for secure communications.


Further, the exposition of the QKD of randomly generated one time pads highlights another recent development
where quantum randomness was demonstrated for the first time recently, using digital quantum computers and RCS
[\href{google jpm 56 qubit QRNG}{https://scitechdaily.com/a-56-qubit-quantum-computer-just-did-what-no-supercomputer-can/}].


\subsection{Where Fools Fear to Tread}

As we see how the cryptographic lens allows us to see how analogue quantum phenomenon can be utilised in cryptographic key protocols,
and the most successful realisation of the promise of digital quantum computing is neatly used to break the one-way functions of asymmetric cryptography,
the temptation would be to look at post-quantum encryption schemes.  

Why turn-away from cryptographic research areas and look at quantum algorithms more generally?
The area of understanding new quantum algorithms, and reverse engineering classical algorithms \emph{cite{something}} is burgeoning.
And the understanding if these new 

Take, as an example, the \href{CRYSTAL-Dilithium NIST "Schnorr-like” lattice-based signature scheme}{https://csrc.nist.gov/CSRC/media/Presentations/crystals-dilithium-round-3-presentation/images-media/session-1-crystals-dilithium-lyubashevsky.pdf}.

Schnorr's identification protocol \emph{cite{Schnorr:1990}} \href{is an example a zero-knowledge protocol}{https://cybersecurity.springeropen.com/articles/10.1186/s42400-023-00198-1} 
that need to convince a verifier knows the discrete logarithm \emph{I may have misunderstood the last point}.
The hardness assumptions are based on the computational problems of the \emph{Shortest Vector Problem (SVP)} 
and \emph{Closest Vector Problem (CVP)}.  
We have jumped from the understanding of quantum phenomenon and promise and pitfalls of quantum technologies to
a specialised topic in mathematics \emph{what's the best description of the topics around SPV and ZK}.  
These are interesting a challenging topics, but should be introduced in their own way.

But moving forward, with the introductory course, into more general quantum algorithms that are not directly applicable to cryptography
provided it's own benefits.
Firstly is, the current generation of analogue quantum annealing computers \emph{cite or reference D-Wave} are being used to attempt
to break certain \emph{Substitution-Permutation Network (SNP)} problems.
This demonstrates that a working understanding of quantum annealing and QUBO solvers is something that is at least in our bailiwick.
More simply, we don't know how the next generation of quantum algorithms could be used to attack encryption protocols, 
so rounding out the programme by looking more generally at these algorithms is justified. 