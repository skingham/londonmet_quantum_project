\section{Introduction}


%\begin{quote}\itshape
%	\textbf{\emph{Objective}}
%	To introduce the purpose, method, and justification for delivering the learning outcomes of the lesson plan. 
%	This section will provide sufficient background on quantum systems, setting the context for the curriculum 
%	and its relevance in preparing students for quantum computing in cryptographic applications.
%\end{quote}\ignorespacesafterend

%%%%%%%%%%%%%%%%%%%%%%%%%%%%%%%%%%%%%%%%%%%%%%%%%%%%%%%%%%%%%%%%%%%%%%%%%%%%%%%%%%%%%%%%%%%%%%%%%%%%%%%%%%%%%%%%%%%%%%%%

In a few short decades quantum computing has evolved from Richard Feynman's a theoretical reflections on quantum simulation 
\cite{Feynman:1986} and David Deutsche's early proposals for universal quantum computation \cite{Deutsch:1985} 
to a \href{https://quantumconsortium.org/stateofthequantumindustry2025}{multi-billion dollar industry}.
The technology is looking set to reshape fields ranging \href{https://blog.google/technology/research/google-quantum-computer-real-world-applications/}
{from drug discovery and advanced materials to financial security and national infrastructure},
and it has already prompted the US National Institute of Standards and Technology (NIST) to reformulate cryptographic standards for a post‑quantum world \cite{NIST:2022}. 

Cryptographic and cyber-security professionals have been on the frontline of these developments.
When Peter Shor published his quantum algorithm for integer factorisation and discrete‑logarithm extraction in the 1990's\cite{Shor:1997},
he ushered in the possibility of \emph{quantum supremacy};
the capacity for quantum computers to solve certain problems that are practically impossible for classical computers.

In the near-term, researchers improving \emph{noisy intermediate‑scale quantum} (NISQ) devices 
(where high error rates and modest qubit counts still constrain circuit depth) 
are still seeking to demonstrate \emph{quantum advantage} in areas such as quantum chemistry, 
quantum cryptography and \emph{quantum machine learning} (QML).   

%These new AI techniques are an active area for data-science and cyber-security researchers and professionals.
%shows that current cryptographic algorithms will broken once 
As we move towards \emph{fault tolerant quantum computers} (FTQCs)
(machines large and robust enough to execute fully fledged, unconstrained quantum algorithms)
%(that can scale to the size and complexity to run unconstrained quantum algorithms)
cryptographers are also focused on the parallel development of \emph{quantum key distribution} (QKD) 
and secure \emph{quantum‑secure communication protocols}.

Quantum algorithms are also inspiring the design of improved classical algorithms 
and so are of wider interest to computer-science researchers.  
Ewin Tang demonstrated this two-way street;
while researching purported quantum advantage in a published quantum‑machine‑learning recommender‑system algorithm,
Tang crafted a purely classical algorithm with comparable performance \cite{Tang:2019}.

Quantum algorithms are inherently dual-use, capable of accelerating scientific breakthroughs whilst also introducing novel threats. 
Consequentially, there is an ethical and social imperative to ensure that quantum expertise is distributed equitably,
rather than being concentrated in a small number of corporate or geopolitical silos.

Providing equitable and responsible access to practical quantum-algorithm technology and training
- envisioned in the UK Government's "\href{https://www.gov.uk/government/publications/national-quantum-strategy?lang=en-gb}{UK National Quantum Strategy}" -
should (must) be a priority for educators and policy makers. 

The challenge, then, is to design and deliver robust syllabi that empower diverse talent pools to 
participate in, and critically evaluate, the evolution of quantum technologies, 
thereby mitigating the risk of bias and unequal benefit.

Students and researchers within the cryptography, data science and applied mathematics arenas 
are uniquely positioned to engage with present-day quantum hardware and software platforms.
Broadening the cadre of professionals who grasp both the promise and the pitfalls of quantum techniques to include these groups 
is achievable, provided we optimise the delivery of quantum programmes.

By tailoring a quantum-computing syllabus to leverage strengths already present in the curriculum 
-- and by offering clear conceptual bridges from classical analogues to quantum phenomena 
%(e.g reversible logic <-> bitwise operations, quantum kernels <-> classical, ) 
-- 
we can shorten the conceptual ramp-up that newcomers face when tackling quantum challenges.
%quantum gate models, quantum variational circuits, and error mitigation techniques, more easily than many others.

Such programmes will also accelerate innovation; 
fresh perspectives coupled with quantum tools can address pressing problems in health, energy and sustainability,
while advancing the Governments mission \textbf{for the UK to be a leading quantum-enabled economy (cite this)}.

The question must be asked as to how to fast‑track a new generation of \emph{quantum‑literate} professionals; 
people who can harness the technology responsibly and while recognising its limitations.
%If they are to be able to converse fluently around range of quantum research areas and applications,
%researchers from a broad range of research areas should be be
This paper argues that the most effective route is to leverage existing expertise and curricula,
tailoring a learning pathway that delivers quantum skills to a broad range of researchers and students.
Cloud-based access to real quantum hardware and powerful simulators, as well as SDKs that can develop deployable quantum-algorithms, 
have made quantum tools sufficiently cheap and ubiquitous enough that 
well‑designed syllabus can serve as a realistic springboard into the coming quantum revolution.

%\enquote{
%	The construction of a quantum computing syllabus for any cohort of students presents challenges. 
%
%	Talk to the need for high-level understanding of a broad range of quantum topics 
%	(e.g. being able to converse fluently around a broad range of quantum research areas and applications).
%	And the need for technical mastery of particular quantum techniques 
%	(e.g. problem solving using quantum algorithms)
%}
%we will tackle several issues; 

The principles of \emph{Outcome-Based Teaching and Learning} (OBTL) provide an appropriate framework for such a curriculum.
\citeauthor{Wong:2011} \cite{Wong:2011} show that
% any course should fulfil part of a program larger curriculum.  
%Their experience demonstrates effective pedagogic methods for delivering computing materials:
effective computing courses share several traits:
\begin{itemize}
	\item \emph{Clear, precise learning outcomes} that state exactly what students should be able to do; for example, "formulate and run a quantum circuit that solves a specified problem".
	\item \emph{Alignment of teaching activities and assessment with those outcomes}: a quantum‑computing syllabus must emphasise hands‑on practice on cloud platforms.
	\item \emph{Explicit mapping of course outcomes to programme‑level goals}, ensuring the quantum unit both builds on existing modules and contributes meaningfully to an integrated degree.
	\item \emph{Foundation‑first delivery}: for cohorts with mixed backgrounds, teaching must begin at the basics and scaffold up to advanced concepts relevant to industry.
	\item \emph{Outcome‑based assessment} that provides timely, constructive feedback on the competencies that matter.
\end{itemize}

By following these guidelines, this dissertation will argue that a learning-outcomes-driven, rather than a content-first, syllabus 
offers the most achievable platform for delivering quantum-computing course 
that can equip a diverse student body with practical skills while reinforcing the broader aims of a modern computing curriculum. 
By focussing on what the student is expected to learn and demonstrate, we can leverage existing strengths 
in cryptography, cyber‑security and data science within other departmental modules
 
The core of this work is delivered in two parts.  
Section 2 proposes, and justifies, a core set of quantum‑computing topics.
%This will be achieved by asking what skill sets would be needed by a student of quantum computing in order to tackle an reasonable research project.  
%and reverse engineering the knowledge needed to support this hypothetical report.
We identify the skills a student would need to have to undertake a realistic research project 
%We will highlight how certain quantum topics can use analogue concepts already being taught within the masters curriculum.
We identify where quantum concepts can be introduced through analogues already taught on the master's programme.
Where such bridges are not feasible, we signpost the additional groundwork that will be required.

%We  be possible for all quantum concepts and topics, and highlighting these will help in the second section of the report.

%By deconstructing one case study we do run the risk of over-fitting the syllabus to the particulars.  
%We will seek to compensate for that by then building out a syllabus-flow from the ground up. 
Section 3 constructs the syllabus flow from first principles. 
%As we do this we attempt to anchor the topics to concrete accessible learning outcomes.
%These outcomes should build solid mental models for quantum computing, 
While basing the design on a single outlook risks over‑fitting to its specifics, 
we mitigate this by anchoring each topic to clear, accessible learning outcomes.
%and provide topics that give students confidence to engage with the quantum computing community.  
These outcomes are intended to build robust mental models and give students the confidence to engage with the wider quantum‑computing community. 
%This last point is helped by OBA as, instead of asking what quantum material should we teach, asks, what must graduates be able to do.
Crucially, OBTL reframes the question from "What quantum material shall we cover?" to "What must graduates be able to do?"
%The flip-side of the risk of our over-fitting course outcomes is the fear of including too much.  
%Focusing on the student's end-state capabilities, we identify only the prerequisite knowledge and skills, pruning everything else. 
To avoid the equal and opposite danger of scope‑creep, we include only the prerequisite knowledge and skills needed to meet those outcomes, pruning everything else.

%The paper is organized as follows.  
%Section 2 reverse engineers a quantum computing final project on
%\emph{'Detecting Market Manipulation via Quantum One-Class Support Vector Machines (OC-SVMs)'} 
%as a realistic, industry-aligned task.  
%By decomposing the project into its supporting competencies
%(quantum kernels, swap-test circuits, $nu$-SVM dual optimisation \emph{add a few things \ldots}) 
%we surface a minimal, sufficient topic set for the course.  
%Section 3 then lays out those topics in a forward-flow syllabus, 
%mapping topics onto specific, measurable learning outcomes.  
%The report finished with section 4 as a review of the out-reach taken with the UK (NQCC) and quantum industry.
Section 4 reflects on the development of this syllabus and delivery of early iterations to actual students.
And Section 5 concludes the report and proposes follow on work to deliver a completed course.
%and details how these efforts impacted the approach of this work.