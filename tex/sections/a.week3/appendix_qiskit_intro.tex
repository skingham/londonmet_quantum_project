    \hypertarget{qiskit-introduction}{%
\section{Week 2: QISKIT Introduction}\label{qiskit-introduction}}

\subsection{SDK Versions and Environment Setup}

For this paper we are using the \href{https://www.ibm.com/quantum/qiskit}{Qiskit SDK} \cite{Qiskit:2023} version \texttt{1.1.0}.
% Our quantum state diagrams are generated using the \href{https://qutip.org}{QuTip}  \emph{Quantum Toolbox in Python} project.
And we will generally want an environment that allows us to run example code in a \href{https://jupyter.org}{Jupyter} notebook.  It is also assumed that 
 a full \texttt{Jupyter} and a \LaTeX environment have been provisioned.
 
 Using \href{https://conda.io/projects/conda/en/latest/user-guide/getting-started.html}{Conda}, our environment is created thus:

\begin{listing}[!ht]
\inputminted{bash}{code/install_qiskit_conda.sh}
\caption{Setting up the QISKIT conda environment}
\label{listing:1}
\end{listing}

Pip can also be used in an appropriate virtual environment:

\begin{listing}[!ht]
\inputminted{bash}{code/install_qiskit_pip.sh}
\caption{Setting up the QISKIT venv/pip environment}
\label{listing:2}
\end{listing}

\subsection{IBM Quantum Platform API Service Setup}

If we only run a emulator locally we don't need to setup an account with the \emph{IBM Quantum Platform} or \emph{IBM Cloud}.  We have registered for the \textbf{Open plan} and so we setup
access to an \href(https://docs.quantum.ibm.com/start/setup-channel){IBM Quantum Channel}.  We can get an \emph{API token} from our IBM Quantum Platform profile page and test the  connection with the following code:

\begin{listing}[!ht]
\inputminted{python}{code/setup_ibm_service_channel.py}
\caption{Test connection to the Qiskit Runtime Service}
\label{listing:3}
\end{listing}

Here are some sample pieces of code to demonstrate building and running
circuits.

All circuits will be run on simulators working on the

    \hypertarget{quantum-fourier-transform-by-hand}{%
\subsection{Quantum Fourier Transform by
Hand}\label{quantum-fourier-transform-by-hand}}

Start with a \texttt{QuantumCircuit} object for a 3-qubit system and
build up a QFT using individual gates.

N.B. Qiskit's least significant bit has the lowest index.

    \begin{tcolorbox}[breakable, size=fbox, boxrule=1pt, pad at break*=1mm,colback=cellbackground, colframe=cellborder]
\prompt{In}{incolor}{1}{\boxspacing}
\begin{Verbatim}[commandchars=\\\{\}]
\PY{k+kn}{from} \PY{n+nn}{qiskit} \PY{k+kn}{import} \PY{n}{QuantumCircuit}
\PY{k+kn}{import} \PY{n+nn}{numpy} \PY{k}{as} \PY{n+nn}{np}
\PY{n}{pi} \PY{o}{=} \PY{n}{np}\PY{o}{.}\PY{n}{pi}

\PY{n}{qft\PYZus{}c} \PY{o}{=} \PY{n}{QuantumCircuit}\PY{p}{(}\PY{l+m+mi}{3}\PY{p}{)}
\PY{c+c1}{\PYZsh{} Hadamard gate is constructed with the \PYZsq{}QuantumCircuit.h\PYZsq{} method}
\PY{n}{qft\PYZus{}c}\PY{o}{.}\PY{n}{h}\PY{p}{(}\PY{l+m+mi}{2}\PY{p}{)}
\PY{c+c1}{\PYZsh{} CROT from qubit 1 to qubit 2}
\PY{n}{qft\PYZus{}c}\PY{o}{.}\PY{n}{cp}\PY{p}{(}\PY{n}{pi}\PY{o}{/}\PY{l+m+mi}{2}\PY{p}{,} \PY{l+m+mi}{1}\PY{p}{,} \PY{l+m+mi}{2}\PY{p}{)}
\PY{c+c1}{\PYZsh{} CROT from qubit 2 to qubit 0}
\PY{n}{qft\PYZus{}c}\PY{o}{.}\PY{n}{cp}\PY{p}{(}\PY{n}{pi}\PY{o}{/}\PY{l+m+mi}{4}\PY{p}{,} \PY{l+m+mi}{0}\PY{p}{,} \PY{l+m+mi}{2}\PY{p}{)} 
\PY{n}{qft\PYZus{}c}\PY{o}{.}\PY{n}{draw}\PY{p}{(}\PY{p}{)}
\end{Verbatim}
\end{tcolorbox}
 
            
\prompt{Out}{outcolor}{1}{}
    
    \begin{center}
    \adjustimage{max size={0.9\linewidth}{0.9\paperheight}}{output_2_0.png}
    \end{center}
    { \hspace*{\fill} \\}
    

    \hypertarget{inverse-qft}{%
\subsection{Inverse QFT}\label{inverse-qft}}

Instead of building all the H, CROT and SWAP gates by hand, we can use a
prebuild QFT class which has this already.

We will use this to build an inverse transform

    \begin{tcolorbox}[breakable, size=fbox, boxrule=1pt, pad at break*=1mm,colback=cellbackground, colframe=cellborder]
\prompt{In}{incolor}{2}{\boxspacing}
\begin{Verbatim}[commandchars=\\\{\}]
\PY{k+kn}{from} \PY{n+nn}{qiskit} \PY{k+kn}{import} \PY{n}{QuantumCircuit}

\PY{n}{nqubits} \PY{o}{=} \PY{l+m+mi}{3}
\PY{n}{number} \PY{o}{=} \PY{l+m+mi}{5}

\PY{n}{circuit} \PY{o}{=} \PY{n}{QuantumCircuit}\PY{p}{(}\PY{n}{nqubits}\PY{p}{)}
\PY{k}{for} \PY{n}{qubit} \PY{o+ow}{in} \PY{n+nb}{range}\PY{p}{(}\PY{n}{nqubits}\PY{p}{)}\PY{p}{:}
    \PY{n}{circuit}\PY{o}{.}\PY{n}{h}\PY{p}{(}\PY{n}{qubit}\PY{p}{)}
\PY{n}{circuit}\PY{o}{.}\PY{n}{p}\PY{p}{(}\PY{n}{number}\PY{o}{*}\PY{n}{pi}\PY{o}{/}\PY{l+m+mi}{4}\PY{p}{,}\PY{l+m+mi}{0}\PY{p}{)}
\PY{n}{circuit}\PY{o}{.}\PY{n}{p}\PY{p}{(}\PY{n}{number}\PY{o}{*}\PY{n}{pi}\PY{o}{/}\PY{l+m+mi}{2}\PY{p}{,}\PY{l+m+mi}{1}\PY{p}{)}
\PY{n}{circuit}\PY{o}{.}\PY{n}{p}\PY{p}{(}\PY{n}{number}\PY{o}{*}\PY{n}{pi}\PY{p}{,}\PY{l+m+mi}{2}\PY{p}{)}
\PY{n}{circuit}\PY{o}{.}\PY{n}{draw}\PY{p}{(}\PY{p}{)}
\end{Verbatim}
\end{tcolorbox}
 
            
\prompt{Out}{outcolor}{2}{}
    
    \begin{center}
    \adjustimage{max size={0.9\linewidth}{0.9\paperheight}}{output_4_0.png}
    \end{center}
    { \hspace*{\fill} \\}
    

    We have Fourier state \(|\tilde{5} \rangle\)

We can create state vector diagrams:

    \begin{tcolorbox}[breakable, size=fbox, boxrule=1pt, pad at break*=1mm,colback=cellbackground, colframe=cellborder]
\prompt{In}{incolor}{3}{\boxspacing}
\begin{Verbatim}[commandchars=\\\{\}]
\PY{k+kn}{from} \PY{n+nn}{qiskit\PYZus{}aer} \PY{k+kn}{import} \PY{n}{AerSimulator}
\PY{k+kn}{from} \PY{n+nn}{qiskit}\PY{n+nn}{.}\PY{n+nn}{visualization} \PY{k+kn}{import} \PY{n}{plot\PYZus{}bloch\PYZus{}multivector}

\PY{n}{qc\PYZus{}init} \PY{o}{=} \PY{n}{circuit}\PY{o}{.}\PY{n}{copy}\PY{p}{(}\PY{p}{)}
\PY{n}{qc\PYZus{}init}\PY{o}{.}\PY{n}{save\PYZus{}statevector}\PY{p}{(}\PY{p}{)}
\PY{n}{sim} \PY{o}{=} \PY{n}{AerSimulator}\PY{p}{(}\PY{p}{)}
\PY{n}{statevector} \PY{o}{=} \PY{n}{sim}\PY{o}{.}\PY{n}{run}\PY{p}{(}\PY{n}{qc\PYZus{}init}\PY{p}{)}\PY{o}{.}\PY{n}{result}\PY{p}{(}\PY{p}{)}\PY{o}{.}\PY{n}{get\PYZus{}statevector}\PY{p}{(}\PY{p}{)}
\PY{n}{plot\PYZus{}bloch\PYZus{}multivector}\PY{p}{(}\PY{n}{statevector}\PY{p}{)}
\end{Verbatim}
\end{tcolorbox}
 
            
\prompt{Out}{outcolor}{3}{}
    
    \begin{center}
    \adjustimage{max size={0.9\linewidth}{0.9\paperheight}}{output_6_0.png}
    \end{center}
    { \hspace*{\fill} \\}
    

    Build the inverse QFT and append it to the circuit.

Also measure all qubit.

    \begin{tcolorbox}[breakable, size=fbox, boxrule=1pt, pad at break*=1mm,colback=cellbackground, colframe=cellborder]
\prompt{In}{incolor}{4}{\boxspacing}
\begin{Verbatim}[commandchars=\\\{\}]
\PY{k+kn}{from} \PY{n+nn}{qiskit}\PY{n+nn}{.}\PY{n+nn}{circuit}\PY{n+nn}{.}\PY{n+nn}{library} \PY{k+kn}{import} \PY{n}{QFT}

\PY{n}{qft} \PY{o}{=} \PY{n}{QFT}\PY{p}{(}\PY{n}{num\PYZus{}qubits}\PY{o}{=}\PY{l+m+mi}{3}\PY{p}{,} \PY{n}{approximation\PYZus{}degree}\PY{o}{=}\PY{l+m+mi}{0}\PY{p}{,} \PY{n}{do\PYZus{}swaps}\PY{o}{=}\PY{k+kc}{True}\PY{p}{,} \PY{n}{inverse}\PY{o}{=}\PY{k+kc}{True}\PY{p}{,} \PY{n}{insert\PYZus{}barriers}\PY{o}{=}\PY{k+kc}{False}\PY{p}{,} \PY{n}{name}\PY{o}{=}\PY{l+s+s1}{\PYZsq{}}\PY{l+s+s1}{qft}\PY{l+s+s1}{\PYZsq{}}\PY{p}{)}\PY{o}{.}\PY{n}{to\PYZus{}gate}\PY{p}{(}\PY{p}{)}
\PY{n}{circuit}\PY{o}{.}\PY{n}{append}\PY{p}{(}\PY{n}{qft}\PY{p}{,} \PY{n}{circuit}\PY{o}{.}\PY{n}{qubits}\PY{p}{[}\PY{p}{:}\PY{n}{nqubits}\PY{p}{]}\PY{p}{)}
\PY{n}{circuit}\PY{o}{.}\PY{n}{measure\PYZus{}all}\PY{p}{(}\PY{p}{)}
\PY{n}{circuit}\PY{o}{.}\PY{n}{decompose}\PY{p}{(}\PY{n}{reps}\PY{o}{=}\PY{l+m+mi}{2}\PY{p}{)}\PY{o}{.}\PY{n}{draw}\PY{p}{(}\PY{p}{)}
\end{Verbatim}
\end{tcolorbox}
 
            
\prompt{Out}{outcolor}{4}{}
    
    \begin{center}
    \adjustimage{max size={0.9\linewidth}{0.9\paperheight}}{output_8_0.png}
    \end{center}
    { \hspace*{\fill} \\}
    

    \begin{tcolorbox}[breakable, size=fbox, boxrule=1pt, pad at break*=1mm,colback=cellbackground, colframe=cellborder]
\prompt{In}{incolor}{5}{\boxspacing}
\begin{Verbatim}[commandchars=\\\{\}]
\PY{k+kn}{from} \PY{n+nn}{qiskit} \PY{k+kn}{import} \PY{n}{transpile}
\PY{k+kn}{from} \PY{n+nn}{qiskit}\PY{n+nn}{.}\PY{n+nn}{visualization} \PY{k+kn}{import} \PY{n}{plot\PYZus{}histogram}
\PY{k+kn}{from} \PY{n+nn}{qiskit\PYZus{}ibm\PYZus{}runtime}\PY{n+nn}{.}\PY{n+nn}{fake\PYZus{}provider} \PY{k+kn}{import} \PY{n}{FakeManilaV2}

\PY{c+c1}{\PYZsh{} A fake 5 qubit backend.}
\PY{n}{backend} \PY{o}{=} \PY{n}{FakeManilaV2}\PY{p}{(}\PY{p}{)}
\PY{n}{shots} \PY{o}{=} \PY{l+m+mi}{2048}
\PY{n}{transpiled\PYZus{}qc} \PY{o}{=} \PY{n}{transpile}\PY{p}{(}\PY{n}{circuit}\PY{p}{,} \PY{n}{backend}\PY{p}{,} \PY{n}{optimization\PYZus{}level}\PY{o}{=}\PY{l+m+mi}{3}\PY{p}{)}

\PY{n}{job} \PY{o}{=} \PY{n}{backend}\PY{o}{.}\PY{n}{run}\PY{p}{(}\PY{n}{transpiled\PYZus{}qc}\PY{p}{,} \PY{n}{shots}\PY{o}{=}\PY{n}{shots}\PY{p}{)}

\PY{n}{counts} \PY{o}{=} \PY{n}{job}\PY{o}{.}\PY{n}{result}\PY{p}{(}\PY{p}{)}\PY{o}{.}\PY{n}{get\PYZus{}counts}\PY{p}{(}\PY{p}{)}
\PY{n}{plot\PYZus{}histogram}\PY{p}{(}\PY{n}{counts}\PY{p}{)}
\end{Verbatim}
\end{tcolorbox}
 
            
\prompt{Out}{outcolor}{5}{}
    
    \begin{center}
    \adjustimage{max size={0.9\linewidth}{0.9\paperheight}}{output_9_0.png}
    \end{center}
    { \hspace*{\fill} \\}
    

    And we recover 5 as 101!
