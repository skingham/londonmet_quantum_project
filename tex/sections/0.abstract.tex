\begin{abstract}
	

%Give the recent advancement and ubiquity of quantum platforms and advanced tooling,
%an introductory course in quantum computation and quantum information 
%is not only an achievable objective for masters and undergraduate students, 
%but desirable for the democratisation of these new technological advances.

Recent progress in cloud-based quantum hardware, high-fidelity simulators and accessible SDKs 
has lowered the barrier to hands-on quantum system experimentation, making an introductory course in quantum computation 
and information both feasible for undergraduate and master's cohorts and essential for the 
democratisation of this emerging technology.

This research project seeks to deliver a comprehensive framework for teaching 
introductory quantum computing algorithms, focusing on the core foundational knowledge needed,
along with a targeted delivery of topics that resonate with students of mathematics, cryptography and data science.

This paper develops a program using Software Development Kits (SDKs) from leading quantum hardware and software companies
that deliver world class quantum systems, emulators and quantum simulators via cloud environments.
This work will identify the essential learning outcomes required for new researchers to become proficient
in constructing quantum circuits of increasing complexity.

%This project develops a teaching framework that couples core mathematical foundations with carefully selected topics designed to resonate with students of mathematics, cryptography and data science. Leveraging industry-standard toolchains of IBM\,Qiskit, Google\,Cirq, Pennylane and D-Wave\,Leap, the curriculum guides learners from single-qubit gates to the construction of increasingly sophisticated circuits on real or emulated hardware.

Starting with the core mathematical concepts, we present the building blocks of current quantum 
algorithms through the lens of Shor's landmark 1990's quantum algorithm for integer factorisation and discrete logarithms.
Using these well understood and demonstrable building blocks, this work then introduces more advanced topics
in a manner that new students will feel confident in further research and work in this exciting industry.

% Shor's landmark algorithm for integer factorisation and discrete logarithms provides the organising lens: its component techniques (state preparation, modular arithmetic, Quantum Fourier Transform and amplitude amplification) are unpacked, demonstrated on contemporary NISQ devices, and then extended to more advanced themes such as block encoding, quantum machine-learning kernels and hybrid classical-quantum workflows.

The programme focuses on using an Outcomes Based Learning (OBL) approach that should give students: 
a practical, hands-on, pathway into the field of quantum computing; 
understanding the limitations of delivering solutions using noisy hardware;
to critically evaluate claims of quantum advantage in current literature; and,
to gain the skills to compete in this rapidly evolving and exciting area of research.

%Explicit learning outcomes ensure graduates can (i) build and execute quantum circuits, (ii) articulate the limitations of noisy hardware, and (iii) critically evaluate claims of quantum advantage in current literature. The framework thus offers a practical pathway for new entrants to gain the competencies required to contribute meaningfully to this rapidly evolving field.


%We will perform a survey of available quantum computing platforms, evaluating them for their usability, scalability
%and suitability for cryptographic applications.  We will look to implement key quantum algorithms used in
%cryptanalysis (Shor, Grover, Quantum phase estimation algorithm (QPE), Quadratic Unconstrained Binary Optimization
%(Ising-QUBO), etc.), as well as introducing the supporting quantum principles and
%mathematical models that underpin these algorithms (Hidden-subgroup problems, combinatorics, optimizations, etc.).
%Further, we will look to see examples of how researchers are applying quantum techniques to attempt to solve the
%Shortest Vector Problem (SVP) and to attack Substitution-Permutation Networks (SPN), which underpins AES symmetric
%encryption.
%Some approaches, such as Quantum Annealing and Coherent Ising Machines, are claiming to show evidence of quantum
%advantage and are of great interest.

%The expectation of this project is to offer a practical pathway for new entrants to the field of quantum computing
%to gain the skills to compete in this rapidly evolving and exciting area of research.
\end{abstract}
