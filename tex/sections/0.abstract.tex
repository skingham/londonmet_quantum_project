\begin{abstract}
	
Recent advancement and ubiquity of quantum platforms and advanced tooling,
an introductory course in quantum computation and quantum information 
is not only an achievable objective for masters and undergraduate students, 
but desirable for the democratisation of these new technological advances.

This research project seeks to deliver a comprehensive framework for teaching 
introductory quantum computing algorithms, focusing on the core foundational knowledge needed,
along with a targeted delivery of topics that resonate with students of mathematics, cryptography and data science.

Delivering a program using Software Development Kits (SDKs) from leading quantum hardware and software companies
that are deliver world class quantum systems, emulators and quantum simulators via cloud environments,
this work to identify the essential learning outcomes required for new researchers to become proficient
in constructing quantum circuits of increasing complexity.

Starting with the core mathematical concepts, we present the building blocks of current quantum 
algorithms through the lens of Shor's landmark 1990's quantum algorithm for integer factorisation and discrete logarithms.
Using these well understood and demonstrable building blocks, this work then introduces more advanced topics
in a manner that new students will feel confident in further research and work in this exciting industry.

The programme focuses on using an Outcomes Based Learning (OBL) approach that should give students 
a practical, hands-on, pathway into the field of quantum computing, 
understanding the limitations of delivering solutions using noisy hardware,
and to critically evaluate claims of quantum advantage in current literature.
to gain the skills to compete in this rapidly evolving and exciting area of research.


%We will perform a survey of available quantum computing platforms, evaluating them for their usability, scalability
%and suitability for cryptographic applications.  We will look to implement key quantum algorithms used in
%cryptanalysis (Shor, Grover, Quantum phase estimation algorithm (QPE), Quadratic Unconstrained Binary Optimization
%(Ising-QUBO), etc.), as well as introducing the supporting quantum principles and
%mathematical models that underpin these algorithms (Hidden-subgroup problems, combinatorics, optimizations, etc.).
%Further, we will look to see examples of how researchers are applying quantum techniques to attempt to solve the
%Shortest Vector Problem (SVP) and to attack Substitution–Permutation Networks (SPN), which underpins AES symmetric
%encryption.
%Some approaches, such as Quantum Annealing and Coherent Ising Machines, are claiming to show evidence of quantum
%advantage and are of great interest.

%The expectation of this project is to offer a practical pathway for new entrants to the field of quantum computing
%to gain the skills to compete in this rapidly evolving and exciting area of research.
\end{abstract}
