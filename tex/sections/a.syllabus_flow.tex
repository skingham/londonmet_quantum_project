\section{Quantum Computing Syllabus Flow for Cryptography}


%%%%%%%%%%%%%%%%%%%%%%%%%%%%%%%%%%%%%%%%%%%%%%%%%%%%%%%%%%%%%%%%%%%%%%%%%%%%%%%%%
%%%%%%%%%%%%%%%%%%%%%%%%%%%%%%%%%%%%%%%%%%%%%%%%%%%%%%%%%%%%%%%%%%%%%%%%%%%%%%%%%
\subsection{Quantum Foundations and Hardware}


%%%%%%%%%%%%%%%%%%%%%%%%%%%%%%%%%%%%%%%%%%%%%%%%%%%%%%%%%%%%%%%%%%%%%%%%%%%%%%%%%
\subsubsection{Quantum computing history}

\textbf{Outcomes}: be able to describe the key contributions of Feynman and Deutsch to the field of quantum computing.

\begin{itemize}
	\item Basic quantum computing history (e.g., Feynman, Deutsch)
\end{itemize}

\textbf{Resources}:

\begin{itemize}
	\item Feynman (1982) "Simulating Physics with Computers"; 

	\item Deutsch (1985) "Quantum theory, the Church-Turing principle and the universal quantum computer."
\end{itemize}


%%%%%%%%%%%%%%%%%%%%%%%%%%%%%%%%%%%%%%%%%%%%%%%%%%%%%%%%%%%%%%%%%%%%%%%%%%%%%%%%%
\subsubsection{Quantum Computing Overview}

\begin{itemize}
	\item Classical vs. quantum computing
	\item Superposition, entanglement, measurement, quantum bits vs classical bits
\end{itemize}
	

\textbf{Resources}:

\begin{itemize}
	\item \citetitle{Nielsen:2000} by \citeauthor{Nielsen:2000} (2000), Chapters 1–2.
	\item Caltex material \footnote{\href{http://users.cms.caltech.edu/~vidick/teaching/120_qcrypto/LN_Week0.pdf}{Caltech edX Week 0 notes} \label{caltech:edX:week0}} %or ref like \footref{caltech:edX:week0}. 
	\item \citetitle{Schumacher:2010}
\end{itemize}


%%%%%%%%%%%%%%%%%%%%%%%%%%%%%%%%%%%%%%%%%%%%%%%%%%%%%%%%%%%%%%%%%%%%%%%%%%%%%%%%%
\subsubsection{The Qubit: Representations and Realizations}

\begin{itemize}
	\item Bloch sphere, quantum state vectors, coherence/decoherence
	
	\item Hardware: superconductors, ion traps, quantum dots, initial photonic systems
\end{itemize}

\textbf{Resources}:

\begin{itemize}
	\item Preskill (2018); 

	\item Monroe et al. (2021); 

	\item Arute et al. (2019)
\end{itemize}


%%%%%%%%%%%%%%%%%%%%%%%%%%%%%%%%%%%%%%%%%%%%%%%%%%%%%%%%%%%%%%%%%%%%%%%%%%%%%%%%%
\subsubsection{Modern Photonic Quantum Machinery}

\begin{itemize}
	\item Photonic qubits, single-photon detection, integrated photonic circuits
	
	\item Industry examples: PsiQuantum, ORCA Computing
\end{itemize}

\textbf{Resources}:

\begin{itemize}
	\item Rudolph (2017);

	\item Bartolucci et al. (2023)
\end{itemize}

	
%%%%%%%%%%%%%%%%%%%%%%%%%%%%%%%%%%%%%%%%%%%%%%%%%%%%%%%%%%%%%%%%%%%%%%%%%%%%%%%%%
%%%%%%%%%%%%%%%%%%%%%%%%%%%%%%%%%%%%%%%%%%%%%%%%%%%%%%%%%%%%%%%%%%%%%%%%%%%%%%%%%
\subsection{Quantum Gates and Circuits}

	
%%%%%%%%%%%%%%%%%%%%%%%%%%%%%%%%%%%%%%%%%%%%%%%%%%%%%%%%%%%%%%%%%%%%%%%%%%%%%%%%%
\subsubsection{Basic Gates \& Operations}

\textbf{Concepts Covered}:

\begin{itemize}
	\item Bra-ket notation and state representation; a way of writing vectors in a 2-D vector space: $|v \rangle \in \mathbb{C}^2$
\index{Bra-ket Notation}
	\item Matrix transformations of quantum states and gates
	\item Basic gates: Pauli (X, Y, Z), Hadamard (H), CNOT, Phase gates, and controlled gates
	\item Principle of reversibility: quantum operations as unitary transformations, implications for circuit construction, contrast to classical irreversibility.
	\item No-Cloning theorem: proofs and intuitive reasoning, consequences for quantum communication and cryptography.
\end{itemize}
	
\textbf{Workshops}:

\begin{itemize}	
	\item IBM Qiskit (core gate library, interactive circuit composer)
	\item Google Cirq (custom gate implementation, visualizations)
\end{itemize}

\textbf{Resources}:

\begin{itemize}
	\item Original no-cloning paper: Wootters \& Zurek, "A Single Quantum Cannot Be Cloned," Nature, 1982, doi:10.1038/299802a0
	\item Explanation of reversibility: \citeauthor{Nielsen:2000}, Quantum Computation and Quantum Information (Chapter 4, sections on unitarity and reversibility).
\end{itemize}


%%%%%%%%%%%%%%%%%%%%%%%%%%%%%%%%%%%%%%%%%%%%%%%%%%%%%%%%%%%%%%%%%%%%%%%%%%%%%%%%%
\subsubsection{Tensor Mathematics and Circuit Composition}

\begin{itemize}
	\item Problems \ldots
\end{itemize}

\textbf{Workshop SDKs/Platforms}:

\begin{itemize}
	\item IBM Qiskit Composer (interactive drag-and-drop circuit composer)
	\item Google Cirq tutorials (introductory lab exercises)
	\item Julia QML/Yao.jl (fast simulations, tensor-network circuits, tensor operations)
	\item Pennylane (tensor circuit building, differentiable programming)
\end{itemize}


%%%%%%%%%%%%%%%%%%%%%%%%%%%%%%%%%%%%%%%%%%%%%%%%%%%%%%%%%%%%%%%%%%%%%%%%%%%%%%%%%
\subsubsection{Intuitive Explanation of Errors and Noise}

Visual examples:

\begin{itemize}
	\item bit-flip errors
	\item phase-flip errors
\end{itemize}


%%%%%%%%%%%%%%%%%%%%%%%%%%%%%%%%%%%%%%%%%%%%%%%%%%%%%%%%%%%%%%%%%%%%%%%%%%%%%%%%%
\subsubsection{NISQ Devices}

\textbf{Workshop SDKs/Platforms}:

\begin{itemize}
	\item IBM Quantum Experience (real-device demonstrations, noise simulations)
	\item Google Cirq (customizable noise models)
\end{itemize}


%%%%%%%%%%%%%%%%%%%%%%%%%%%%%%%%%%%%%%%%%%%%%%%%%%%%%%%%%%%%%%%%%%%%%%%%%%%%%%%%%
\subsubsection{Error Correction Codes}
	
\begin{itemize}
	\item Surface Codes
	\item Colour Codes
\end{itemize}

\textbf{Workshop SDKs/Platforms}:

\begin{itemize}
	\item IBM Qiskit Noise Simulator (interactive noise examples)
	\item IBM Qiskit Ignis (specialized quantum error correction toolkit)
	\item Google Cirq (topological code simulations, custom circuit design)
	\item (Additional tool): Stim (Google's specialized quantum error correction simulator)
\end{itemize}


%%%%%%%%%%%%%%%%%%%%%%%%%%%%%%%%%%%%%%%%%%%%%%%%%%%%%%%%%%%%%%%%%%%%%%%%%%%%%%%%%
%%%%%%%%%%%%%%%%%%%%%%%%%%%%%%%%%%%%%%%%%%%%%%%%%%%%%%%%%%%%%%%%%%%%%%%%%%%%%%%%%
\subsection{Quantum Algorithms and Classical Cryptography}


%%%%%%%%%%%%%%%%%%%%%%%%%%%%%%%%%%%%%%%%%%%%%%%%%%%%%%%%%%%%%%%%%%%%%%%%%%%%%%%%%
\subsubsection{Classical Cryptography Problems}

Touch on the maths from prior units which are relevant (modular arithmetic, integer factorisation and periods).


%%%%%%%%%%%%%%%%%%%%%%%%%%%%%%%%%%%%%%%%%%%%%%%%%%%%%%%%%%%%%%%%%%%%%%%%%%%%%%%%%
\subsubsection{Quantum Fourier Transform and Modular Arithmetic}

\textbf{Workshop SDKs/Platforms}:

\begin{itemize}
	\item IBM Qiskit (QFT tutorials, modular exponentiation implementations)
	\item Pennylane (QFT example notebooks)
\end{itemize}


%%%%%%%%%%%%%%%%%%%%%%%%%%%%%%%%%%%%%%%%%%%%%%%%%%%%%%%%%%%%%%%%%%%%%%%%%%%%%%%%%
\subsubsection{Shor's Algorithm}

\textbf{Workshop SDKs/Platforms}:

\begin{itemize}
	\item IBM Qiskit (detailed practical implementations, interactive tutorials)
	\item Pennylane (modular arithmetic circuits)
\end{itemize}


%%%%%%%%%%%%%%%%%%%%%%%%%%%%%%%%%%%%%%%%%%%%%%%%%%%%%%%%%%%%%%%%%%%%%%%%%%%%%%%%%
\subsubsection{Grover’s Algorithm}

\textbf{Workshop SDKs/Platforms}:

\begin{itemize}
	\item IBM Qiskit (Grover's search algorithm tutorials, interactive examples)
\end{itemize}


%%%%%%%%%%%%%%%%%%%%%%%%%%%%%%%%%%%%%%%%%%%%%%%%%%%%%%%%%%%%%%%%%%%%%%%%%%%%%%%%%
\subsubsection{Deutsch–Jozsa and Simon’s Algorithms}

\textbf{Workshop SDKs/Platforms}:

\begin{itemize}
	\item IBM Qiskit (clear examples and interactive notebooks)
	\item (oracle problems, hands-on demonstrations)
\end{itemize}


%%%%%%%%%%%%%%%%%%%%%%%%%%%%%%%%%%%%%%%%%%%%%%%%%%%%%%%%%%%%%%%%%%%%%%%%%%%%%%%%%
%%%%%%%%%%%%%%%%%%%%%%%%%%%%%%%%%%%%%%%%%%%%%%%%%%%%%%%%%%%%%%%%%%%%%%%%%%%%%%%%%
\subsection{Advanced Quantum Algorithms}
	
	
%%%%%%%%%%%%%%%%%%%%%%%%%%%%%%%%%%%%%%%%%%%%%%%%%%%%%%%%%%%%%%%%%%%%%%%%%%%%%%%%%
\subsubsection{Introduction to Classical Optimisation/ML}

\emph{hopefully aligns with some data-analysis units}

\textbf{Workshop SDKs/Platforms}:

\begin{itemize}
	\item Pennylane (smooth classical-to-quantum transition tutorials)
	\item Julia QML/Yao tutorials (high-performance QML simulations)
\end{itemize}


%%%%%%%%%%%%%%%%%%%%%%%%%%%%%%%%%%%%%%%%%%%%%%%%%%%%%%%%%%%%%%%%%%%%%%%%%%%%%%%%%
\subsubsection{Quantum Annealing and D-Wave Systems}
	
\textbf{Workshop SDKs/Platforms}:

\begin{itemize}
	\item D-Wave Leap (Quantum annealing experiments, practical QUBO solutions)
\end{itemize}


%%%%%%%%%%%%%%%%%%%%%%%%%%%%%%%%%%%%%%%%%%%%%%%%%%%%%%%%%%%%%%%%%%%%%%%%%%%%%%%%%
\subsubsection{Quantum Unconstrained Binary Optimization (QUBO)}
	
\textbf{Workshop SDKs/Platforms}:

\begin{itemize}
	\item D-Wave Leap platform (directly implement optimization problems)
	\item Pennylane (variational quantum algorithms for optimization)
	\item Google Cirq (QAOA tutorials)
\end{itemize}


%%%%%%%%%%%%%%%%%%%%%%%%%%%%%%%%%%%%%%%%%%%%%%%%%%%%%%%%%%%%%%%%%%%%%%%%%%%%%%%%%
\subsubsection{Quantum Algorithms for Graph Problems}
	
\textbf{Workshop SDKs/Platforms}:

\begin{itemize}
	\item Google Cirq (QAOA for MaxCut, detailed graph problems examples)
	\item D-Wave Leap (Ising models, graph optimization problems)
\end{itemize}


%%%%%%%%%%%%%%%%%%%%%%%%%%%%%%%%%%%%%%%%%%%%%%%%%%%%%%%%%%%%%%%%%%%%%%%%%%%%%%%%%
\subsubsection{Quantum Machine Learning (QML)}
	
\textbf{Workshop SDKs/Platforms}:

\begin{itemize}
	\item Pennylane (core QML package, variational circuits, quantum neural nets)
	\item Julia Quantum ML/QML.jl (efficient QML experiments, classical-quantum hybrid models)
	\item TensorFlow Quantum (Cirq-based, optional for broader ML integrations)
\end{itemize}

\pagebreak