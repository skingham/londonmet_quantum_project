\section{Introduction}


\begin{quote}\itshape
	\textbf{\emph{Objective}}
	To introduce the purpose, method, and justification for delivering the learning outcomes of the lesson plan. This section will provide sufficient background on quantum systems, setting the context for the curriculum and its relevance in preparing students for quantum computing in cryptographic applications.
\end{quote}\ignorespacesafterend

Advances in quantum computing are regularly in the press \emph{cite google:willow:2024} with corresponding pronouncements
of impending cataclysms \emph{cite ava-labs-c-founder-on-bitcoin-vulnerability}.  It is increasingly more important that new
researchers from all backgrounds have the opportunity to engage with these new and exciting technologies, to understand the
promise and limitations, the proven from the possible, and the hype from the reality.  By engaging students equally, there
is hope that solutions to a broad spectrum of research areas will be accomplished.

How best to engage students and researchers with quantum computing is the aim of this paper.  New quantum research papers are
being released with seeming exponential growth.  And with the boom and bust cycles of quantum computing manufactures \emph{cite the-recent-bust}
\emph{cite the-more-recent-boom} there are also more competing technologies vying for attention.

The aim of this paper is to develop a syllabus suitable for students interesting in gaining practical skills in quantum computing, 
to a level that can be applied to a number of application domains.

Much ink is spilled on the topic of quantum attacks on private key encryption such as RSA, Diffie-Hellman, El-Gamal and eliptic curve schemes.
These rely on the difficulty of solving hidden-subgroup problems.  But there is also interest in attacking \emph{cite the Chinese paper} 
symmetric schemes that use substitution-permutation networks.  So it is reasonable that cryptographic researchers have a broad understanding 
of the algorithms being tackled using quantum systems.

\begin{quote}\itshape
\textbf{\emph{How to Achieve This}}

\emph{Method and Justification}: Clearly state why a structured lesson plan is necessary to equip students with the skills to work with quantum algorithms and hardware. Justify the choice of using quantum simulators, emulators, and physical hardware in the curriculum, emphasizing the relevance of hands-on experience in mastering quantum principles.
\end{quote}\ignorespacesafterend

To create an effective short syllabus on practical quantum computing we need clear objectives and goals.  
The UK National Quantum Computing Centre runs an annual Quantum Hackaton \cite{NQCC:2024}.
Their reports on the problems given to teams, the technologies used and the results give concise overview of the 
techniques and systems that the syllabus should aim to cover.\cite{NQCC:2023}.

\begin{quote}\itshape
	\textbf{\emph{Background}}
	Provide a brief history and overview of quantum computing, covering essential topics such as qubits, quantum gates, and the basics of quantum mechanics. 
	
	\textbf{\emph{Challenges}}
	\emph{Balancing Detail and Clarity}: Given the technical complexity of quantum systems, it may be challenging to provide sufficient background without overwhelming the reader. Aim to strike a balance between depth and accessibility, focusing on the essential concepts needed for the lesson plan.
\end{quote}\ignorespacesafterend


Outcome-based teaching and learning \cite{Spady:1982} has been used to successfully develop computing syllabi \cite{Wong:2011}
and is a useful tool to ground the structure of the lesson plans developed here. \emph{expand on the outline of the tools and how they are appropriate}

This focus on outcomes is particularly important as the field of quantum computing is expanding rapidly and it would be impossible to cover all
aspects of the field, the background quantum physics and historical context, and deliver practical mastery of the current toolsets.
But this outcome focus must be supported by the necessary mental models and mathematical understanding of how modern quantum systems work and 
the trade-offs that must be understood to choose the correct computing systems and quantum algorithms for a given application domain.

Fortunately other researcher have grappled with some of these issues and there are multiple resources with the typology of quantum algorithms
\cite{Arnault:2024} \cite{Montanaro:2016}.  \citeauthor{Arnault:2024} helpfully divides the landscape into quantum primitives, mathematical 
classes and application domains.  
The focus of this paper is to chart a course through this large field of possibilities to find a path that builds from well understood 
primitive building blocks, understanding their mathematical techniques and the expected speed-ups, if any, over classical techniques,
and defining learning outcomes that can be used, flexibly, in a wide variety of application domains.  
The earliest practical quantum circuits of Shor's paper attacked asymmetric cryptographic systems, and this is a good place to start
as we focus on building well defined learning goals with clear assessments to ensure students can reach these goals \cite{Spady:1982}.


\begin{quote}\itshape
	\textbf{\emph{Challenges}}
	\emph{Supporting Quantum Topics}: It's our case (and in \cite{Abhijith:2018}) that we do not need to exhaustively describe quantum
	mechanics in order to gain technical mastery of quantum computing platforms.  There are some principals that have strong explanatory
	effect ($Pr(q) = ||q||^{2}$ and Amplitude Amplification running in $\frac{1}{\sqrt(p)}$ p-trials, reversibility requirement of 
	quantum circuits). Expand this point here.
\end{quote}\ignorespacesafterend

The organisation of this paper is as follows.  Section 2 reviews the current state of quantum computing and broad classes of algorithms
in order to understand the current challenges of achieving quantum supremacy and the choices open to practitioners in solving different 
classes of problems.  
This section then surveys and compares the software development kits (SDKs) that can target different hardware platforms.  
Section 3 describes roadmap of quantum primitives and mathematical classes chosen as our focus and demonstrates how these build to
solve an initial set of problem domains.  Section 4 is a reflective analysis of the roadmap and learning outcomes.  
Section 5 ends with a further developments for the syllabus.

