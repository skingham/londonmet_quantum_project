\section{Introduction}


\begin{quote}\itshape
	\textbf{\emph{Objective}}
	To introduce the purpose, method, and justification for delivering the learning outcomes of the lesson plan. 
	This section will provide sufficient background on quantum systems, setting the context for the curriculum 
	and its relevance in preparing students for quantum computing in cryptographic applications.
\end{quote}\ignorespacesafterend


\textbf{\emph{Problem Statement}}

In a few short decades Quantum Computing has come from a theoretical curiosity \cite{Deutsch:1985} \cite{Feynman:1986} 
to a \href{https://quantumconsortium.org/stateofthequantumindustry2025}{multi-billion dollar industry}.
The technology is looking set to reshape critical domains; \href{https://blog.google/technology/research/google-quantum-computer-real-world-applications/}
{from drug discovery and advanced materials to financial security and national infrastructre}.

Quantum algorithms are inherently dual-use 
-- capable of accelerating breakthroughs while also threatening current cryptographic systems -- 
there is an ethical and social imperative to ensure that quantum expertise is distributed equitably
rather than being concentrated into a handful of corporate or geopolitical silos.

Providing equitable and responsible access to practical quantum algorithm technology \& training should then 
be a priority for educators and policy makers[\href{https://www.gov.uk/government/publications/national-quantum-strategy?lang=en-gb}{UK National Quantum Strategy}]. 

The problem then becomes, how to develop and deliver robust syllabi to empower diverse talent pools to 
participate in -- and critically evaluate -- the development of quantum technologies, 
mitigating risks of bias and unequal benefit.

Students and researchers within the cryptography, data science and applied mathematics arenas 
are uniquely positioned to engage with present-day quantum hardware and software development platforms.
Broadening the set of professionals who understand promise and pitfalls of quantum techniques to these groups 
feels like an achievable outcome if we can optimise the delivery of quantum learnings.

By tailoring quantum-computing curricula to leverage existing strengths 
-- providing conceptual bridges to quantum phenomena -- 
this cohort should grasp quantum gate models, quantum variational circuits, and error mitigation techniques
more easily than many others.

The additional benefit of the delivery of such programmes will be to accelerate innovation, 
by coupling fresh perspectives with quantum tools to tackle challenges in health, energy and sustainability,
and deliver on the governments mission \textbf{for the UK to be a leading quantum-enabled economy}.


\textbf{\emph{Aim}}

Fast‑track a new generation of "quantum‑literate" professionals who are equipped to harness the technology responsibly and recognize its limitations.

The paper is organized as follows \ldots