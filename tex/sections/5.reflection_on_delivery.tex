%%%%%%%%%%%%%%%%%%%%%%%%%%%%%%%%%%%%%%%%%%%%%%%%%%%%%%%%%%%%%%%%%%%%%%%%%%%%%%%%%%%%%%%%%%%%%%%%%%%%%%%%%%%%%%%%%%%%%%%%
\section{Project Reflection}

The development of this paper has been quite intricate.  
%An implied assertion of this paper that learning quantum computing techniques 
%is both rewarding and not as difficult as may be supposed.
It rests on the assertion that learning quantum‑computing techniques 
is not only rewarding, but less forbidding than commonly assumed.
My experience, as my own understanding of quantum information and computation matured during this project, 
has reinforced this view; although pulling together a coherent teaching plan has been a challenge.

Along side the development of the course outline and tutorial materials, 
an informal pilot has been running to road-test the materials and approach. 
Although I don't have measurable or reportable responses from the cohort for this report, 
anecdotally the material appears to be engaging to students new to the field.

Three realisations have helped crystallise the final approach and content:
\begin{enumerate}
\item \emph{Make background assumptions explicit}: 
During the delivery of a tutorial on single qubit transforms and their related gates,
when noting that any arbitrary unitary operator can be decomposed 
into a set of single-qubit gates plus the two-qubit controlled-NOT (CNOT) gate \cite{Nielsen:2010},
the analogy failed when I compared this to the universality of NAND and NOR gates 
in the construction of classical logic gates \cite{Wikipedia:UniversalLogicGates}.
It demonstrated to me that the introduction material needs to be self-contained,
supplying all foundational material.

\item \emph{Practising what I preach}: This fed into the realisation that, 
although the paper advocates Outcome‑Based Teaching and Learning (OBTL), 
I wasn't embracing this, centring on content not outcomes.  
This realisation helped reorient both the tutorial content to focus on explicit learning objectives,
and to simplify the syllabus outline to become a clear map from foundations to the programme goal \cite{Wong:2011}.

\item \emph{Shor as a centrepiece}: Whilst refactoring all the material that I'd collected, 
that second realisation allowed me to see that one of those earliest papers, by Peter Shor, 
embodied many core quantum building blocks that later research generalised and developed into new and exciting techniques. 
Shor (along with Grover) also provides natural talking points for hybrid computing, the limitations of NISQ devices, 
and architectural decisions taken when delivering end-to-end quantum solutions.  
Shor's integer factorisation algorithm, itself, becomes an analogue for more advanced quantum computing techniques
and centrepiece for an introductory programme.
\end{enumerate}