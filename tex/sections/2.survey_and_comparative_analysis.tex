\section{Survey of Quantum Computing and Comparative Analysis}

This survey seeks to identify quantum computing industry leaders with technology road-maps that would align with students
and researchers having confidence that the technology will evolve and be a stable platform for research and development.


\begin{quote}\itshape
	\textbf{\emph{How to Achieve This}}
	\emph{Background Information}: Include explanations of key quantum algorithms (like Shor’s and Grover’s) and their significance in fields like cryptography.
\end{quote}\ignorespacesafterend

\subsection{Digital and Analogue Quantum Computing}

Two classes of analogue quantum computers

The survey traces the development of quantum algorithms from Deutsch's algorithm (1985) and Shor's algorithm (1994) to modern-day 
quantum primitives and applications.
Shor's algorithm is highlighted as the paradigm-shifting discovery, marking the transition from black-box quantum speedups to real-world applications.

Quantum computing has evolved from theoretical foundations to an exciting field with the potential to transform computation. Its origins trace back to the 1980s when pioneers like Richard Feynman proposed using quantum systems to simulate quantum phenomena, addressing the limitations of classical computers in modeling quantum mechanics. In 1985, David Deutsch formalized the concept of quantum computation, distinguishing programmable quantum computers from quantum simulators.

The field gained significant momentum in 1994 with Shor’s algorithm, which demonstrated quantum computers' ability to factorize large numbers exponentially faster than classical algorithms. This breakthrough highlighted quantum computing's promise for solving real-world problems, especially in cryptography. Around the same time, algorithms like Grover’s search algorithm showed quadratic speedups for database searches, establishing a foundation for quantum algorithm research.

Over the last three decades, the development of quantum algorithms has diversified. Exact algorithms, such as Shor’s and Grover’s, provide mathematically proven speedups for specific problems. However, the current focus has shifted towards heuristic algorithms, like the Quantum Approximate Optimization Algorithm (QAOA) and Variational Quantum Eigensolver (VQE), which are designed for Noisy Intermediate-Scale Quantum (NISQ) devices. These heuristic methods, inspired by classical principles and quantum phenomena, lack provable guarantees but hold promise for practical applications.

NISQ vs. LSQ: Challenges and Promises

The current era of quantum computing is defined by NISQ devices, which operate with limited qubits and significant noise. While these devices have demonstrated quantum supremacy for specialized problems, such as Google’s milestone in 2019, their practical applications are constrained by high error rates and a lack of scalability. Researchers are exploring hybrid quantum-classical methods to mitigate these limitations, but achieving reliable computation requires transitioning to Large-Scale Quantum (LSQ) systems.

LSQ systems, characterized by millions of error-corrected qubits, represent the future of quantum computing. They promise to unlock the full potential of quantum algorithms, enabling breakthroughs in cryptography, optimization, and scientific simulations. However, the path to LSQ is fraught with challenges, including hardware scalability, error correction, and managing the complexity of quantum systems.
Exact vs. Heuristic Algorithms

Exact Algorithms: These include Shor’s and Grover’s algorithms, which provide provable speedups for tasks like factoring and search.

Heuristic Algorithms: NISQ-compatible methods like QAOA and VQE offer flexibility and adaptability but lack formal proofs of quantum advantage. They rely on empirical performance and are currently the focus of research in optimization, chemistry, and machine learning.

This dichotomy illustrates the trade-offs between theoretical guarantees and practical feasibility, emphasizing the need for diverse approaches to algorithm design.


\subsection{Exact vs. Heuristic Algorithms}

Exact algorithms include Shor’s and Grover’s algorithms, where speed-ups are mathematically provable and apply to well-defined computational tasks.  
Their success underscores the power of quantum computing but also highlights its reliance on fault-tolerant hardware for implementation.

Heuristic Algorithms: Noisy Intermediate-Scale Quantum (NISQ)-era algorithms are often inspired by classical techniques or physics-based intuition 
lack provable quantum advantage but offer potential for near-term applications (e.g.,  Quantum Approximate Optimization Algorithm (QAOA), 
Variational Quantum Eigensolver (VQE)).

Challenges in Quantum Advantage:

The survey emphasizes the difficulty of assessing quantum speedups for practical applications, given the need for:
End-to-end analyses of quantum algorithms.
Explicit accounting of classical pre- and post-processing.
Comparison with state-of-the-art classical algorithms.
Many proposed quantum applications remain underexplored, particularly in their integration into real-world workflows.

Resource Requirements and Fault-Tolerance:

A major focus is on the overhead introduced by fault-tolerant quantum computing, including quantum error correction and the cost of implementing non-Clifford gates.
Resource estimates for leading quantum algorithms are contextualized with these fault-tolerance considerations.

State of the Field:

The survey identifies gaps in "end-to-end" analyses for concrete applications, stressing the importance of aligning quantum capabilities with user-relevant tasks.

Opportunities and Dynamic Landscape:

It underscores the evolving interplay between quantum advancements (e.g., improved algorithms and hardware) and classical computing innovations, which continuously redefine the baseline for achieving quantum advantage.

-----
Comprehensive Classification Framework:

The text presents a structured methodology to classify quantum algorithms based on various criteria, such as the fundamental mathematical problem solved, the computational model used, and whether the algorithm is heuristic or proven.
This can directly support your project by providing a systematic way to organize and teach quantum algorithms in your syllabus.

\subsection{NISQ vs. LSQ Algorithms}

The distinction between algorithms designed for NISQ-era devices (e.g., QAOA, VQE) and those for LSQ systems (e.g., Shor’s algorithm) aligns well with your focus on the opportunities and challenges of transitioning from NISQ to LSQ.
Highlighting this distinction in your project would help students understand the limitations and potential of current quantum hardware.


The discussion on exact algorithms, such as Shor’s and Grover’s, versus heuristic algorithms, like QAOA, emphasizes the lack of provable performance guarantees for many NISQ-era algorithms.

This distinction is crucial for teaching students the theoretical underpinnings and practical considerations of using quantum algorithms for cryptographic and optimization problems.


\subsection{Sampling and Oracular Algorithms}

The examples of sampling algorithms (e.g., Boson Sampling) and oracular algorithms (e.g., Grover’s algorithm) can provide context for teaching specific types of quantum algorithms.


----


2. NISQ Era Challenges and Opportunities

NISQ Devices:
Defines the NISQ phase, emphasizing the challenges of scaling quantum computations due to noise and the lack of full error correction.
Discusses Shor’s introduction of quantum error correction, paving the way for fault-tolerant quantum computing.

Quantum Supremacy:
Highlights milestones such as Google's and IBM's achievements, while acknowledging limitations due to noise and scaling.
Discusses hybrid classical-quantum approaches as a transitional solution.

Relevance:
Directly aligns with your project's emphasis on the difference between exact and heuristic algorithms, as many NISQ-era algorithms fall into the heuristic category.
Provides context for teaching students about current hardware limitations and their impact on algorithm design and implementation.


NISQ devices are transitional quantum computers with noise and limited scalability.

Quantum supremacy claims by IBM, Xanadu, and Google.

Challenges include decoherence, error correction, and scaling qubits.

Applications in cryptography, optimization, machine learning, chemistry, and finance.

Quantum simulators as tools for testing and developing algorithms.

Hybrid classical-quantum machine learning models as emerging applications.

3. Quantum Algorithms

Discussion of Algorithms:
Covers foundational algorithms like Shor’s and Grover’s, which demonstrate provable quantum speedups, alongside heuristic approaches designed for NISQ devices.
Highlights how these algorithms leverage quantum principles like superposition and entanglement for computational advantage.

Relevance:
A valuable resource for the algorithmic portion of your lesson plan, offering examples of both exact and heuristic algorithms.
Connects well with your aim to teach students about the mathematical models underpinning algorithms and their practical applications.

4. Applications and Real-World Use Cases

Fields of Application:
Explores quantum computing’s applications in cryptography, optimization, chemistry, finance, and energy, showcasing its potential to address real-world challenges.
Includes discussions on quantum simulators and their role in prototyping algorithms.

Relevance:
Provides examples for your curriculum that tie quantum algorithms to practical use cases, particularly in cryptography and optimization.

5. Educational Utility

Contributions to Teaching:
Highlights quantum circuits, gates, and measurement as building blocks for understanding quantum algorithms.
Discusses hybrid classical-quantum machine learning models, introducing students to cutting-edge developments.

Relevance:
Useful for designing structured lessons on quantum circuits and their role in algorithm design.
Introduces advanced concepts like hybrid models, which could appeal to postgraduate students aiming to combine quantum and classical approaches.

Building Blocks of Quantum Circuits

Key Themes from Text:
Quantum gates (Hadamard, CNOT, etc.) and measurements as foundational elements.
Quantum circuits as representations of computational paths.
Examples of circuits implementing Shor’s and Grover’s algorithms.


---

Skills and Tools for Advancing Quantum Research



Insights on Quantum Algorithms and Applications for Lesson Plan Alignment
1. Foundational Algorithms

These algorithms are essential for understanding quantum computing's basic principles and advantages.
Shor’s Algorithm:

Application: Factoring large integers, computing discrete logarithms, and breaking RSA encryption.
Significance:
Exact algorithm with provable exponential speedup over classical methods.
Highlights quantum computing’s impact on cryptography and the necessity for post-quantum cryptographic methods.
Lesson Plan Focus:
Teach the mathematical foundation of Shor’s algorithm (modular arithmetic, quantum Fourier transform).
Emphasize its role in cryptography and its dependence on fault-tolerant quantum systems.

Grover’s Algorithm:

Application: Unstructured search problems, including database search and cryptographic key searches.
Significance:
Quadratic speedup over classical brute-force search methods.
Versatile subroutine for many heuristic quantum algorithms.
Lesson Plan Focus:
Cover oracle construction and amplitude amplification.
Demonstrate its practical limitations (quadratic vs. exponential speedup).

2. Heuristic Algorithms for NISQ Devices

These algorithms are adaptable for near-term quantum systems but lack provable quantum advantage.

Quantum Approximate Optimization Algorithm (QAOA):

Application: Optimization problems such as Max-Cut and scheduling.
Significance:
Designed for NISQ devices, with applications in logistics, finance, and supply chain management.
Combines quantum and classical optimization.
Lesson Plan Focus:
Teach how to encode problems into QUBO or Ising models.
Introduce parameterized quantum circuits and their optimization.

Variational Quantum Eigensolver (VQE):

Application: Quantum chemistry, material science, and energy minimization problems.
Significance:
A hybrid algorithm that computes ground-state energies of molecular systems.
Highlights the use of quantum circuits for simulations in chemistry.
Lesson Plan Focus:
Cover variational principles and their implementation in quantum circuits.
Discuss classical-quantum feedback loops and error mitigation strategies.

3. Specialized Applications

Advanced use cases showcasing quantum computing’s potential in niche fields.
Quantum Amplitude Estimation (QAE):

Application: Monte Carlo simulations, risk analysis in finance, and probability estimation.
Significance:
Builds on Grover’s algorithm to estimate amplitudes with quadratic speedup.
Central to quantum-enhanced statistical methods.
Lesson Plan Focus:
Introduce amplitude amplification techniques.
Demonstrate its use in finance (e.g., option pricing) and statistical problems.

Quantum-Enhanced Long Short-Term Memory (LSTM):

Application: Sequence learning tasks in machine learning, including time-series forecasting and language modeling.
Significance:
Hybrid model combining classical LSTMs with quantum operations.
Demonstrates quantum computing’s role in advancing AI.
Lesson Plan Focus:
Explain the integration of quantum circuits in classical machine learning models.
Discuss potential advantages and current hardware limitations.

4. Foundational Primitives and Subroutines

These are building blocks for more complex algorithms.
Quantum Phase Estimation (QPE):

Application: Subroutine for Shor’s algorithm, HHL algorithm, and QAE.
Significance:
Provides a mechanism to extract eigenvalues from unitary operations.
Enables tasks like solving linear equations and simulating quantum systems.
Lesson Plan Focus:
Teach the interplay between QPE and the quantum Fourier transform.
Use simple eigenvalue problems to demonstrate its functionality.

Quadratic Unconstrained Binary Optimization (QUBO):

Application: Optimization problems across industries, from portfolio management to logistics.
Significance:
Universal representation for combinatorial optimization problems.
Maps directly to quantum annealers and QAOA circuits.
Lesson Plan Focus:
Teach students how to encode real-world problems into QUBO format.
Explore solutions using both classical and quantum solvers.

5. Emerging Research and Real-World Impact

These applications highlight the ongoing advancements in quantum computing.
Shortest Vector Problem (SVP):

Application: Lattice-based cryptography and cryptanalysis.
Significance:
Central to post-quantum cryptography research.
Quantum algorithms like quantum annealing and QAE are being explored for solving SVP.
Lesson Plan Focus:
Introduce lattice problems and their cryptographic importance.
Discuss how quantum techniques could impact post-quantum cryptographic standards.

Substitution-Permutation Networks (SPN):

Application: Cryptanalysis of symmetric encryption algorithms like AES.
Significance:
Demonstrates quantum computing’s relevance in analyzing modern cryptographic protocols.
Lesson Plan Focus:
Teach SPN structures and quantum methods for potential attacks.
Discuss the implications for symmetric cryptographic security.


\begin{quote}\itshape
	\emph{Learning Outcomes}: Outline the specific competencies students will gain, such as understanding quantum gates, building quantum circuits, and applying algorithms to solve cryptographic challenges.
	
	\textbf{\emph{Challenges}}
	\emph{Justification for Methods}: Justifying the educational methods for a rapidly evolving field like quantum computing requires carefully citing recent pedagogical research and technological advancements.
\end{quote}\ignorespacesafterend

To thrive in this rapidly advancing field, students and researchers require a broad palette of skills and toolsets:

Foundational Knowledge: A deep understanding of quantum mechanics, linear algebra, and classical computing.
Algorithmic Proficiency: Familiarity with both exact and heuristic quantum algorithms, their mathematical underpinnings, and real-world applications.
Practical Experience: Hands-on practice with quantum hardware, simulators, and emulators to bridge theoretical concepts with implementation challenges.
Interdisciplinary Awareness: Insights from fields like cryptography, optimization, and machine learning to contextualize quantum advancements.

As the field evolves, this diverse skillset will enable researchers to navigate the complexities of quantum computing and contribute to groundbreaking discoveries. By addressing the challenges of NISQ systems and harnessing the promise of LSQ, quantum computing holds the potential to revolutionize how we solve some of the world’s most pressing computational problems.


Lesson Plan Alignment

By focusing on these algorithms and applications, the lesson plan can achieve the following:

Introduce Foundational Concepts: Teach the principles of quantum mechanics that underpin quantum algorithms (e.g., superposition, entanglement).
Develop Practical Skills: Provide hands-on experience with quantum simulators and SDKs for building circuits and implementing algorithms.
Bridge Theory and Application: Highlight real-world use cases in cryptography, optimization, and machine learning to demonstrate quantum computing’s potential.
Foster Critical Thinking: Discuss the limitations and challenges of heuristic algorithms and NISQ devices, preparing students for the evolving landscape of quantum computing.


\subsection{Survey and Comparative Analysis of Quantum Computing Systems}


%%%%%%%%%%%%%%%%%%%%%%%%%%%%%%%%%%%%%%%%%%%%%%%%%%%%%%%%%%%%%%%%%%%%%%%%%%%%%%%%
\begin{quote}\itshape
\textbf{\emph{Objective}}

To provide a comprehensive comparison of available quantum hardware, simulators, and emulators, and analyze their suitability for educational purposes. This section should also cover technical aspects like qubit types, gate mechanics, and practical limitations (e.g., thermal noise), helping students understand the real-world constraints of quantum systems.

\textbf{\emph{How to Achieve This}}

\emph{Comparative Analysis}: Review and compare various quantum computing systems (e.g., IBM Qiskit, Google Cirq, Rigetti Forest, D-Wave) and their SDKs. Focus on usability, scalability, community support, and the types of qubit systems (e.g., superconducting qubits, trapped ions).
      
\emph{Technical Details}: Discuss the mechanics of quantum gates (e.g., single-qubit and multi-qubit gates) and practical issues like decoherence and thermal noise. Explain how these factors impact quantum computation and the limitations they impose on algorithm performance.
    
\emph{Suitability for Education}: Evaluate each system’s practicality for teaching purposes. Discuss whether simulators or emulators are preferable for certain topics, and how they support students’ understanding before transitioning to physical hardware.

\textbf{\emph{Challenges}}

\emph{Technical Complexity}: Explaining concepts like thermal noise and gate fidelity can be challenging for readers new to quantum computing. Providing clear analogies and visuals may help convey these ideas effectively.
    
\emph{Keeping the Survey Current}: Quantum technology evolves quickly, so it may be difficult to ensure that the survey reflects the latest advancements. Rely on recent literature and technical resources, and acknowledge that some aspects may become outdated.
\end{quote}\ignorespacesafterend
    
%%%%%%%%%%%%%%%%%%%%%%%%%%%%%%%%%%%%%%%%%%%%%%%%%%%%%%%%%%%%%%%%%%%%%%%%%%%%%%%%
%%%%%%%%%%%%%%%%%%%%%%%%%%%%%%%%%%%%%%%%%%%%%%%%%%%%%%%%%%%%%%%%%%%%%%%%%%%%%%%%
\begin{itemize}
    \item Introduce the concept of quantum computing and its potential impact.
\end{itemize}

\subsection{Types of Quantum Systems}
\begin{itemize}
    \item \textbf{Quantum Computers}:
        \begin{itemize}
            \item Discuss different types of quantum computers:
                \begin{itemize}
                    \item \textbf{Gate-based quantum computers}: 
                        \begin{itemize}
                            \item Explain the concept of qubits and how they are used to perform computations.
                            \item Introduce examples like IBM Qiskit, Google Cirq, and Rigetti Forest.
                        \end{itemize}
                    \item \textbf{Quantum annealers}:
                        \begin{itemize}
                            \item Explain the principles behind quantum annealing and its applications.
                            \item Provide an example like D-Wave. 
                        \end{itemize}
                \end{itemize}
            \item \textbf{Focus on the types of qubit systems used in each}: 
                \begin{itemize}
                    \item Superconducting qubits.
                    \item Trapped ions.
                    \item Neutral atoms.
                    \item Other types (quantum dots spins qubits, NV centers qubits, topological qubits, NMR qubits, photon qubits).
                \end{itemize}
        \end{itemize}
    \item \textbf{Simulators}:
        \begin{itemize}
            \item Define quantum simulators and differentiate them from emulators.
            \item Explain their role in simulating quantum systems using classical computers. 
            \item Highlight their use in research and education for understanding quantum phenomena.
        \end{itemize}
    \item \textbf{Emulators}:
        \begin{itemize}
            \item Define quantum emulators and explain how they emulate the behaviour of quantum computers on classical systems.
            \item Discuss their limitations in terms of scalability (qubit number and memory capacity).
            \item Emphasise their usefulness for educational purposes:
                \begin{itemize}
                    \item Learning quantum programming.
                    \item Testing and debugging quantum algorithms. 
                    \item Visualising quantum algorithm data. 
                \end{itemize}
        \end{itemize}
    \item \textbf{Software Development Kits (SDKs)}:
        \begin{itemize}
            \item Review the SDKs associated with each quantum computing system. 
                \begin{itemize}
                    \item IBM Qiskit.
                    \item Google Cirq. 
                    \item Rigetti Forest.
                    \item D-Wave.
                \end{itemize}
            \item \textbf{Comparative Analysis:} 
                \begin{itemize}
                    \item Evaluate the usability, scalability, and community support of each SDK. 
                    \item Discuss the pros and cons of each platform for educational use. 
                \end{itemize}
        \end{itemize}
\end{itemize}

\subsection{Technical Details}
\begin{itemize}
    \item \textbf{Quantum Gates:} 
        \begin{itemize}
            \item Explain the mechanics of:
                \begin{itemize}
                    \item Single-qubit gates. 
                    \item Multi-qubit gates (e.g., CNOT gate). 
                \end{itemize}
            \item Discuss how quantum gates are used to manipulate qubits and perform quantum computations.
        \end{itemize}
    \item \textbf{Practical Limitations}:
        \begin{itemize}
            \item \textbf{Decoherence}:
                \begin{itemize}
                    \item Define decoherence and its impact on qubit stability. 
                    \item Explain how decoherence limits the time available for quantum computations. 
                \end{itemize}
            \item \textbf{Thermal Noise}:
                \begin{itemize}
                    \item Describe thermal noise and its effects on quantum systems. 
                    \item Discuss how cooling techniques are used to mitigate thermal noise. 
                \end{itemize}
            \item \textbf{Impact on Algorithm Performance}:
                \begin{itemize}
                    \item Explain how decoherence and thermal noise lead to errors in quantum computations.
                    \item Discuss how these errors impose limitations on algorithm performance and the need for error correction techniques.
                \end{itemize}
        \end{itemize}
\end{itemize}

\subsection{Suitability for Education}
\begin{itemize}
    \item \textbf{Practicality for Teaching}:
        \begin{itemize}
            \item Evaluate the suitability of each system for teaching different quantum computing concepts.
            \item Consider factors such as: 
                \begin{itemize}
                    \item Ease of use.
                    \item Availability of educational resources and tutorials.
                    \item Cost and accessibility. 
                \end{itemize}
        \end{itemize}
    \item \textbf{Simulators vs. Emulators}:
        \begin{itemize}
            \item Discuss when simulators or emulators are preferable for specific learning objectives. 
                \begin{itemize}
                    \item Use simulators to introduce fundamental quantum phenomena and concepts.
                    \item Use emulators to teach quantum programming and algorithm development. 
                \end{itemize}
        \end{itemize}
    \item \textbf{Transitioning to Physical Hardware}:
        \begin{itemize}
            \item Explain how using simulators and emulators can prepare students for working with real quantum computers. 
            \item Highlight the importance of understanding practical limitations before using physical hardware.
        \end{itemize}
\end{itemize}
%%%%%%%%%%%%%%%%%%%%%%%%%%%%%%%%%%%%%%%%%%%%%%%%%%%%%%%%%%%%%%%%%%%%%%%%%%%%%%%%


\subsection{Quantum Computing Systems}

As thermal energy can cause decoherence in many qubit technologies, increasing the noise in the probabilistic solution
results.
Because of this, manufacturers can use error correction by constructing logical qubits from several physical qubits,
increasing the accuracy of their systems.
IBM, for example in their Heron processor, used two qubit gates to achieve a 99.7\% fidelity for a total of 156
programmable qubits \cite{IBM:heronr2:2024}.  

\begin{enumerate}
\item \textbf{IBM} (USA): Holds the record for the highest number of operational qubits with the 433-qubit Osprey QPU
  \cite{IBM:ossprey:2024} based on Josephson junctions.
  As noted, they have achieved up to 99.7\% fidelities.
  Their cloud provisioned platform is the Qiskit \cite{Qiskit:2023} framework for quantum circuit construction and
  algorithm development.
\item \textbf{Google} (USA): Known for its Sycamore processor, they have developed custom electronics for qubit control.
  They provide access to their quantum computers through the Google Quantum AI platform.
\item \textbf{IQM} (Finland): Recognized as a key player in Europe, IQM focuses on building customized superconducting
  qubit QPUs.
  They achieved 99.9\% two-qubit gate fidelity and 1 millisecond coherence time in 2024.
\item \textbf{Origin Quantum} (China):  One of the largest quantum computing startups in China.
  They offer a 24-qubit superconducting system and develop various software tools, including an operating system,
  programming framework, and quantum machine learning framework.
\item \textbf{Amazon} (USA):  Provides cloud access to third-party quantum computers, including those based on
  superconducting qubits, through Amazon Braket.
\item \textbf{Alibaba} (China): Uses data centers to emulate quantum algorithms exceeding 50 qubits.
\item \textbf{QuantWare} (Netherlands):  Develops superconducting qubit chips and aims to make quantum computers more scalable.
\item \textbf{Rigetti} (USA): Offers an 84-qubit QPU.
 They have developed their own quantum programming language and software tools, including the Forest SDK.
\item \textbf{Quantinuum} (USA): A result of the merger between Honeywell Quantum Solutions and Cambridge Quantum Computing,
  Quantinuum is a major player in trapped-ion quantum computing.
 They claim to have achieved a quantum volume of 219.
 They offer access to their systems through cloud platforms and have developed software solutions for various applications.
\item \textbf{IonQ} (USA): Another major player in the trapped-ion space, IonQ offers its systems through cloud platforms.
\item \textbf{Pasqal} (France): lans to offer rack-mounted systems for their trapped-ion quantum computers.
 They have developed a software platform called Pulser for controlling neutral atom quantum processors.
\item \textbf{Alpine Quantum Technologies (AQT)} (Austria): Offers trapped-ion quantum computers integrated into
  standard 19-inch racks.
\end{enumerate}

Neutral Atoms:  Key Players: Pasqal (France), QuEra (USA), Atoms Computing (USA), and PlanQC (Germany).
 These companies use lasers to control neutral atoms as qubits.

Photon Qubits: Key Players: PsiQuantum (USA), Xanadu (Canada).
 Photon qubits are challenging to scale due to their probabilistic nature.

Quantum Dots Spins Qubits: Key Players: Intel is a major company researching quantum dots spin qubits.

NV Centers Qubits: Key Players: Quantum Brilliance (Australia) is developing room temperature NV centers QPUs.

Topological Qubits: Key Players: Microsoft is the only major company focusing on topological qubits based on Majorana fermions, though the technology is still in its early stages.


\subsection{Quantum Simulation Systems}

\begin{enumerate}
\item \textbf{D-Wave} (Canada): Quantum annealing technology.  While different from gate-based quantum computing,
  D-Wave also announced plans for gate-based systems.
\end{enumerate}


\subsection{Quantum SDKs and Emulators}

Most major quantum hardware manufacturers have also developed their own software platforms for constructing quantum
circuits and integrating with their computing infrastructure.
These platforms will typically also have quantum emulators that allow development without the expense of actually
running on an actual quantum machine.
We also include pure quantum emulators.

\begin{enumerate}
\item \textbf{IBM: Qiskit}: Includes Aer emulator
\item \textbf{Rigetti: Forest SDK}:
\item \textbf{Google: Google Quantum AI}:
\item \textbf{Microsoft: Q\# language, Azure Quantum platform}:
\item \textbf{D-Wave: Ocean SDK}:
\item \textbf{Pasqal: Pulser}:
\item \textbf{Classiq}: ISV Algorithm development
\item \textbf{QEDma Quantum Computing}: ISV Algorithm development: Classiq, 
\item \textbf{Atos QLM}: Emulator
\item \textbf{Q-CTRL}: Circuit optimization
\item \textbf{QSimulate}: Hybrid quantum-classical computing: 
\end{enumerate}


\subsection{Other Considerations for evaluation}

Back in \citeyear{Ft:Gourianov:2022} \citeauthor{Ft:Gourianov:2022}, in the Financial Times newspaper \cite{Ft:Gourianov:2022}
warned of the financial excesses of irrational exuberance due to over-optimism of the prospects of quantum computing.  

\emph{rewrite everything from this point forward}

\subsection{Classes of Quantum Algorithms}

Using \citeauthor{Arnault:2024} \citeyear{Arnault:2024} \cite{Arnault:2024} classification of mathematical classes
and application domains of quantum algorithms.

\begin{itemize}
\item \textbf{Hidden-Subgroup Problems}: This class of problems involves finding a subgroup $H$ of a group $G$ such
  that a function $f(x)$ is constant “on” each coset of $H$, meaning that, for all $g_1, g_2 \in G$, $f(g_1) = f(g_2)$
  if and only if $g_1H = g_2H$.
  Shor’s algorithm, which solves the problems of prime-numbers factorisation and of computing discrete logarithms,
  can be generalised as an HSP for finite Abelian groups.

\item \textbf{Linear Algebra}: Linear-algebra problems involve linear equations, linear transformations and their
  representations using matrices.
  For example, the Quantum Singular-Value Decomposition (QSVD) algorithm produces the singular-value decomposition of a matrix.

\item \textbf{Dynamical Systems}:  Dynamical systems are mathematical models used to describe the evolution of a
  system over time, typically governed by differential equations (e.g. an equation that is an ordinary differential
  equation with respect to time, with finite differences in space).
  Quantum algorithms that belong to this class seek to predict and analyse the system’s time evolution, e.g., by
  preparing an initial state, encoding information therein, evolving it, or evaluating key features of this evolution.
  For instance, the Quantum Lanczos algorithm computes the ground state of some Hamiltonian, which is a key feature of
  the evolution of the system described by the Hamiltonian.
  
\item \textbf{Stochastic Processes \& Statistics}:  A stochastic process is a mathematical object that describes the
  evolution of a random variable over time. Statistics is the discipline that concerns the collection, organisation,
  analysis, interpretation, and presentation of data.
  An example of an algorithm that solves a stochastic-processes problem is the Gaussian-Boson-Sampling (GBS)
  Matrix-Point Process - which, in addition, is a sampling algorithm. An example of an algorithm that solves a
  statistics problem is the Quantum k-means algorithm.

\item \textbf{Optimisation}: “Optimisation” refers to the class of problems that deal with finding the best solution
  among a given set of candidates. The “best” solution is determined by minimizing or maximising some target function
  while satisfying a set of constraints.
  Optimisation includes numerical optimisation and combinatorial optimisation problems. An example is the
  Quantum Linear Regression algorithm, whose fundamental mathematical problem is linear regression using least-squares
  optimisation, which is convex optimisation (which is a subclass of numerical optimisation).

\item \textbf{Combinatorics}: Combinatorial problems involve finding, for a discrete finite set of objects that
  satisfy some given conditions, either a grouping, or an ordering, or an assignment. They can be divided into
  three basic types: enumeration problems, existence problems, and optimisation problems.
  An algorithm classified under “Combinatorics” in the classification table focuses on solving problems like
  graph-theoretical problems, search tasks, counting tasks, etc. For example, Grover’s search algorithm has been
  extended to Quantum Counting, and both can be classified as combinatorial.
\end{itemize}

\subsection{D-Wave}

The question of whether D-Wave annealers offer a true quantum advantage over classical computing is still under
debate within the scientific community.

D-Wave's software development environment, Ocean, provides tools for hybrid algorithm development. Hybrid algorithms
combine classical computing with quantum annealing, allowing users to leverage the strengths of both approaches.

\begin{itemize}
\item \textbf{Optimisation}: D-Wave quantum annealers are specifically designed to solve optimisation problems by
  finding the global minimum energy state of a system. Many problems can be translated into quantum annealing problems
  using QUBO (Quadratic Unconstrained Binary Optimisation) or Ising problem formulations. D-Wave's most efficient tools
  are its hybrid solvers, which use both classical computing and their annealer.
  
\item \textbf{Combinatorics}: Optimisation problems are a subset of combinatorial problems, so D-Wave annealers can be
  used to solve some problems in the Combinatorics class.
  D-Wave annealers can theoretically solve NP-complete problems.

\item \textbf{Machine Learning}: D-Wave's quantum annealers can be used for machine learning tasks because they are
  well-suited to finding the minimum energy of complex systems, which is equivalent to searching for a minimum level
  of errors in the adjustment of the weights of neurons in a neural network.
  D-Wave has tested a Restricted Boltzmann Machine (RBM) model and a hybrid algorithm for image recognition using a
  variational circuit and a hybrid algorithm.
\end{itemize}
