\section{Survey and Comparative Analysis}

This survey seeks to identify quantum computing industry leaders with technology road-maps that would align with students
and researchers having confidence that the technology will evolve and be a stable platform for research and development.

\subsection{Quantum Computing Systems}

As thermal energy can cause decoherence in many qubit technologies, increasing the noise in the probabilistic solution
results.
Because of this, manufacturers can use error correction by constructing logical qubits from several physical qubits,
increasing the accuracy of their systems.
IBM, for example in their Heron processor, used two qubit gates to achieve a 99.7\% fidelity for a total of 156
programmable qubits \cite{IBM:heronr2:2024}.  

\begin{enumerate}
\item \textbf{IBM} (USA): Holds the record for the highest number of operational qubits with the 433-qubit Osprey QPU
  \cite{IBM:ossprey:2024} based on Josephson junctions.
  As noted, they have achieved up to 99.7\% fidelities.
  Their cloud provisioned platform is the Qiskit \cite{Qiskit:2023} framework for quantum circuit construction and
  algorithm development.
\item \textbf{Google} (USA): Known for its Sycamore processor, they have developed custom electronics for qubit control.
  They provide access to their quantum computers through the Google Quantum AI platform.
\item \textbf{IQM} (Finland): Recognized as a key player in Europe, IQM focuses on building customized superconducting
  qubit QPUs.
  They achieved 99.9\% two-qubit gate fidelity and 1 millisecond coherence time in 2024.
\item \textbf{Origin Quantum} (China):  One of the largest quantum computing startups in China.
  They offer a 24-qubit superconducting system and develop various software tools, including an operating system,
  programming framework, and quantum machine learning framework.
\item \textbf{Amazon} (USA):  Provides cloud access to third-party quantum computers, including those based on
  superconducting qubits, through Amazon Braket.
\item \textbf{Alibaba} (China): Uses data centers to emulate quantum algorithms exceeding 50 qubits.
\item \textbf{QuantWare} (Netherlands):  Develops superconducting qubit chips and aims to make quantum computers more scalable.
\item \textbf{Rigetti} (USA): Offers an 84-qubit QPU.
 They have developed their own quantum programming language and software tools, including the Forest SDK.
\item \textbf{Quantinuum} (USA): A result of the merger between Honeywell Quantum Solutions and Cambridge Quantum Computing,
  Quantinuum is a major player in trapped-ion quantum computing.
 They claim to have achieved a quantum volume of 219.
 They offer access to their systems through cloud platforms and have developed software solutions for various applications.
\item \textbf{IonQ} (USA): Another major player in the trapped-ion space, IonQ offers its systems through cloud platforms.
\item \textbf{Pasqal} (France): lans to offer rack-mounted systems for their trapped-ion quantum computers.
 They have developed a software platform called Pulser for controlling neutral atom quantum processors.
\item \textbf{Alpine Quantum Technologies (AQT)} (Austria): Offers trapped-ion quantum computers integrated into
  standard 19-inch racks.
\end{enumerate}

Neutral Atoms:  Key Players: Pasqal (France), QuEra (USA), Atoms Computing (USA), and PlanQC (Germany).
 These companies use lasers to control neutral atoms as qubits.

Photon Qubits: Key Players: PsiQuantum (USA), Xanadu (Canada).
 Photon qubits are challenging to scale due to their probabilistic nature.

Quantum Dots Spins Qubits: Key Players: Intel is a major company researching quantum dots spin qubits.

NV Centers Qubits: Key Players: Quantum Brilliance (Australia) is developing room temperature NV centers QPUs.

Topological Qubits: Key Players: Microsoft is the only major company focusing on topological qubits based on Majorana fermions, though the technology is still in its early stages.


\subsection{Quantum Simulation Systems}

\begin{enumerate}
\item \textbf{D-Wave} (Canada): Quantum annealing technology.  While different from gate-based quantum computing,
  D-Wave also announced plans for gate-based systems.
\end{enumerate}


\subsection{Quantum SDKs and Emulators}

Most major quantum hardware manufacturers have also developed their own software platforms for constructing quantum
circuits and integrating with their computing infrastructure.
These platforms will typically also have quantum emulators that allow development without the expense of actually
running on an actual quantum machine.
We also include pure quantum emulators.

\begin{enumerate}
\item \textbf{IBM: Qiskit}: Includes Aer emulator
\item \textbf{Rigetti: Forest SDK}:
\item \textbf{Google: Google Quantum AI}:
\item \textbf{Microsoft: Q\# language, Azure Quantum platform}:
\item \textbf{D-Wave: Ocean SDK}:
\item \textbf{Pasqal: Pulser}:
\item \textbf{Classiq}: ISV Algorithm development
\item \textbf{QEDma Quantum Computing}: ISV Algorithm development: Classiq, 
\item \textbf{Atos QLM}: Emulator
\item \textbf{Q-CTRL}: Circuit optimization
\item \textbf{QSimulate}: Hybrid quantum-classical computing: 
\end{enumerate}


\subsection{Other Considerations for evaluation}

Back in \citeyear{Ft:Gourianov:2022} \citeauthor{Ft:Gourianov:2022}, in the Financial Times newspaper \cite{Ft:Gourianov:2022}
warned of the financial excesses of irrational exuberance due to over-optimism of the prospects of quantum computing.  
