    \hypertarget{unit-2.2-bell-states-and-superdense-coding}{%
\section{Jupyter Notebook Unit 2.2 : Bell States and Superdense
Coding}\label{unit-2.2-bell-states-and-superdense-coding}}

\hypertarget{superposition}{%
\subsection*{1. Superposition}\label{superposition}}

Last week we stated that a qubit can be in the basic states
\(\lvert 0 \rangle = \begin{bmatrix} 1 \\ 0 \end{bmatrix}\) and
\(\lvert 1 \rangle = \begin{bmatrix} 0 \\ 1 \end{bmatrix}\) or in a
superposition of those states:
\(\lvert \psi\rangle = \alpha \lvert 0 \rangle + \beta \lvert 1 \rangle\).

If we apply the \emph{Hadamard} operator,
\(H = \frac{1}{\sqrt{2}} \begin{bmatrix} 1 & 1 \\ 1 & -1 \end{bmatrix}\),
to \(\lvert 0 \rangle = \begin{bmatrix} 1 \\ 0 \end{bmatrix}\), we get
the following state:

\[
\frac{1}{\sqrt{2}} \begin{bmatrix} 1 & 1 \\ 1 & -1 \end{bmatrix} \begin{bmatrix} 1 \\ 0 \end{bmatrix} = \begin{bmatrix} \frac{1}{\sqrt{2}} \\ \frac{1}{\sqrt{2}} \end{bmatrix}
\]

Probabilies are calculated by taking the sum of the squares of the
absolute values of the amplitudes, giving a 50\% chance of measuring
either 0 or 1:

\[
\lvert \frac{1}{\sqrt{2}}\lvert ^2 + \lvert \frac{1}{\sqrt{2}}\lvert ^2 = 1
\]

Creating this in QISKIT, we construct the circuit below:

    \begin{tcolorbox}[breakable, size=fbox, boxrule=1pt, pad at break*=1mm,colback=cellbackground, colframe=cellborder]
\prompt{In}{incolor}{1}{\boxspacing}
\begin{Verbatim}[commandchars=\\\{\}]
\PY{k+kn}{from} \PY{n+nn}{qiskit} \PY{k+kn}{import} \PY{n}{QuantumCircuit}
\PY{c+c1}{\PYZsh{} Create a new circuit with one qubit and add a Hadamard gate to qubit 0}
\PY{n}{qc} \PY{o}{=} \PY{n}{QuantumCircuit}\PY{p}{(}\PY{l+m+mi}{1}\PY{p}{)}
\PY{n}{qc}\PY{o}{.}\PY{n}{h}\PY{p}{(}\PY{l+m+mi}{0}\PY{p}{)}
\PY{n}{qc}\PY{o}{.}\PY{n}{measure\PYZus{}all}\PY{p}{(}\PY{p}{)}
\PY{n}{qc}\PY{o}{.}\PY{n}{draw}\PY{p}{(}\PY{l+s+s2}{\PYZdq{}}\PY{l+s+s2}{mpl}\PY{l+s+s2}{\PYZdq{}}\PY{p}{)}
\end{Verbatim}
\end{tcolorbox}
 
            
\prompt{Out}{outcolor}{1}{}
    
    \begin{center}
    \adjustimage{max size={0.9\linewidth}{0.9\paperheight}}{figures/unit_2.2_bell-states-and-superdense-coding_1_0.png}
    \end{center}
    { \hspace*{\fill} \\}
    

    Measuring the state of the qubit, we now see roughly 50\% split between
\(\lvert 0 \rangle\) and \(\lvert 1 \rangle\).

    \begin{tcolorbox}[breakable, size=fbox, boxrule=1pt, pad at break*=1mm,colback=cellbackground, colframe=cellborder]
\prompt{In}{incolor}{2}{\boxspacing}
\begin{Verbatim}[commandchars=\\\{\}]
\PY{k+kn}{from} \PY{n+nn}{qiskit\PYZus{}aer} \PY{k+kn}{import} \PY{n}{AerSimulator}
\PY{k+kn}{from} \PY{n+nn}{qiskit}\PY{n+nn}{.}\PY{n+nn}{visualization} \PY{k+kn}{import} \PY{n}{plot\PYZus{}histogram}

\PY{c+c1}{\PYZsh{} Simulator backend}
\PY{n}{backend} \PY{o}{=} \PY{n}{AerSimulator}\PY{p}{(}\PY{p}{)}

\PY{c+c1}{\PYZsh{} Run and get counts}
\PY{n}{result} \PY{o}{=} \PY{n}{backend}\PY{o}{.}\PY{n}{run}\PY{p}{(}\PY{n}{qc}\PY{p}{,} \PY{n}{shots}\PY{o}{=}\PY{l+m+mi}{100}\PY{p}{)}\PY{o}{.}\PY{n}{result}\PY{p}{(}\PY{p}{)}
\PY{n}{counts} \PY{o}{=} \PY{n}{result}\PY{o}{.}\PY{n}{get\PYZus{}counts}\PY{p}{(}\PY{n}{qc}\PY{p}{)}
\PY{n}{plot\PYZus{}histogram}\PY{p}{(}\PY{n}{counts}\PY{p}{,} \PY{n}{title}\PY{o}{=}\PY{l+s+s1}{\PYZsq{}}\PY{l+s+s1}{Superposition counts}\PY{l+s+s1}{\PYZsq{}}\PY{p}{)}
\end{Verbatim}
\end{tcolorbox}
 
            
\prompt{Out}{outcolor}{2}{}
    
    \begin{center}
    \adjustimage{max size={0.9\linewidth}{0.9\paperheight}}{figures/unit_2.2_bell-states-and-superdense-coding_3_0.png}
    \end{center}
    { \hspace*{\fill} \\}
    

    \hypertarget{entanglement-and-bell-states}{%
\subsection*{2. Entanglement and Bell
States}\label{entanglement-and-bell-states}}

John Bell proved that the \emph{spooky action at a distance} that
Einstien, Podolsky \& Rosen first pointed out, has a correlation that is
stronger that could exist in classical systems.\\
Bell's 1964 experiment proved that there was no hidden variable being
shared between the entangled qubits.

Measuring one qubit, in an entangled state, instantaneously determines
the value of the other.\\
From his experiment he showed four unique distinguishable orthogonal
entangled states that are often called the Bell states
(\(\Phi^+, \Phi^-, \Psi^+, \Psi^-\)) of the four-dimensional Hilbert
space for two qubits:

\[
\lvert \Phi^+\rangle = \frac{1}{\sqrt{2}}(\lvert 00\rangle + \lvert 11\rangle) \\
\lvert \Phi^-\rangle = \frac{1}{\sqrt{2}}(\lvert 00\rangle - \lvert 11\rangle) \\
\lvert \Psi^+\rangle = \frac{1}{\sqrt{2}}(\lvert 01\rangle + \lvert 10\rangle) \\
\lvert \Psi^-\rangle = \frac{1}{\sqrt{2}}(\lvert 01\rangle - \lvert 10\rangle)
\]

\begin{itemize}
\item
  Bell states are maximally entangled states showing quantum
  correlations.
\item
  Measuring one qubit alone gives a maximally mixed state (identity
  matrix), revealing no information.
\item
  Measuring correlated observables (e.g.,
  \(\langle X_1 X_2\rangle\)\hspace{0pt}, \(\langle Z_1 Z_2\rangle\))
  distinguishes Bell states.
\end{itemize}

    \hypertarget{how-do-we-put-two-qubits-into-one-of-these-maximally-entagled-states}{%
\subsection*{How do we put two qubits into one of these maximally
entagled
states?}\label{how-do-we-put-two-qubits-into-one-of-these-maximally-entagled-states}}

We have seen a simple circuit with a \emph{Hadamard Gate} and a
\emph{Controlled Not (CNOT)} gate.

\[
CNOT = \begin{bmatrix} 1 & 0 & 0 & 0 \\ 0 & 1 & 0 & 0 \\ 0 & 0 & 0 & 1 \\ 0 & 0  & 1 & 0 \end{bmatrix}
\]

The Hadamard operator and the CNOT gate to generate the first Bell state
\(\Phi^+\) given both qubits start in the \(\lvert 00\rangle\) initial
state.

    \begin{tcolorbox}[breakable, size=fbox, boxrule=1pt, pad at break*=1mm,colback=cellbackground, colframe=cellborder]
\prompt{In}{incolor}{3}{\boxspacing}
\begin{Verbatim}[commandchars=\\\{\}]
\PY{k+kn}{from} \PY{n+nn}{qiskit} \PY{k+kn}{import} \PY{n}{QuantumCircuit}

\PY{c+c1}{\PYZsh{} Create a new circuit with two qubits}
\PY{n}{qc\PYZus{}bell} \PY{o}{=} \PY{n}{QuantumCircuit}\PY{p}{(}\PY{l+m+mi}{2}\PY{p}{,} \PY{l+m+mi}{2}\PY{p}{)}
 
\PY{c+c1}{\PYZsh{} Add a Hadamard gate to qubit 0}
\PY{n}{qc\PYZus{}bell}\PY{o}{.}\PY{n}{h}\PY{p}{(}\PY{l+m+mi}{0}\PY{p}{)}
 
\PY{c+c1}{\PYZsh{} Perform a controlled\PYZhy{}X gate on qubit 1, controlled by qubit 0}
\PY{n}{qc\PYZus{}bell}\PY{o}{.}\PY{n}{cx}\PY{p}{(}\PY{l+m+mi}{0}\PY{p}{,} \PY{l+m+mi}{1}\PY{p}{)}
\PY{n}{qc\PYZus{}bell}\PY{o}{.}\PY{n}{measure}\PY{p}{(}\PY{p}{[}\PY{l+m+mi}{0}\PY{p}{,} \PY{l+m+mi}{1}\PY{p}{]}\PY{p}{,} \PY{p}{[}\PY{l+m+mi}{0}\PY{p}{,} \PY{l+m+mi}{1}\PY{p}{]}\PY{p}{)}
\PY{n}{qc\PYZus{}bell}\PY{o}{.}\PY{n}{draw}\PY{p}{(}\PY{l+s+s2}{\PYZdq{}}\PY{l+s+s2}{mpl}\PY{l+s+s2}{\PYZdq{}}\PY{p}{)}
\end{Verbatim}
\end{tcolorbox}
 
            
\prompt{Out}{outcolor}{3}{}
    
    \begin{center}
    \adjustimage{max size={0.9\linewidth}{0.9\paperheight}}{figures/unit_2.2_bell-states-and-superdense-coding_6_0.png}
    \end{center}
    { \hspace*{\fill} \\}
    

    \begin{tcolorbox}[breakable, size=fbox, boxrule=1pt, pad at break*=1mm,colback=cellbackground, colframe=cellborder]
\prompt{In}{incolor}{4}{\boxspacing}
\begin{Verbatim}[commandchars=\\\{\}]
\PY{c+c1}{\PYZsh{} Simulate}
\PY{n}{backend} \PY{o}{=} \PY{n}{AerSimulator}\PY{p}{(}\PY{p}{)}
\PY{n}{result} \PY{o}{=} \PY{n}{backend}\PY{o}{.}\PY{n}{run}\PY{p}{(}\PY{n}{qc\PYZus{}bell}\PY{p}{,} \PY{n}{shots}\PY{o}{=}\PY{l+m+mi}{1024}\PY{p}{)}\PY{o}{.}\PY{n}{result}\PY{p}{(}\PY{p}{)}
\PY{n}{plot\PYZus{}histogram}\PY{p}{(}\PY{n}{result}\PY{o}{.}\PY{n}{get\PYZus{}counts}\PY{p}{(}\PY{p}{)}\PY{p}{)}
\end{Verbatim}
\end{tcolorbox}
 
            
\prompt{Out}{outcolor}{4}{}
    
    \begin{center}
    \adjustimage{max size={0.9\linewidth}{0.9\paperheight}}{figures/unit_2.2_bell-states-and-superdense-coding_7_0.png}
    \end{center}
    { \hspace*{\fill} \\}
    

    \hypertarget{quantum-computations-are-reversible}{%
\subsection*{2. Quantum computations are
reversible}\label{quantum-computations-are-reversible}}

All quantum circuits - unitary transformations - are reversible, and we
can get the initial conditions from any quantum system.

If we want to recover the initial conditions, we apply the gates in the
reverse order:

    \begin{tcolorbox}[breakable, size=fbox, boxrule=1pt, pad at break*=1mm,colback=cellbackground, colframe=cellborder]
\prompt{In}{incolor}{5}{\boxspacing}
\begin{Verbatim}[commandchars=\\\{\}]
\PY{k+kn}{from} \PY{n+nn}{qiskit} \PY{k+kn}{import} \PY{n}{QuantumCircuit}
\PY{c+c1}{\PYZsh{} Create a new circuit with two qubits}
\PY{n}{qc} \PY{o}{=} \PY{n}{QuantumCircuit}\PY{p}{(}\PY{l+m+mi}{2}\PY{p}{,} \PY{l+m+mi}{2}\PY{p}{)}

\PY{c+c1}{\PYZsh{} Add a Hadamard gate to qubit 0 and  a controlled\PYZhy{}X gate on qubit 1, controlled by qubit 0}
\PY{n}{qc}\PY{o}{.}\PY{n}{h}\PY{p}{(}\PY{l+m+mi}{0}\PY{p}{)}         
\PY{n}{qc}\PY{o}{.}\PY{n}{cx}\PY{p}{(}\PY{l+m+mi}{0}\PY{p}{,} \PY{l+m+mi}{1}\PY{p}{)}     \PY{c+c1}{\PYZsh{} Create entanglement}

 \PY{c+c1}{\PYZsh{} Reverse the circuit}
\PY{n}{qc}\PY{o}{.}\PY{n}{cx}\PY{p}{(}\PY{l+m+mi}{0}\PY{p}{,} \PY{l+m+mi}{1}\PY{p}{)}
\PY{n}{qc}\PY{o}{.}\PY{n}{h}\PY{p}{(}\PY{l+m+mi}{0}\PY{p}{)}

\PY{n}{qc}\PY{o}{.}\PY{n}{measure}\PY{p}{(}\PY{p}{[}\PY{l+m+mi}{0}\PY{p}{,} \PY{l+m+mi}{1}\PY{p}{]}\PY{p}{,} \PY{p}{[}\PY{l+m+mi}{0}\PY{p}{,} \PY{l+m+mi}{1}\PY{p}{]}\PY{p}{)}
\PY{n}{qc}\PY{o}{.}\PY{n}{draw}\PY{p}{(}\PY{l+s+s2}{\PYZdq{}}\PY{l+s+s2}{mpl}\PY{l+s+s2}{\PYZdq{}}\PY{p}{)}
\end{Verbatim}
\end{tcolorbox}
 
            
\prompt{Out}{outcolor}{5}{}
    
    \begin{center}
    \adjustimage{max size={0.9\linewidth}{0.9\paperheight}}{figures/unit_2.2_bell-states-and-superdense-coding_9_0.png}
    \end{center}
    { \hspace*{\fill} \\}
    

    We end up in the initial \(\lvert 00 \rangle\) state with 100\%
probability:

    \begin{tcolorbox}[breakable, size=fbox, boxrule=1pt, pad at break*=1mm,colback=cellbackground, colframe=cellborder]
\prompt{In}{incolor}{6}{\boxspacing}
\begin{Verbatim}[commandchars=\\\{\}]
\PY{c+c1}{\PYZsh{} Simulate}
\PY{n}{backend} \PY{o}{=} \PY{n}{AerSimulator}\PY{p}{(}\PY{p}{)}
\PY{n}{result} \PY{o}{=} \PY{n}{backend}\PY{o}{.}\PY{n}{run}\PY{p}{(}\PY{n}{qc}\PY{p}{,} \PY{n}{shots}\PY{o}{=}\PY{l+m+mi}{1024}\PY{p}{)}\PY{o}{.}\PY{n}{result}\PY{p}{(}\PY{p}{)}
\PY{n}{plot\PYZus{}histogram}\PY{p}{(}\PY{n}{result}\PY{o}{.}\PY{n}{get\PYZus{}counts}\PY{p}{(}\PY{p}{)}\PY{p}{)}
\end{Verbatim}
\end{tcolorbox}
 
            
\prompt{Out}{outcolor}{6}{}
    
    \begin{center}
    \adjustimage{max size={0.9\linewidth}{0.9\paperheight}}{figures/unit_2.2_bell-states-and-superdense-coding_11_0.png}
    \end{center}
    { \hspace*{\fill} \\}
    

    If we gave Bob a pair of entangled qubits in some unknown Bell state,
all they would have to do to recover the initial conditions is to simply
apply the CNOT and Hadamard gates to the pair of qubits.

\hypertarget{superdense-encoding}{%
\subsection*{3. Superdense Encoding}\label{superdense-encoding}}

As described in \texttt{Nielsen:2000} (Nielsen \& Chuang pp97-98), it
involves two parties, Alice and Bob, who are a long way from each other.

The goal in to transmit some classical information from Alice to Bob. We
can show how Alice can send two classical bits of information to Bob by
sending only one single qubit.

In the initial setup, Alica and Bob share a pair of quibits in the
entagled state
\(\lvert \psi\rangle \;=\; \frac{\lvert 00 \rangle + \lvert 11\rangle}{\sqrt{2}}\).

Alice then takes possession of the first qubit, and Bob takes the
second.

We want to show that: 1. Alice can apply a transform to only her qubit;
1. Alice sends her qubit to Bob; 1. Bob now has an two qubits in an
unknown but maximally entangled Bell state; 1. Bob applies the CNOT and
Hadamard gates to the pair of qubits; 1. From measuring the two-qubits,
Bob can discern the state.

\hypertarget{sending-state-00}{%
\subsubsection*{Sending state 00}\label{sending-state-00}}

\begin{enumerate}
\def\labelenumi{\arabic{enumi}.}
\tightlist
\item
  Alice does nothing;
\item
  The state is: \textgreater{}
  \[\frac{1}{\sqrt{2}}(\lvert 00\rangle + \lvert 11 \rangle)\]
\end{enumerate}

    \begin{tcolorbox}[breakable, size=fbox, boxrule=1pt, pad at break*=1mm,colback=cellbackground, colframe=cellborder]
\prompt{In}{incolor}{7}{\boxspacing}
\begin{Verbatim}[commandchars=\\\{\}]
\PY{n}{qc\PYZus{}bell} \PY{o}{=} \PY{n}{QuantumCircuit}\PY{p}{(}\PY{l+m+mi}{2}\PY{p}{,} \PY{l+m+mi}{2}\PY{p}{)}
\PY{n}{qc\PYZus{}bell}\PY{o}{.}\PY{n}{h}\PY{p}{(}\PY{l+m+mi}{0}\PY{p}{)}        \PY{c+c1}{\PYZsh{} Create entanglement}
\PY{n}{qc\PYZus{}bell}\PY{o}{.}\PY{n}{cx}\PY{p}{(}\PY{l+m+mi}{0}\PY{p}{,} \PY{l+m+mi}{1}\PY{p}{)}    \PY{c+c1}{\PYZsh{} Perform a controlled\PYZhy{}X gate on qubit 1, controlled by qubit 0}

\PY{c+c1}{\PYZsh{} Bob recieves Alice\PYZsq{}s qubit and reverses the Hadamard and CNOT to reveal the message:}
\PY{n}{qc\PYZus{}bell}\PY{o}{.}\PY{n}{cx}\PY{p}{(}\PY{l+m+mi}{0}\PY{p}{,} \PY{l+m+mi}{1}\PY{p}{)}    \PY{c+c1}{\PYZsh{} Perform a controlled\PYZhy{}X gate on qubit 1, controlled by qubit 0}
\PY{n}{qc\PYZus{}bell}\PY{o}{.}\PY{n}{h}\PY{p}{(}\PY{l+m+mi}{0}\PY{p}{)}        \PY{c+c1}{\PYZsh{} Create entanglement}
\PY{n}{qc\PYZus{}bell}\PY{o}{.}\PY{n}{measure}\PY{p}{(}\PY{p}{[}\PY{l+m+mi}{0}\PY{p}{,} \PY{l+m+mi}{1}\PY{p}{]}\PY{p}{,} \PY{p}{[}\PY{l+m+mi}{0}\PY{p}{,} \PY{l+m+mi}{1}\PY{p}{]}\PY{p}{)}

\PY{c+c1}{\PYZsh{} Simulate}
\PY{n}{backend} \PY{o}{=} \PY{n}{AerSimulator}\PY{p}{(}\PY{p}{)}
\PY{n}{result} \PY{o}{=} \PY{n}{backend}\PY{o}{.}\PY{n}{run}\PY{p}{(}\PY{n}{qc\PYZus{}bell}\PY{p}{,} \PY{n}{shots}\PY{o}{=}\PY{l+m+mi}{1024}\PY{p}{)}\PY{o}{.}\PY{n}{result}\PY{p}{(}\PY{p}{)}
\PY{n}{plot\PYZus{}histogram}\PY{p}{(}\PY{n}{result}\PY{o}{.}\PY{n}{get\PYZus{}counts}\PY{p}{(}\PY{p}{)}\PY{p}{)}
\end{Verbatim}
\end{tcolorbox}
 
            
\prompt{Out}{outcolor}{7}{}
    
    \begin{center}
    \adjustimage{max size={0.9\linewidth}{0.9\paperheight}}{figures/unit_2.2_bell-states-and-superdense-coding_13_0.png}
    \end{center}
    { \hspace*{\fill} \\}
    

    \hypertarget{sending-state-01}{%
\subsubsection*{Sending state 01}\label{sending-state-01}}

Start in the joint bell state above.

\begin{enumerate}
\def\labelenumi{\arabic{enumi}.}
\tightlist
\item
  Alice applies the Pauli X operertor
  \(\begin{bmatrix} 0 & 1 \\ 1 & 0 \end{bmatrix}\).
\item
  The combined system has the state: \textgreater{}
  \[\frac{1}{\sqrt{2}}(\lvert 10\rangle + \lvert 01\rangle)\]
\end{enumerate}

    \begin{tcolorbox}[breakable, size=fbox, boxrule=1pt, pad at break*=1mm,colback=cellbackground, colframe=cellborder]
\prompt{In}{incolor}{8}{\boxspacing}
\begin{Verbatim}[commandchars=\\\{\}]
\PY{n}{qc\PYZus{}bell} \PY{o}{=} \PY{n}{QuantumCircuit}\PY{p}{(}\PY{l+m+mi}{2}\PY{p}{,} \PY{l+m+mi}{2}\PY{p}{)}
\PY{n}{qc\PYZus{}bell}\PY{o}{.}\PY{n}{h}\PY{p}{(}\PY{l+m+mi}{0}\PY{p}{)}        \PY{c+c1}{\PYZsh{} Create entanglement}
\PY{n}{qc\PYZus{}bell}\PY{o}{.}\PY{n}{cx}\PY{p}{(}\PY{l+m+mi}{0}\PY{p}{,} \PY{l+m+mi}{1}\PY{p}{)}    \PY{c+c1}{\PYZsh{} Perform a controlled\PYZhy{}X gate on qubit 1, controlled by qubit 0}

\PY{n}{qc\PYZus{}bell}\PY{o}{.}\PY{n}{x}\PY{p}{(}\PY{l+m+mi}{0}\PY{p}{)}        \PY{c+c1}{\PYZsh{} Apply the (X) operator to qubit 0 to change the Bell state}

\PY{c+c1}{\PYZsh{} Bob recieves Alice\PYZsq{}s qubit and reverses the Hadamard and CNOT to reveal the message:}
\PY{n}{qc\PYZus{}bell}\PY{o}{.}\PY{n}{cx}\PY{p}{(}\PY{l+m+mi}{0}\PY{p}{,} \PY{l+m+mi}{1}\PY{p}{)}    \PY{c+c1}{\PYZsh{} Perform a controlled\PYZhy{}X gate on qubit 1, controlled by qubit 0}
\PY{n}{qc\PYZus{}bell}\PY{o}{.}\PY{n}{h}\PY{p}{(}\PY{l+m+mi}{0}\PY{p}{)}        \PY{c+c1}{\PYZsh{} Create entanglement}
\PY{n}{qc\PYZus{}bell}\PY{o}{.}\PY{n}{measure}\PY{p}{(}\PY{p}{[}\PY{l+m+mi}{0}\PY{p}{,} \PY{l+m+mi}{1}\PY{p}{]}\PY{p}{,} \PY{p}{[}\PY{l+m+mi}{0}\PY{p}{,} \PY{l+m+mi}{1}\PY{p}{]}\PY{p}{)}
\PY{n}{qc\PYZus{}bell}\PY{o}{.}\PY{n}{draw}\PY{p}{(}\PY{l+s+s2}{\PYZdq{}}\PY{l+s+s2}{mpl}\PY{l+s+s2}{\PYZdq{}}\PY{p}{)}
\end{Verbatim}
\end{tcolorbox}
 
            
\prompt{Out}{outcolor}{8}{}
    
    \begin{center}
    \adjustimage{max size={0.9\linewidth}{0.9\paperheight}}{figures/unit_2.2_bell-states-and-superdense-coding_15_0.png}
    \end{center}
    { \hspace*{\fill} \\}
    

    \begin{tcolorbox}[breakable, size=fbox, boxrule=1pt, pad at break*=1mm,colback=cellbackground, colframe=cellborder]
\prompt{In}{incolor}{9}{\boxspacing}
\begin{Verbatim}[commandchars=\\\{\}]
\PY{c+c1}{\PYZsh{} Simulate}
\PY{n}{backend} \PY{o}{=} \PY{n}{AerSimulator}\PY{p}{(}\PY{p}{)}
\PY{n}{result} \PY{o}{=} \PY{n}{backend}\PY{o}{.}\PY{n}{run}\PY{p}{(}\PY{n}{qc\PYZus{}bell}\PY{p}{,} \PY{n}{shots}\PY{o}{=}\PY{l+m+mi}{1024}\PY{p}{)}\PY{o}{.}\PY{n}{result}\PY{p}{(}\PY{p}{)}
\PY{n}{plot\PYZus{}histogram}\PY{p}{(}\PY{n}{result}\PY{o}{.}\PY{n}{get\PYZus{}counts}\PY{p}{(}\PY{p}{)}\PY{p}{)}
\end{Verbatim}
\end{tcolorbox}
 
            
\prompt{Out}{outcolor}{9}{}
    
    \begin{center}
    \adjustimage{max size={0.9\linewidth}{0.9\paperheight}}{figures/unit_2.2_bell-states-and-superdense-coding_16_0.png}
    \end{center}
    { \hspace*{\fill} \\}
    

    \hypertarget{sending-state-10}{%
\subsubsection*{Sending state 10}\label{sending-state-10}}

\begin{enumerate}
\def\labelenumi{\arabic{enumi}.}
\tightlist
\item
  Alice applies the Z-operator
  \(\begin{bmatrix} 1 & 0 \\ 0 & -1 \end{bmatrix}\)
\item
  The state is now: \textgreater{}
  \[\frac{1}{\sqrt{2}}(\lvert 00\rangle - \lvert 11\rangle)\]
\end{enumerate}

    \begin{tcolorbox}[breakable, size=fbox, boxrule=1pt, pad at break*=1mm,colback=cellbackground, colframe=cellborder]
\prompt{In}{incolor}{10}{\boxspacing}
\begin{Verbatim}[commandchars=\\\{\}]
\PY{n}{qc\PYZus{}bell} \PY{o}{=} \PY{n}{QuantumCircuit}\PY{p}{(}\PY{l+m+mi}{2}\PY{p}{,} \PY{l+m+mi}{2}\PY{p}{)}
\PY{n}{qc\PYZus{}bell}\PY{o}{.}\PY{n}{h}\PY{p}{(}\PY{l+m+mi}{0}\PY{p}{)}        \PY{c+c1}{\PYZsh{} Create entanglement}
\PY{n}{qc\PYZus{}bell}\PY{o}{.}\PY{n}{cx}\PY{p}{(}\PY{l+m+mi}{0}\PY{p}{,} \PY{l+m+mi}{1}\PY{p}{)}    \PY{c+c1}{\PYZsh{} Perform a controlled\PYZhy{}X gate on qubit 1, controlled by qubit 0}

\PY{n}{qc\PYZus{}bell}\PY{o}{.}\PY{n}{z}\PY{p}{(}\PY{l+m+mi}{0}\PY{p}{)}        \PY{c+c1}{\PYZsh{} Apply the (Z) operator to qubit 0 to change the Bell state}

\PY{c+c1}{\PYZsh{} Bob recieves Alice\PYZsq{}s qubit and reverses the Hadamard and CNOT to reveal the message:}
\PY{n}{qc\PYZus{}bell}\PY{o}{.}\PY{n}{cx}\PY{p}{(}\PY{l+m+mi}{0}\PY{p}{,} \PY{l+m+mi}{1}\PY{p}{)}    \PY{c+c1}{\PYZsh{} Perform a controlled\PYZhy{}X gate on qubit 1, controlled by qubit 0}
\PY{n}{qc\PYZus{}bell}\PY{o}{.}\PY{n}{h}\PY{p}{(}\PY{l+m+mi}{0}\PY{p}{)}        \PY{c+c1}{\PYZsh{} Create entanglement}
\PY{n}{qc\PYZus{}bell}\PY{o}{.}\PY{n}{measure}\PY{p}{(}\PY{p}{[}\PY{l+m+mi}{0}\PY{p}{,} \PY{l+m+mi}{1}\PY{p}{]}\PY{p}{,} \PY{p}{[}\PY{l+m+mi}{0}\PY{p}{,} \PY{l+m+mi}{1}\PY{p}{]}\PY{p}{)}
\PY{n}{qc\PYZus{}bell}\PY{o}{.}\PY{n}{draw}\PY{p}{(}\PY{l+s+s2}{\PYZdq{}}\PY{l+s+s2}{mpl}\PY{l+s+s2}{\PYZdq{}}\PY{p}{)}
\end{Verbatim}
\end{tcolorbox}
 
            
\prompt{Out}{outcolor}{10}{}
    
    \begin{center}
    \adjustimage{max size={0.9\linewidth}{0.9\paperheight}}{figures/unit_2.2_bell-states-and-superdense-coding_18_0.png}
    \end{center}
    { \hspace*{\fill} \\}
    

    \begin{tcolorbox}[breakable, size=fbox, boxrule=1pt, pad at break*=1mm,colback=cellbackground, colframe=cellborder]
\prompt{In}{incolor}{11}{\boxspacing}
\begin{Verbatim}[commandchars=\\\{\}]
\PY{c+c1}{\PYZsh{} Simulate}
\PY{n}{backend} \PY{o}{=} \PY{n}{AerSimulator}\PY{p}{(}\PY{p}{)}
\PY{n}{result} \PY{o}{=} \PY{n}{backend}\PY{o}{.}\PY{n}{run}\PY{p}{(}\PY{n}{qc\PYZus{}bell}\PY{p}{,} \PY{n}{shots}\PY{o}{=}\PY{l+m+mi}{1024}\PY{p}{)}\PY{o}{.}\PY{n}{result}\PY{p}{(}\PY{p}{)}
\PY{n}{plot\PYZus{}histogram}\PY{p}{(}\PY{n}{result}\PY{o}{.}\PY{n}{get\PYZus{}counts}\PY{p}{(}\PY{p}{)}\PY{p}{)}
\end{Verbatim}
\end{tcolorbox}
 
            
\prompt{Out}{outcolor}{11}{}
    
    \begin{center}
    \adjustimage{max size={0.9\linewidth}{0.9\paperheight}}{figures/unit_2.2_bell-states-and-superdense-coding_19_0.png}
    \end{center}
    { \hspace*{\fill} \\}
    

    \hypertarget{sending-state-11}{%
\subsubsection*{Sending state 11:}\label{sending-state-11}}

\begin{enumerate}
\def\labelenumi{\arabic{enumi}.}
\tightlist
\item
  Alice applies the X-operator then the Z-operator;
\item
  The bell state is \textgreater{}
  \[\frac{1}{\sqrt{2}}(\lvert 01\rangle - \lvert 10\rangle)\]
\end{enumerate}

    \begin{tcolorbox}[breakable, size=fbox, boxrule=1pt, pad at break*=1mm,colback=cellbackground, colframe=cellborder]
\prompt{In}{incolor}{12}{\boxspacing}
\begin{Verbatim}[commandchars=\\\{\}]
\PY{n}{qc\PYZus{}bell} \PY{o}{=} \PY{n}{QuantumCircuit}\PY{p}{(}\PY{l+m+mi}{2}\PY{p}{,} \PY{l+m+mi}{2}\PY{p}{)}
\PY{n}{qc\PYZus{}bell}\PY{o}{.}\PY{n}{h}\PY{p}{(}\PY{l+m+mi}{0}\PY{p}{)}        \PY{c+c1}{\PYZsh{} Create entanglement}
\PY{n}{qc\PYZus{}bell}\PY{o}{.}\PY{n}{cx}\PY{p}{(}\PY{l+m+mi}{0}\PY{p}{,} \PY{l+m+mi}{1}\PY{p}{)}    \PY{c+c1}{\PYZsh{} Perform a controlled\PYZhy{}X gate on qubit 1, controlled by qubit 0}

\PY{n}{qc\PYZus{}bell}\PY{o}{.}\PY{n}{x}\PY{p}{(}\PY{l+m+mi}{0}\PY{p}{)}        \PY{c+c1}{\PYZsh{} Apply the (XZ) operator to qubit 0 to change the Bell state}
\PY{n}{qc\PYZus{}bell}\PY{o}{.}\PY{n}{z}\PY{p}{(}\PY{l+m+mi}{0}\PY{p}{)}

\PY{c+c1}{\PYZsh{} Bob recieves Alice\PYZsq{}s qubit and reverses the Hadamard and CNOT to reveal the message:}
\PY{n}{qc\PYZus{}bell}\PY{o}{.}\PY{n}{cx}\PY{p}{(}\PY{l+m+mi}{0}\PY{p}{,} \PY{l+m+mi}{1}\PY{p}{)}    \PY{c+c1}{\PYZsh{} Perform a controlled\PYZhy{}X gate on qubit 1, controlled by qubit 0}
\PY{n}{qc\PYZus{}bell}\PY{o}{.}\PY{n}{h}\PY{p}{(}\PY{l+m+mi}{0}\PY{p}{)}        \PY{c+c1}{\PYZsh{} Create entanglement}
\PY{n}{qc\PYZus{}bell}\PY{o}{.}\PY{n}{measure}\PY{p}{(}\PY{p}{[}\PY{l+m+mi}{0}\PY{p}{,} \PY{l+m+mi}{1}\PY{p}{]}\PY{p}{,} \PY{p}{[}\PY{l+m+mi}{0}\PY{p}{,} \PY{l+m+mi}{1}\PY{p}{]}\PY{p}{)}
\PY{n}{qc\PYZus{}bell}\PY{o}{.}\PY{n}{draw}\PY{p}{(}\PY{l+s+s2}{\PYZdq{}}\PY{l+s+s2}{mpl}\PY{l+s+s2}{\PYZdq{}}\PY{p}{)}
\end{Verbatim}
\end{tcolorbox}
 
            
\prompt{Out}{outcolor}{12}{}
    
    \begin{center}
    \adjustimage{max size={0.9\linewidth}{0.9\paperheight}}{figures/unit_2.2_bell-states-and-superdense-coding_21_0.png}
    \end{center}
    { \hspace*{\fill} \\}
    

    \begin{tcolorbox}[breakable, size=fbox, boxrule=1pt, pad at break*=1mm,colback=cellbackground, colframe=cellborder]
\prompt{In}{incolor}{13}{\boxspacing}
\begin{Verbatim}[commandchars=\\\{\}]
\PY{c+c1}{\PYZsh{} Simulate}
\PY{n}{backend} \PY{o}{=} \PY{n}{AerSimulator}\PY{p}{(}\PY{p}{)}
\PY{n}{result} \PY{o}{=} \PY{n}{backend}\PY{o}{.}\PY{n}{run}\PY{p}{(}\PY{n}{qc\PYZus{}bell}\PY{p}{,} \PY{n}{shots}\PY{o}{=}\PY{l+m+mi}{1024}\PY{p}{)}\PY{o}{.}\PY{n}{result}\PY{p}{(}\PY{p}{)}
\PY{n}{plot\PYZus{}histogram}\PY{p}{(}\PY{n}{result}\PY{o}{.}\PY{n}{get\PYZus{}counts}\PY{p}{(}\PY{p}{)}\PY{p}{)}
\end{Verbatim}
\end{tcolorbox}
 
            
\prompt{Out}{outcolor}{13}{}
    
    \begin{center}
    \adjustimage{max size={0.9\linewidth}{0.9\paperheight}}{figures/unit_2.2_bell-states-and-superdense-coding_22_0.png}
    \end{center}
    { \hspace*{\fill} \\}
    

    \begin{tcolorbox}[breakable, size=fbox, boxrule=1pt, pad at break*=1mm,colback=cellbackground, colframe=cellborder]
\prompt{In}{incolor}{14}{\boxspacing}
\begin{Verbatim}[commandchars=\\\{\}]
\PY{k+kn}{from} \PY{n+nn}{qiskit} \PY{k+kn}{import} \PY{n}{QuantumCircuit}

\PY{c+c1}{\PYZsh{} Create a new circuit with two qubits}
\PY{n}{qc} \PY{o}{=} \PY{n}{QuantumCircuit}\PY{p}{(}\PY{l+m+mi}{2}\PY{p}{,} \PY{l+m+mi}{2}\PY{p}{)}

\PY{c+c1}{\PYZsh{} Add a Hadamard gate to qubit 0 and  a controlled\PYZhy{}X gate on qubit 1, controlled by qubit 0}
\PY{n}{qc}\PY{o}{.}\PY{n}{h}\PY{p}{(}\PY{l+m+mi}{0}\PY{p}{)}
\PY{n}{qc}\PY{o}{.}\PY{n}{cx}\PY{p}{(}\PY{l+m+mi}{0}\PY{p}{,} \PY{l+m+mi}{1}\PY{p}{)}

\PY{c+c1}{\PYZsh{} Alice encodes her message}
\PY{c+c1}{\PYZsh{} 00 ; do nothing}
\PY{c+c1}{\PYZsh{} 01 : apply X}
\PY{c+c1}{\PYZsh{} 10 : apply Z}
\PY{c+c1}{\PYZsh{} 11 : apply XZ}
\PY{c+c1}{\PYZsh{}qc.x(0)}
\PY{c+c1}{\PYZsh{}qc.z(0)}

\PY{c+c1}{\PYZsh{} Reverse}
\PY{n}{qc}\PY{o}{.}\PY{n}{cx}\PY{p}{(}\PY{l+m+mi}{0}\PY{p}{,} \PY{l+m+mi}{1}\PY{p}{)}
\PY{n}{qc}\PY{o}{.}\PY{n}{h}\PY{p}{(}\PY{l+m+mi}{0}\PY{p}{)}

\PY{n}{qc}\PY{o}{.}\PY{n}{measure}\PY{p}{(}\PY{p}{[}\PY{l+m+mi}{0}\PY{p}{,} \PY{l+m+mi}{1}\PY{p}{]}\PY{p}{,} \PY{p}{[}\PY{l+m+mi}{0}\PY{p}{,} \PY{l+m+mi}{1}\PY{p}{]}\PY{p}{)}
\PY{n}{qc}\PY{o}{.}\PY{n}{draw}\PY{p}{(}\PY{l+s+s2}{\PYZdq{}}\PY{l+s+s2}{mpl}\PY{l+s+s2}{\PYZdq{}}\PY{p}{)}

\PY{c+c1}{\PYZsh{} Simulate}
\PY{n}{backend} \PY{o}{=} \PY{n}{AerSimulator}\PY{p}{(}\PY{p}{)}
\PY{n}{result} \PY{o}{=} \PY{n}{backend}\PY{o}{.}\PY{n}{run}\PY{p}{(}\PY{n}{qc}\PY{p}{,} \PY{n}{shots}\PY{o}{=}\PY{l+m+mi}{1024}\PY{p}{)}\PY{o}{.}\PY{n}{result}\PY{p}{(}\PY{p}{)}
\PY{n}{plot\PYZus{}histogram}\PY{p}{(}\PY{n}{result}\PY{o}{.}\PY{n}{get\PYZus{}counts}\PY{p}{(}\PY{p}{)}\PY{p}{)}
\end{Verbatim}
\end{tcolorbox}
 
            
\prompt{Out}{outcolor}{14}{}
    
    \begin{center}
    \adjustimage{max size={0.9\linewidth}{0.9\paperheight}}{figures/unit_2.2_bell-states-and-superdense-coding_23_0.png}
    \end{center}
    { \hspace*{\fill} \\}
    


