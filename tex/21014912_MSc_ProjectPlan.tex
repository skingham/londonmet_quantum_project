%------------------------------------------------------------------------------%
%
%
%------------------------------------------------------------------------------%

%%%%%%%%%%%%%%%%%%%%%%%%%%%%%%%%%%%%%%%%%%%%%%%%%%%%%%%%%%%%%%%%%%%%%%%%%%%%%%%%
% Document and encoding 
\documentclass[11pt,a4paper]{article}

%%%%%%%%%%%%%%%%%%%%%%%%%%%%%%%%%%%%%%%%%%%%%%%%%%%%%%%%%%%%%%%%%%%%%%%%%%%%%%%%
% Local fonts, styles, packages, and references.
\usepackage{local_doc}

\addbibresource{references.bib} 

%%%%%%%%%%%%%%%%%%%%%%%%%%%%%%%%%%%%%%%%%%%%%%%%%%%%%%%%%%%%%%%%%%%%%%%%%%%%%%%%
% Title info
\newcommand{\mytitle}{Preparing Cryptographers for the Quantum Era}
\newcommand{\mysubtitle}{Learning Outcomes and Algorithmic Frameworks for Quantum Computing Applications}
\newcommand{\myauthor}{Stuart Kingham}
\newcommand{\myid}{21014912}

\author{\myauthor: ID \myid}
\title{\mytitle}
\date{\today}

%%%%%%%%%%%%%%%%%%%%%%%%%%%%%%%%%%%%%%%%%%%%%%%%%%%%%%%%%%%%%%%%%%%%%%%%%%%%%%%
%
\begin{document}
\doublespacing

%\maketitle

%%% **Title Page**
%%% **Project Title**:
%%% **Student Name**: [Your Name]
%%% **Supervisor Name**: [Supervisor's Name]
%%% **Course Title**: MSc in [Your Course Title]
%%% **Date**: [Date of Submission]

\begin{titlepage}
  \topskip0pt
  \vspace*{\fill}
  \begin{center}
       \vspace*{1cm}

       {\LARGE CT7P01 MSc Project Proposal}

       \vspace*{1cm}
       {\large \textbf{\mytitle}}
       
       \vspace{0.2cm}
       {\large \mysubtitle}
            
       %\vspace{1.5cm}

       \vfill

       \textbf{\myauthor : ID \myid}

       \vfill
                        
       \vspace{0.8cm}
     
       %\includegraphics[width=0.4\textwidth]{university}

       Supervisor: Graham Taylor-Russel \\
       \\
       MSc in Cryptography \\
       School of Computing and Digital Media\\
       London Metropolitan University\\
       \today
            
  \end{center}
  \vspace*{\fill}
\end{titlepage}


\pagebreak

\newpage

\singlespacing 
\tableofcontents
\listoffigures
%\listoftables
\doublespacing

\pagenumbering{roman}
\pagenumbering{arabic}
\newpage

\section{Project Introduction}

This research project seeks to detail a comprehensive framework for teaching quantum computing algorithms to
graduate and postgraduate cryptography students.
I will focus on the use of Software Development Kits (SDKs) from leading quantum hardware and software companies
that are deliver world class quantum systems, emulators and quantum simulators via cloud environments.
The study seeks to identify the essential learning outcomes required for new researchers to become proficient
in constructing quantum circuits of increasing complexity.

We will perform a survey of available quantum computing platorms, evaluating them for their usability, scalability
and suitability for cryptographic applications.  We will look to implement key quantum algorithms used in
cryptanalysis (Shor, Grover, QPE, Ising-QUBO, etc.), as well as introducing the supporting mathematical models that
underpin these algorithms (Hidden-subgroup problems, combinatorics, optimizations, etc.) Further, we will look to
see examples of how researchers are applying quantum techniques to attempt to solve the Shortest Vector Problem (SVP)
or to attack Substitution–Permutation Networks (SPN) which underpin AES symmetric encryption.  These approaches, such as
Quantum Annealing and Coherent Ising Machines are claiming to show evidence of quantum advantage and are of great
interest.

The expectation of this project is to offer a practical pathway for new entrants to the field of quantum computing
to gain the skills to compete in this rapidly evolving and exciting area of research.

\subsection{High-Level Overview of Objectives}

Quantum algorithms are at the forefront of modern cryptographic research due to the challenges they pose to
traditional encryption methods.
This project focuses on developing a curriculum that introduces cryptographic students to these algorithms
using SDKs from quantum hardware manufacturers like IBM, Google, and Rigetti.

We wish to deliver a syllabus for developing quantum computing solutions using current hardware and algorithms.

This project seeks to define the essential learning outcomes for postgraduate students aiming to develop novel
applications using quantum computing.
The research will identify key quantum algorithms -- such as the Quantum Fourier Transform, quantum binary searches,
and quantum annealing -- that are crucial for solving complex problems, including integer factorization,
optimization challenges, the Shortest Vector Problem (SVP), and attacks on Substitution–Permutation Networks (SPN).

The syllabus will be designed to equip students and researchers with both theoretical knowledge and practical skills,
enabling them to work with quantum hardware from major manufacturers.
Additionally, the project aims to prepare university staff and students for participation in national quantum
computing events, such as the National Quantum Computing Centre (NQCC) Quantum Hackathon, by providing them with a
clear understanding of how to apply quantum algorithms to real-world cryptographic problems.

This project aims to develop a curriculum that teaches graduate and postgraduate cryptographic students about quantum
computing algorithms using software development kits (SDKs) from leading quantum hardware manufacturers.
The project will explore key quantum algorithms such as Shor’s and Grover’s and examine their application
in cryptography.
The methodology involves integrating quantum computing SDKs into a structured teaching program, testing their
effectiveness through practical exercises.
Key findings will assess how well students grasp the complexities of quantum algorithms and their implications
in cryptography.

The rationale for choosing this topic lies in the increasing importance of quantum computing in cryptography.
As quantum computers evolve, cryptographic students need to understand quantum algorithms to stay ahead of potential
threats to current cryptographic systems.

Our research objectives:
\begin{itemize}
\item To develop a curriculum that effectively teaches quantum computing algorithms to cryptographic students.
\item To integrate quantum SDKs into practical exercises.
\end{itemize}

Research Questions:
\begin{itemize}
\item How can SDKs be used to teach quantum algorithms in cryptography?
\item What challenges do students face in understanding quantum algorithms, and how can these be mitigated?
\end{itemize}

\subsection{Background Significance of the Study}

Quantum algorithms represent a groundbreaking shift in the field of cryptography.
As quantum computers have the potential to break widely used cryptographic protocols such as RSA and ECC (Elliptic Curve Cryptography), there is an urgent need to study these algorithms to ensure future-proof cryptographic solutions.
Teaching these algorithms to graduate and postgraduate cryptographic students equips the next generation of professionals to tackle these future challenges head-on.

In addition to quantum algorithms like Shor’s \cite{Shor:1994} and Grover’s \cite{Grover:1996}, research into quantum annealing has demonstrated its potential in solving cryptographic challenges such as the Shortest Vector Problem (SVP), which is a core problem in lattice-based cryptography.
Quantum annealing has been explored as a means of finding approximate solutions to SVP, which is essential for understanding potential vulnerabilities in post-quantum cryptographic schemes.
Moreover, there is ongoing research into using quantum annealing to attack Substitution–Permutation Network (SPN) algorithms, such as those used in AES symmetric encryption standards.
These advancements highlight the growing importance of quantum techniques in both breaking and securing cryptographic protocols.

\subsection{Relevance to Cryptography}

\citetitle{Arnault:2024}. 
Quantum computing introduces new approaches to problem-solving, particularly in cryptography.
Classical cryptography relies on computational hardness assumptions, which quantum computers threaten to undermine.
By understanding quantum algorithms, students can develop new cryptographic protocols resistant to quantum attacks, ensuring the security of digital communications, data storage, and financial systems in the post-quantum era.

\subsection{Historic Background}

The development of quantum algorithms has its roots in seminal works, particularly those by Shor (1994) and Grover (1996).
In his paper \citetitle{Shor:1994} "Algorithms for Quantum Computation: Discrete Logarithms and Factoring" Shor introduced an algorithm capable of factoring large integers and solving discrete logarithms exponentially faster than classical algorithms.
This directly challenges the security of widely used cryptographic systems, including RSA.

Shor’s algorithm builds on the earlier work of \citeauthor{Deutsch:1992} Deutsch and Jozsa, as well as Berthiaume and Brassard, who demonstrated that quantum computers could solve problems in the bounded-error probabilistic polynomial time (BPP) complexity class more efficiently than classical computers.
Shor also relied on advances in the Quantum Fourier Transform (QFT), initially developed by \citeauthor{Coppersmith:1994} Coppersmith in 1994, which is a crucial component of his factoring algorithm.

Similarly, Grover’s \cite{Grover:1996} \citetitle{Grover:1996} "A Fast Quantum Mechanical Algorithm for Database Search" (1996) presented an algorithm capable of searching an unsorted database quadratically faster than any classical algorithm.
Grover's algorithm has broad implications for data retrieval and optimization problems within cryptography.
Based on the template provided in the document, here’s a rewritten structure for your project proposal, tailored to follow the headings and subsections outlined in the template:


\section{Literature Review}

This section provides a summary of key literature in the field of quantum computing and cryptography.
Major works such as Shor’s and Grover’s algorithms will be reviewed, alongside research on the practical application of quantum computing SDKs.
Special attention will be paid to the use of quantum annealing in solving the Shortest Vector Problem (SVP) and potential attacks on AES through Substitution–Permutation Networks (SPN).
The literature will help identify gaps in existing educational approaches to teaching quantum algorithms and shape the development of the proposed curriculum.

\section{Project Management and Personal Development Planning (PDP)}

\subsection{Milestones}

\begin{itemize}

\end{itemize}

\subsection{Project Plan}

A detailed timeline and milestones are laid out for the project:

% **Phase 1**: Literature Review (Months 1-2)
% **Phase 2**: Curriculum Design (Months 3-4)
% **Phase 3**: Testing and Validation (Months 5-6)
% **Phase 4**: Dissertation Writing (Months 7-8)
% **Phase 5**: Submission and Reflection (Month 9)

\subsection{Planning and Organization}

The project is organized into manageable phases with clear objectives, milestones, and deadlines.
Progress will be regularly reviewed to ensure the project stays on track.

\subsection{Personal Development}

Reflection on skills development, such as learning new quantum SDKs, and receiving feedback from peers and mentors will be documented throughout the project.

\section{Methodology and Approach}

\subsection{Chosen Methodology}

The project will employ a practical, hands-on methodology, integrating quantum computing SDKs into the teaching modules.
A combination of lectures, tutorials, and practical exercises will help students learn the algorithms.

\subsection{Tools and Techniques}

SDKs such as IBM Qiskit, Google Cirq, and Rigetti Forest will be used to design exercises.
These tools will allow students to run quantum algorithms and experiment with quantum circuits in simulated environments.

\subsection{Evaluation}
 
The effectiveness of the curriculum will be evaluated through testing with a small group of students, with feedback collected to refine the teaching methods.

\section{Technical Level and Skills Development}

The technical complexity of the solution lies in the integration of quantum computing SDKs into cryptography education.
This section will analyze the novelty of the approach and assess how the curriculum enhances students’ understanding of both quantum algorithms and cryptography.
Personal skill development in areas like quantum programming and pedagogical strategies will also be reflected upon.

\subsection{Survey of Quantum Computing Systems and Development Environments}

On reason for excluding Quantum Key Distribution (QKD) is simply that research in this area is very dependent on access to
labs with the equipment specialised for the task.
And so our aim is to focus on quantum computing soltions that enable students and researchers access to resources that are
enabled by cloud computing and associate software development kits.


\section{Mitigants against potential fails}


%%%%%%%%%%%%%%%%%%%%%%%%%%%%%%%%%%%%%%%%%%%%%%%%%%%%%%%%%%%%%%%%%%%%%%%%%%%%%%%%
\pagebreak

\printbibliography

\pagebreak

\appendix

Any additional material that supports the proposal, such as data sheets, code listings, or detailed diagrams.

\section{Background Material}

\subsection{}



\pagebreak

\printindex

\end{document}

%%% Local Variables:
%%% mode: latex
%%% TeX-master: t
%%% End: